%!TEX root = ../report.tex

\chapter{Prediction} % (fold)
\label{chap:prediction}
\section{Introduction}
In this chapter it is presented the algorithm used for predicting the trajectory that the target object follows so that the robot can anticipate it and keep the object always on sight.
For this purpose, the algorithm chosen is a Kalman filter \cite{kalman}.

\section{Methods}
The choice of the algorithm used for predicting the trajectory is not reduced only to the Kalman filter. Actually, there is also the option of implementing an Extended Kalman filter or a particle filter. The advantage of the Extended Kalman filter over the basic one is that the first one linearizes non-linear models locally. However, the movement prediction can be approximated by the equations of a constantly accelerated movement stated in equation \ref{eq:motion}, that can be written as a linear combination of matrices as it can be seen in equation \ref{eq:motion_mat}. Therefore, there is no need of linearizing them, so the option of the Extended Kalman filter is rejected.

\begin{equation}
\begin{multlined}
p=p_{0}+v·t+\frac{1}{2}·a·t^{2} \\
v=v_{0}+a·t \\
a=a_{0} \\
\end{multlined}
\label{eq:motion}
\end{equation}

\begin{equation}
\begin{bmatrix}
x \\ y \\ z \\ v_{x} \\ v_{y} \\ v_{z} \\ a_{x} \\ a_{y} \\ a_{z}
\end{bmatrix}_{m+1}
=
\begin{bmatrix}
1 & 0 & 0 & t & 0 & 0 & \frac{1}{2}·t^{2} & 0 & 0 \\
0 & 1 & 0 & 0 & t & 0 & 0 & \frac{1}{2}·t{2} & 0 \\
0 & 0 & 1 & 0 & 0 & t & 0 & 0 & \frac{1}{2}·t^{2}	\\
0 & 0 & 0 & 1 & 0 & 0 & t & 0 & 0 \\
0 & 0 & 0 & 0 & 1 & 0 & 0 & t & 0 \\
0 & 0 & 0 & 0 & 0 & 1 & 0 & 0 & t \\
0 & 0 & 0 & 0 & 0 & 0 & 1 & 0 & 0 \\
0 & 0 & 0 & 0 & 0 & 0 & 0 & 1 & 0 \\
0 & 0 & 0 & 0 & 0 & 0 & 0 & 0 & 1
\end{bmatrix}
·
\begin{bmatrix}
x \\ y \\ z \\ v_{x} \\ v_{y} \\ v_{z} \\ a_{x} \\ a_{y} \\ a_{z}
\end{bmatrix}_{m}
\label{eq:motion_mat}
\end{equation}

On the other hand, the particle filter doesn't make much sense for this case. The main advantage about this filter is that it can use nonlinear equations \cite{kalman_lectures}, and the equations chosen for describing are linear. In addition, this filter "gives an approximate solution to an exact model" while the Kalman filter "gives an exact solution to an approximate model". Giving the fact that the Kalman filter is also much faster than the other one, it makes it the best option for this project.\\

The way the Kalman filter works is by computing an a priori estimate of the state of the object to track \cite{OReilly}, being in the case of this project its 3D position, speed and acceleration. Equation \ref{eq:prev_state} shows how this a priori estimate is computed from the previous state of the system, where $x_{k}^{-}$ and $x_{k-1}$ are equivalent to the first term of equation \ref{eq:motion_mat}, F (transfer matrix) is equivalent to the matrix of equation \ref{eq:motion_mat} and $w_{k}$ is the process noise. This last variable is usually a random variable associated with random forces that change the state of the system, which in the case of this projects are quite important as the object to track is constantly being moved randomly by a human. It is assumed that this variable has a normal distribution $N(0,Q_{k})$ for a covariance matrix $Q_{k}$.
Finally, matrix B relates the control inputs defined by $u_{k}$ to the state of the system. However, as the system has no control inputs being measured this term is removed for the sake of this project.

\begin{equation}
x_{k}^{-}=F·x_{k-1}+B·u_{k}+w_{k}
\label{eq:prev_state}
\end{equation}

Then, some measurements are made. This can be direct measurements of the state of the system or not. In the case of this project they are not, as the only variables measured by the stereo system are the x, y and z positions of the object. The relation of this measurements is shown in equation \ref{eq:measurements}, where $z_{k}^{-}$ and $H_{k}$ (measurement matrix) are shown in equation \ref{eq:measurements_mats} and $v_{k}$ is the measurement noise. As in the previous case, this variable is independent from the Kalman algorithm and depends on the noise introduced my the measurement system, which in this project is the stereo vision system. It is again supposed that this variable has a normal distribution $N(0,R_{k}$ for a covariance matrix $R_{k}$.

\begin{equation}
z_{k}^{-}=H_{k}·x_{k}+v_{k}
\label{eq:measurements}
\end{equation}

\begin{equation}
z_{k}^{-}=
\begin{bmatrix}
z_{x} \\ z_{y} \\ z_{z}
\end{bmatrix}
\qquad
H_{k}=
\begin{bmatrix}
1 & 0 & 0 \\
0 & 1 & 0 \\
0 & 0 & 1 \\
0 & 0 & 0 \\
0 & 0 & 0 \\
0 & 0 & 0 \\
0 & 0 & 0 \\
0 & 0 & 0 \\
0 & 0 & 0 \\
\end{bmatrix}
\label{eq:measurements_mats}
\end{equation}

With the a priori estimation of the state of the system and the measurements done, what is left is to compute the a priori error covariance \cite{OReilly} based on the transfer matrix, the previous error covariance and $Q_{k}$ (the process covariance). For doing so, equation \ref{eq:apriori_covariance} is applied.

\begin{equation}
P_{k}^{-}=F·P_{k-1}·F^{T}+Q_{k-1}
\label{eq:apriori_covariance}
\end{equation}

Having the a priori error covariance computed, then it is time to calculate the \emph{Kalman gain} or \emph{Bleeding factor} using equation \ref{eq:kalman_gain}. This matrix establishes how much the estimation of the state can be trusted given its error covariances. When this matrix has vary high values it means that the a priori state prediction is trustworthy and it will be included in the final state prediction. However, if this values are close to 0 it means that this prediction is not that good and the final state prediction will rely mainly in the measurements.

\begin{equation}
K_{k}=P_{k}^{-}·H_{k}^{T}·(H_{k}·P_{k}^{-}·H_{k}^{T}+R_{k})
\label{eq:kalman_gain}
\end{equation}

Finally, whenever there is a new measurement, the final state prediction $x_{k}$ is computed using equation \ref{eq:final_state} and the error covariance matrix $P_{k}$ is updated using equation \ref{eq:error_covariance}.

\begin{equation}
x_{k}=x_{k}^{-}+K_{k}·(z_{k}^{-}-H_{k}·x_{k}^{-})
\label{eq:final_state}
\end{equation}

\begin{equation}
P_{k}=(I-K_{k}·H_{k})·P_{k}^{-}
\label{eq:error_covariance}
\end{equation}

\subsection{Covariances estimation}
Kalman filter has some critique points, being one of them having a model close enough to the real system's behavior (it doesn't have to be an exact model though). Another one is giving a good estimation of the process noise covariance matrix ($Q_{k}$) and the measurement noise covariance matrix ($R_{k}$).\\

For computing the measurement noise covariance matrix an experiment was performed. First, the tracked object was kept still in front of the cameras without any human interaction to disturb it. Then, 90 outputs from the balltracker node (chapter \ref{chap:ros_nodes_structure}) are recorded into a file, consisting this points of the relative 3D position of the tracked object to the cameras. Considering that the first measured point is the correct solution\footnote{This assumption is quite brave. However, this covariance matrix is just an approximation of the real one and the algorithm corrects its own covariance matrix on the fly as stated above.}, this value is subtracted from the rest of the measurements leaving then the error. Finally, the covariance of this values is computing, yielding the values in equation \ref{eq:Rk}.

\begin{equation}
R_{k}=
\begin{bmatrix}
0.0003059 & 0.0000698 & -0.0005001 \\
0.0000698 & 0.0000990 & -0.0000384 \\
-0.0005001 & -0.0000384 & 0.0009040 \\
\end{bmatrix}
\label{eq:Rk}
\end{equation}

It can be seen that the values of the covariances are rather low, being the highest ones for the z axis, which is the one that requires more computation to be calculated. Having the exact position of the object respect to the camera instead of using the first point as the correct one would probably provide higher error covariances. However, for having that value and external measurement unit reliable enough should be used. \\

On the other hand, computing the process covariance matrix is not such a straightforward process. The reason for this is that the process being measured in this project is an object being randomly moved by a human, which implies sudden changes of direction, shacking of the object while trying to keep it still and some others... Hence, the covariances for this process have to be rather high in comparison with the measurement ones.

Nevertheless, for giving a rough approximation, a setup was prepared. In this one, a human was trying to hold the object still in front of the cameras without any external help. Meanwhile, 157 outputs from the balltracker node were recorded into a file like in the previous setup. Then, the speeds and accelerations are derived from this positions giving the time differences provided by ROS timestamps. If the process was perfect and the human was able to hold the ball perfectly still all the positions would be almost the same (only the measurement error would affect them) and the speeds and accelerations would be 0 in all axis. Therefore, assuming again that the initial position is the correct one, its value is subtracted from the rest of the positions and the covariance matrix of the whole dataset is computed, obtaining the results shown in equation \ref{eq:Qk}.

\begin{equation}
Q_{k}=
\resizebox{.8 \textwidth}{!}{$
\begin{bmatrix}
0.0003 & 0.0001 & -0.0005 & 0.0000 & -0.0000 & -0.0001 & -0.0000 & 0.0000 & 0.0000 \\
0.0001 & 0.0001 & -0.0000 & -0.0000 & 0.0000 & 0.0000 & 0.0000 & -0.0000 & -0.0001 \\
-0.0005 & -0.0000 & 0.0009 & -0.0000 & 0.0001 & 0.0002 & 0.0001 & -0.0001 & -0.0001 \\
0.0000 & -0.0000 & -0.0000 & 0.0001 & -0.0001 & -0.0003 & 0.0002 & -0.0003 & -0.0008 \\
-0.0000 & 0.0000 & 0.0001 & -0.0001 & 0.0003 & 0.0004 & -0.0002 & 0.0008 & 0.0012 \\
-0.0001 & 0.0000 & 0.0002 & -0.0003 & 0.0004 & 0.0011 & -0.0006 & 0.0014 & 0.0035 \\
-0.0000 & 0.0000 & 0.0001 & 0.0002 & -0.0002 & -0.0006 & 0.0011 & -0.0014 & -0.0046 \\
0.0000 & -0.0000 & -0.0001 & -0.0003 & 0.0008 & 0.0014 & -0.0014 & 0.0052 & 0.0087 \\
0.0000 & -0.0001 & -0.0001 & -0.0008 & 0.0012 & 0.0035 & -0.0046 & 0.0087 & 0.0241 \\
\end{bmatrix}$}
\label{eq:Qk}
\end{equation}

As it can be seen in the above matrix, the covariances are not very high values once again. Actually, the values that correspond to the covariances computed for the measurement which are shown in the first 3 columns and rows, are the same as the ones computed in equation \ref{eq:Rk}. Nonetheless, holding the object still just produces some small movements on a local point. For making this approximation more reliable the values are increased one magnitude order so that they become a more accurate representation of the randomness of moving the object freely in front of the cameras.

% chapter prediction (end)