\chapter{Calibration} % (fold)
\label{cha:calibration}

\section{Camera calibration} % (fold)
\label{sec:camera_calibration}

% section camera_calibration (end)

\section{Robot calibration} % (fold)
\label{sec:robot_calibration}
The robot's calibration has not been carried out. The reasons that support this decision are:
\begin{enumerate}
	\item For the \textbf{extrinsic calibration}, the calibration with the camera in the tool mount could have been implemented. But the lack of real and precise measurements mechanisms that would led to real calibration data, haven't been found. If founded, the experiment would have been to place a reference object with in a known position and then calculate its theoretical position based on the information from the cameras and the robot's configuration.
	\item In the case of the \textbf{intrinsic calibration}, the same reasons are given. No valid measurements tools were found for the measurements of the link's length, angles and poses in the work cell. However, if this tools were available, the process would be to measure the link's length to create a Denavit Hartenberg forward kinematics model so the measurements would have been treated easily. Sending the robot to an specified Q, the difference between the real configuration and the desired one would have lead to a model of intrinsic calibration.
\end{enumerate}

Despite the robot's calibration has not been carried out, a plot (see figure TODO) with the error between the desired configuration and the real one is shown. As it can be seen, the error is not in the order of affect determinately to our project.
% section robot_calibration (end)

% chapter calibration (end)