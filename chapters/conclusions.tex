%!TEX root = ../report.tex

\chapter{Conclusions} % (fold)
\label{cha:conclusions}
As a general conclusion, the developed system has been successfully tested in both simulation and the workcell, and subjected to experiments to measure the magnitude of the errors in the performance, as showed in Chapter \ref{cha:experiments}.
It has proved robustness and reliability, although certain limits imposed by the robot conditions have constrained the goodness of the final results. Some specific conclusions and further work are discussed at this point.

\section{Stereo vision}
On the vision part of the system, some main points can be highlighted.
The final implementation of the feature extraction algorithm for the image processing part, together with the specific design of the target have resulted in an illumination and background invariant detector, able to find reliably the target under changing environmental conditions.
Furthermore, the multithreading processing implemented to speed up the reception of the pictures acquired by the cameras, together with the use of the ROS message filters approximate sync policy to ensure their synchronization, have supposed a relevant enhancement of the system performance.
However, something that might be left as further work could be the application of the epipolar geometry theory to the correspondence problem in the detection part, in order to make it faster and more accurate as a previous step to the triangulation. 
Now two separate detection processes are carried out in each image, which work correctly, even though the epipolar constraint is not checked for the output coordinates.


% chapter conclusions (end)