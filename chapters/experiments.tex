%!TEX root = ../report.tex
\chapter{Experiments}
\label{cha:experiments}
In order to verify the theory and method whose implementation is described in the previous chapters, a number of experiments are performed.

The experiments are designed to verify or examine the following properties of the system:
\begin{itemize}
    \item The actual PA10 joint configuration versus the desired configuration.
    \item Correctness of the estimated target 3D position obtained by triangulation. %todo
    \item The error between the detected target 3D position and the Kalman filter prediction.
    \item Paths planned for obstacle avoidance in the environment.
\end{itemize}


\section{Joint positional error}
The first experiment to be carried out is to explore whether the PA10 correctly reaches the desired configurations.
Variations between the true (as reported by the PA10 controller) and desired configurations can easily affect other parts of the system.

The experiment is performed by sending a number of desired joint configurations to the PA10 controller and afterwards reading back the resulting configuration as reported by the controller node. Figure~\ref{fig:q_real_desired} shows the resulting data.

\begin{figure}[htb]
    \centering
    \resizebox{.8\columnwidth}{!}{%
        \begin{tikzpicture}[gnuplot]
%% generated with GNUPLOT 4.6p4 (Lua 5.1; terminal rev. 99, script rev. 100)
%% Wed 27 May 2015 02:02:52 AM CEST
\path (0.000,0.000) rectangle (12.500,8.750);
\gpcolor{color=gp lt color border}
\gpsetlinetype{gp lt border}
\gpsetlinewidth{1.00}
\draw[gp path] (1.504,0.985)--(1.684,0.985);
\draw[gp path] (11.947,0.985)--(11.767,0.985);
\node[gp node right,font={\fontsize{9pt}{10.8pt}\selectfont}] at (1.320,0.985) {-2};
\draw[gp path] (1.504,1.725)--(1.684,1.725);
\draw[gp path] (11.947,1.725)--(11.767,1.725);
\node[gp node right,font={\fontsize{9pt}{10.8pt}\selectfont}] at (1.320,1.725) {-1.5};
\draw[gp path] (1.504,2.464)--(1.684,2.464);
\draw[gp path] (11.947,2.464)--(11.767,2.464);
\node[gp node right,font={\fontsize{9pt}{10.8pt}\selectfont}] at (1.320,2.464) {-1};
\draw[gp path] (1.504,3.204)--(1.684,3.204);
\draw[gp path] (11.947,3.204)--(11.767,3.204);
\node[gp node right,font={\fontsize{9pt}{10.8pt}\selectfont}] at (1.320,3.204) {-0.5};
\draw[gp path] (1.504,3.943)--(1.684,3.943);
\draw[gp path] (11.947,3.943)--(11.767,3.943);
\node[gp node right,font={\fontsize{9pt}{10.8pt}\selectfont}] at (1.320,3.943) { 0};
\draw[gp path] (1.504,4.683)--(1.684,4.683);
\draw[gp path] (11.947,4.683)--(11.767,4.683);
\node[gp node right,font={\fontsize{9pt}{10.8pt}\selectfont}] at (1.320,4.683) { 0.5};
\draw[gp path] (1.504,5.423)--(1.684,5.423);
\draw[gp path] (11.947,5.423)--(11.767,5.423);
\node[gp node right,font={\fontsize{9pt}{10.8pt}\selectfont}] at (1.320,5.423) { 1};
\draw[gp path] (1.504,6.162)--(1.684,6.162);
\draw[gp path] (11.947,6.162)--(11.767,6.162);
\node[gp node right,font={\fontsize{9pt}{10.8pt}\selectfont}] at (1.320,6.162) { 1.5};
\draw[gp path] (1.504,6.902)--(1.684,6.902);
\draw[gp path] (11.947,6.902)--(11.767,6.902);
\node[gp node right,font={\fontsize{9pt}{10.8pt}\selectfont}] at (1.320,6.902) { 2};
\draw[gp path] (1.504,7.641)--(1.684,7.641);
\draw[gp path] (11.947,7.641)--(11.767,7.641);
\node[gp node right,font={\fontsize{9pt}{10.8pt}\selectfont}] at (1.320,7.641) { 2.5};
\draw[gp path] (1.504,8.381)--(1.684,8.381);
\draw[gp path] (11.947,8.381)--(11.767,8.381);
\node[gp node right,font={\fontsize{9pt}{10.8pt}\selectfont}] at (1.320,8.381) { 3};
\draw[gp path] (1.504,0.985)--(1.504,1.165);
\draw[gp path] (1.504,8.381)--(1.504,8.201);
\node[gp node center,font={\fontsize{9pt}{10.8pt}\selectfont}] at (1.504,0.677) { 0};
\draw[gp path] (2.548,0.985)--(2.548,1.165);
\draw[gp path] (2.548,8.381)--(2.548,8.201);
\node[gp node center,font={\fontsize{9pt}{10.8pt}\selectfont}] at (2.548,0.677) { 50};
\draw[gp path] (3.593,0.985)--(3.593,1.165);
\draw[gp path] (3.593,8.381)--(3.593,8.201);
\node[gp node center,font={\fontsize{9pt}{10.8pt}\selectfont}] at (3.593,0.677) { 100};
\draw[gp path] (4.637,0.985)--(4.637,1.165);
\draw[gp path] (4.637,8.381)--(4.637,8.201);
\node[gp node center,font={\fontsize{9pt}{10.8pt}\selectfont}] at (4.637,0.677) { 150};
\draw[gp path] (5.681,0.985)--(5.681,1.165);
\draw[gp path] (5.681,8.381)--(5.681,8.201);
\node[gp node center,font={\fontsize{9pt}{10.8pt}\selectfont}] at (5.681,0.677) { 200};
\draw[gp path] (6.726,0.985)--(6.726,1.165);
\draw[gp path] (6.726,8.381)--(6.726,8.201);
\node[gp node center,font={\fontsize{9pt}{10.8pt}\selectfont}] at (6.726,0.677) { 250};
\draw[gp path] (7.770,0.985)--(7.770,1.165);
\draw[gp path] (7.770,8.381)--(7.770,8.201);
\node[gp node center,font={\fontsize{9pt}{10.8pt}\selectfont}] at (7.770,0.677) { 300};
\draw[gp path] (8.814,0.985)--(8.814,1.165);
\draw[gp path] (8.814,8.381)--(8.814,8.201);
\node[gp node center,font={\fontsize{9pt}{10.8pt}\selectfont}] at (8.814,0.677) { 350};
\draw[gp path] (9.858,0.985)--(9.858,1.165);
\draw[gp path] (9.858,8.381)--(9.858,8.201);
\node[gp node center,font={\fontsize{9pt}{10.8pt}\selectfont}] at (9.858,0.677) { 400};
\draw[gp path] (10.903,0.985)--(10.903,1.165);
\draw[gp path] (10.903,8.381)--(10.903,8.201);
\node[gp node center,font={\fontsize{9pt}{10.8pt}\selectfont}] at (10.903,0.677) { 450};
\draw[gp path] (11.947,0.985)--(11.947,1.165);
\draw[gp path] (11.947,8.381)--(11.947,8.201);
\node[gp node center,font={\fontsize{9pt}{10.8pt}\selectfont}] at (11.947,0.677) { 500};
\draw[gp path] (1.504,8.381)--(1.504,0.985)--(11.947,0.985)--(11.947,8.381)--cycle;
\node[gp node center,rotate=-270,font={\fontsize{9pt}{10.8pt}\selectfont}] at (0.246,4.683) {angle (rad)};
\node[gp node center,font={\fontsize{9pt}{10.8pt}\selectfont}] at (6.725,0.215) {time (s)};
\node[gp node right,font={\fontsize{9pt}{10.8pt}\selectfont}] at (10.479,8.047) {q-real-0};
\gpcolor{color=gp lt color 0}
\gpsetlinetype{gp lt plot 0}
\draw[gp path] (10.663,8.047)--(11.579,8.047);
\draw[gp path] (1.504,5.666)--(1.525,5.666)--(1.546,5.666)--(1.567,5.666)--(1.588,5.666)%
  --(1.608,5.666)--(1.629,5.666)--(1.650,5.666)--(1.671,5.666)--(1.692,5.666)--(1.713,5.666)%
  --(1.734,5.666)--(1.755,6.056)--(1.776,6.056)--(1.796,6.056)--(1.817,6.056)--(1.838,6.056)%
  --(1.859,6.056)--(1.880,6.056)--(1.901,6.056)--(1.922,6.056)--(1.943,6.056)--(1.963,6.056)%
  --(1.984,6.056)--(2.005,6.056)--(2.026,6.220)--(2.047,6.220)--(2.068,6.220)--(2.089,6.220)%
  --(2.110,6.220)--(2.131,6.220)--(2.151,6.220)--(2.172,6.220)--(2.193,6.220)--(2.214,6.220)%
  --(2.235,6.220)--(2.256,6.220)--(2.277,6.220)--(2.298,6.220)--(2.319,6.235)--(2.339,6.235)%
  --(2.360,6.235)--(2.381,6.235)--(2.402,6.235)--(2.423,6.235)--(2.444,6.235)--(2.465,6.235)%
  --(2.486,6.235)--(2.507,6.235)--(2.527,6.235)--(2.548,6.235)--(2.569,6.235)--(2.590,6.235)%
  --(2.611,6.235)--(2.632,6.235)--(2.653,6.235)--(2.674,6.235)--(2.695,6.235)--(2.715,6.235)%
  --(2.736,6.235)--(2.757,5.881)--(2.778,5.881)--(2.799,5.881)--(2.820,5.881)--(2.841,5.881)%
  --(2.862,5.881)--(2.882,5.881)--(2.903,5.881)--(2.924,5.881)--(2.945,5.881)--(2.966,5.881)%
  --(2.987,5.881)--(3.008,5.881)--(3.029,5.881)--(3.050,5.881)--(3.070,5.881)--(3.091,5.881)%
  --(3.112,5.881)--(3.133,5.881)--(3.154,5.881)--(3.175,5.881)--(3.196,5.881)--(3.217,5.881)%
  --(3.238,5.881)--(3.258,5.881)--(3.279,5.881)--(3.300,5.881)--(3.321,5.881)--(3.342,5.881)%
  --(3.363,5.881)--(3.384,5.881)--(3.405,5.881)--(3.426,5.881)--(3.446,5.881)--(3.467,5.881)%
  --(3.488,5.881)--(3.509,5.881)--(3.530,5.881)--(3.551,5.881)--(3.572,5.881)--(3.593,5.881)%
  --(3.613,5.881)--(3.634,5.881)--(3.655,5.881)--(3.676,5.881)--(3.697,5.881)--(3.718,5.881)%
  --(3.739,5.881)--(3.760,5.881)--(3.781,5.881)--(3.801,5.881)--(3.822,5.881)--(3.843,5.881)%
  --(3.864,5.881)--(3.885,5.881)--(3.906,5.881)--(3.927,5.881)--(3.948,5.881)--(3.969,5.881)%
  --(3.989,5.881)--(4.010,5.881)--(4.031,5.881)--(4.052,5.881)--(4.073,5.881)--(4.094,5.881)%
  --(4.115,5.881)--(4.136,5.881)--(4.157,5.881)--(4.177,5.881)--(4.198,5.881)--(4.219,5.881)%
  --(4.240,5.881)--(4.261,5.881)--(4.282,5.881)--(4.303,5.881)--(4.324,5.881)--(4.344,5.881)%
  --(4.365,5.881)--(4.386,5.881)--(4.407,5.881)--(4.428,5.881)--(4.449,5.881)--(4.470,5.881)%
  --(4.491,5.881)--(4.512,5.881)--(4.532,5.881)--(4.553,5.881)--(4.574,5.881)--(4.595,5.881)%
  --(4.616,5.881)--(4.637,5.881)--(4.658,5.881)--(4.679,5.881)--(4.700,5.881)--(4.720,5.881)%
  --(4.741,5.881)--(4.762,5.881)--(4.783,5.881)--(4.804,5.881)--(4.825,5.881)--(4.846,5.881)%
  --(4.867,5.881)--(4.888,5.881)--(4.908,5.881)--(4.929,5.881)--(4.950,5.881)--(4.971,5.881)%
  --(4.992,5.881)--(5.013,5.881)--(5.034,5.881)--(5.055,5.881)--(5.076,5.881)--(5.096,5.881)%
  --(5.117,5.881)--(5.138,5.881)--(5.159,5.881)--(5.180,5.881)--(5.201,5.881)--(5.222,5.881)%
  --(5.243,5.881)--(5.263,5.881)--(5.284,5.827)--(5.305,5.827)--(5.326,5.827)--(5.347,5.827)%
  --(5.368,5.827)--(5.389,5.827)--(5.410,5.827)--(5.431,5.827)--(5.451,5.827)--(5.472,5.827)%
  --(5.493,5.827)--(5.514,5.827)--(5.535,5.827)--(5.556,5.827)--(5.577,5.827)--(5.598,5.827)%
  --(5.619,5.827)--(5.639,5.827)--(5.660,5.827)--(5.681,5.827)--(5.702,5.827)--(5.723,5.594)%
  --(5.744,5.594)--(5.765,5.594)--(5.786,5.594)--(5.807,5.594)--(5.827,5.594)--(5.848,5.594)%
  --(5.869,5.594)--(5.890,5.594)--(5.911,5.594)--(5.932,5.594)--(5.953,5.594)--(5.974,5.594)%
  --(5.994,5.594)--(6.015,5.594)--(6.036,5.594)--(6.057,5.594)--(6.078,5.411)--(6.099,5.411)%
  --(6.120,5.411)--(6.141,5.411)--(6.162,5.411)--(6.182,5.411)--(6.203,5.411)--(6.224,5.411)%
  --(6.245,5.411)--(6.266,5.411)--(6.287,5.411)--(6.308,5.411)--(6.329,5.411)--(6.350,5.411)%
  --(6.370,5.411)--(6.391,6.019)--(6.412,6.019)--(6.433,6.019)--(6.454,6.019)--(6.475,6.019)%
  --(6.496,6.019)--(6.517,6.019)--(6.538,6.019)--(6.558,6.019)--(6.579,6.019)--(6.600,6.019)%
  --(6.621,6.019)--(6.642,6.019)--(6.663,6.019)--(6.684,6.019)--(6.705,6.019)--(6.726,6.019)%
  --(6.746,6.019)--(6.767,6.019)--(6.788,6.019)--(6.809,6.019)--(6.830,6.019)--(6.851,6.019)%
  --(6.872,6.019)--(6.893,6.019)--(6.913,6.019)--(6.934,6.019)--(6.955,6.019)--(6.976,6.019)%
  --(6.997,6.019)--(7.018,6.019)--(7.039,6.019)--(7.060,6.019)--(7.081,6.019)--(7.101,6.019)%
  --(7.122,6.019)--(7.143,6.019)--(7.164,6.019)--(7.185,6.019)--(7.206,6.019)--(7.227,6.019)%
  --(7.248,6.019)--(7.269,6.019)--(7.289,6.019)--(7.310,6.019)--(7.331,6.019)--(7.352,6.019)%
  --(7.373,6.019)--(7.394,6.019)--(7.415,6.019)--(7.436,6.019)--(7.457,6.019)--(7.477,6.019)%
  --(7.498,6.019)--(7.519,6.019)--(7.540,6.019)--(7.561,6.019)--(7.582,6.019)--(7.603,6.019)%
  --(7.624,6.019)--(7.644,6.019)--(7.665,6.019)--(7.686,6.019)--(7.707,6.019)--(7.728,6.019)%
  --(7.749,6.019)--(7.770,6.019)--(7.791,6.019)--(7.812,6.019)--(7.832,6.019)--(7.853,6.019)%
  --(7.874,6.019)--(7.895,6.019)--(7.916,6.019)--(7.937,6.019)--(7.958,6.019)--(7.979,6.019)%
  --(8.000,6.019)--(8.020,6.019)--(8.041,6.019)--(8.062,6.019)--(8.083,6.019)--(8.104,6.019)%
  --(8.125,6.019)--(8.146,6.019)--(8.167,6.019)--(8.188,6.019)--(8.208,6.019)--(8.229,6.019)%
  --(8.250,6.019)--(8.271,6.019)--(8.292,6.019)--(8.313,6.019)--(8.334,6.019)--(8.355,6.019)%
  --(8.375,6.019)--(8.396,6.019)--(8.417,6.019)--(8.438,6.019)--(8.459,6.019)--(8.480,6.019)%
  --(8.501,6.019)--(8.522,6.019)--(8.543,6.019)--(8.563,6.019)--(8.584,6.019)--(8.605,6.019)%
  --(8.626,6.019)--(8.647,6.019)--(8.668,6.019)--(8.689,6.019)--(8.710,6.019)--(8.731,6.019)%
  --(8.751,6.019)--(8.772,6.019)--(8.793,6.019)--(8.814,6.019)--(8.835,6.019)--(8.856,6.019)%
  --(8.877,6.019)--(8.898,6.019)--(8.919,8.104)--(8.939,8.104)--(8.960,8.104)--(8.981,8.104)%
  --(9.002,8.104)--(9.023,8.104)--(9.044,8.104)--(9.065,8.104)--(9.086,8.104)--(9.107,8.104)%
  --(9.127,8.104)--(9.148,8.104)--(9.169,8.104)--(9.190,8.104)--(9.211,8.104)--(9.232,8.104)%
  --(9.253,8.104)--(9.274,8.104)--(9.294,8.104)--(9.315,8.104)--(9.336,8.104)--(9.357,8.104)%
  --(9.378,8.104)--(9.399,8.104)--(9.420,8.104)--(9.441,8.104)--(9.462,8.104)--(9.482,8.104)%
  --(9.503,8.104)--(9.524,8.104)--(9.545,8.104)--(9.566,8.104)--(9.587,8.104)--(9.608,8.104)%
  --(9.629,8.104)--(9.650,8.104)--(9.670,8.104)--(9.691,8.104)--(9.712,8.104)--(9.733,8.104)%
  --(9.754,8.104)--(9.775,8.104)--(9.796,8.104)--(9.817,8.104)--(9.838,8.104)--(9.858,8.104)%
  --(9.879,8.104)--(9.900,8.104)--(9.921,8.104)--(9.942,8.104)--(9.963,8.104)--(9.984,8.104)%
  --(10.005,8.104)--(10.025,8.104)--(10.046,8.104)--(10.067,8.104)--(10.088,8.104)--(10.109,8.104)%
  --(10.130,8.104)--(10.151,8.104)--(10.172,8.104)--(10.193,8.104)--(10.213,8.104)--(10.234,8.104)%
  --(10.255,8.104)--(10.276,8.104)--(10.297,8.104)--(10.318,8.104)--(10.339,8.104)--(10.360,8.104)%
  --(10.381,8.104)--(10.401,8.104)--(10.422,8.104)--(10.443,8.104)--(10.464,8.104)--(10.485,8.104)%
  --(10.506,8.104)--(10.527,8.104)--(10.548,8.104)--(10.569,8.104)--(10.589,8.104)--(10.610,8.104)%
  --(10.631,8.104)--(10.652,8.104)--(10.673,8.104)--(10.694,8.104)--(10.715,8.104)--(10.736,8.104)%
  --(10.756,8.104)--(10.777,8.104)--(10.798,8.104)--(10.819,8.104)--(10.840,8.104)--(10.861,8.104)%
  --(10.882,8.104)--(10.903,8.104)--(10.924,8.104)--(10.944,8.104)--(10.965,8.104)--(10.986,8.104)%
  --(11.007,8.104)--(11.028,8.104);
\gpcolor{color=gp lt color border}
\node[gp node right,font={\fontsize{9pt}{10.8pt}\selectfont}] at (10.479,7.739) {q-desired-0};
\gpcolor{color=gp lt color 1}
\gpsetlinetype{gp lt plot 1}
\draw[gp path] (10.663,7.739)--(11.579,7.739);
\draw[gp path] (1.504,5.668)--(1.525,5.668)--(1.546,5.668)--(1.567,5.668)--(1.588,5.668)%
  --(1.608,5.668)--(1.629,5.668)--(1.650,5.668)--(1.671,5.668)--(1.692,5.668)--(1.713,5.668)%
  --(1.734,6.205)--(1.755,6.205)--(1.776,6.205)--(1.796,6.205)--(1.817,6.205)--(1.838,6.205)%
  --(1.859,6.205)--(1.880,6.205)--(1.901,6.205)--(1.922,6.205)--(1.943,6.205)--(1.963,6.205)%
  --(1.984,6.205)--(2.005,6.221)--(2.026,6.221)--(2.047,6.221)--(2.068,6.221)--(2.089,6.221)%
  --(2.110,6.221)--(2.131,6.221)--(2.151,6.221)--(2.172,6.221)--(2.193,6.221)--(2.214,6.221)%
  --(2.235,6.221)--(2.256,6.221)--(2.277,6.221)--(2.298,6.221)--(2.319,6.237)--(2.339,6.237)%
  --(2.360,6.237)--(2.381,6.237)--(2.402,6.237)--(2.423,6.237)--(2.444,6.237)--(2.465,6.237)%
  --(2.486,6.237)--(2.507,6.237)--(2.527,6.237)--(2.548,6.237)--(2.569,6.237)--(2.590,6.237)%
  --(2.611,6.237)--(2.632,6.237)--(2.653,6.237)--(2.674,6.237)--(2.695,6.237)--(2.715,6.237)%
  --(2.736,6.237)--(2.757,5.611)--(2.778,5.611)--(2.799,5.611)--(2.820,5.611)--(2.841,5.611)%
  --(2.862,5.611)--(2.882,5.611)--(2.903,5.611)--(2.924,5.611)--(2.945,5.611)--(2.966,5.611)%
  --(2.987,5.611)--(3.008,5.611)--(3.029,5.611)--(3.050,5.611)--(3.070,5.611)--(3.091,5.611)%
  --(3.112,5.611)--(3.133,5.611)--(3.154,5.611)--(3.175,5.611)--(3.196,5.611)--(3.217,5.611)%
  --(3.238,5.611)--(3.258,5.611)--(3.279,5.611)--(3.300,5.611)--(3.321,5.611)--(3.342,5.611)%
  --(3.363,5.611)--(3.384,5.611)--(3.405,5.611)--(3.426,5.611)--(3.446,5.611)--(3.467,5.611)%
  --(3.488,5.611)--(3.509,5.611)--(3.530,5.611)--(3.551,5.611)--(3.572,5.611)--(3.593,5.611)%
  --(3.613,5.611)--(3.634,5.611)--(3.655,5.611)--(3.676,5.611)--(3.697,5.611)--(3.718,5.611)%
  --(3.739,5.611)--(3.760,5.611)--(3.781,5.611)--(3.801,5.611)--(3.822,5.611)--(3.843,5.611)%
  --(3.864,5.611)--(3.885,5.611)--(3.906,5.611)--(3.927,5.611)--(3.948,5.611)--(3.969,5.611)%
  --(3.989,5.611)--(4.010,5.611)--(4.031,5.611)--(4.052,5.611)--(4.073,5.611)--(4.094,5.611)%
  --(4.115,5.611)--(4.136,5.611)--(4.157,5.611)--(4.177,5.611)--(4.198,5.611)--(4.219,5.611)%
  --(4.240,5.611)--(4.261,5.611)--(4.282,5.611)--(4.303,5.611)--(4.324,5.611)--(4.344,5.611)%
  --(4.365,5.611)--(4.386,5.611)--(4.407,5.611)--(4.428,5.611)--(4.449,5.611)--(4.470,5.611)%
  --(4.491,5.611)--(4.512,5.611)--(4.532,5.611)--(4.553,5.611)--(4.574,5.611)--(4.595,5.611)%
  --(4.616,5.611)--(4.637,5.611)--(4.658,5.611)--(4.679,5.611)--(4.700,5.611)--(4.720,5.611)%
  --(4.741,5.611)--(4.762,5.611)--(4.783,5.611)--(4.804,5.611)--(4.825,5.611)--(4.846,5.611)%
  --(4.867,5.611)--(4.888,5.611)--(4.908,5.611)--(4.929,5.611)--(4.950,5.611)--(4.971,5.611)%
  --(4.992,5.611)--(5.013,5.611)--(5.034,5.611)--(5.055,5.611)--(5.076,5.611)--(5.096,5.611)%
  --(5.117,5.611)--(5.138,5.611)--(5.159,5.611)--(5.180,5.611)--(5.201,5.611)--(5.222,5.611)%
  --(5.243,5.611)--(5.263,5.857)--(5.284,5.857)--(5.305,5.857)--(5.326,5.857)--(5.347,5.857)%
  --(5.368,5.857)--(5.389,5.857)--(5.410,5.857)--(5.431,5.857)--(5.451,5.857)--(5.472,5.857)%
  --(5.493,5.857)--(5.514,5.857)--(5.535,5.857)--(5.556,5.857)--(5.577,5.857)--(5.598,5.857)%
  --(5.619,5.857)--(5.639,5.857)--(5.660,5.857)--(5.681,5.857)--(5.702,5.516)--(5.723,5.516)%
  --(5.744,5.516)--(5.765,5.516)--(5.786,5.516)--(5.807,5.516)--(5.827,5.516)--(5.848,5.516)%
  --(5.869,5.516)--(5.890,5.516)--(5.911,5.516)--(5.932,5.516)--(5.953,5.516)--(5.974,5.516)%
  --(5.994,5.516)--(6.015,5.516)--(6.036,5.516)--(6.057,5.405)--(6.078,5.405)--(6.099,5.405)%
  --(6.120,5.405)--(6.141,5.405)--(6.162,5.405)--(6.182,5.405)--(6.203,5.405)--(6.224,5.405)%
  --(6.245,5.405)--(6.266,5.405)--(6.287,5.405)--(6.308,5.405)--(6.329,5.405)--(6.350,5.405)%
  --(6.370,5.405)--(6.391,8.243)--(6.412,8.243)--(6.433,8.243)--(6.454,8.243)--(6.475,8.243)%
  --(6.496,8.243)--(6.517,8.243)--(6.538,8.243)--(6.558,8.243)--(6.579,8.243)--(6.600,8.243)%
  --(6.621,8.243)--(6.642,8.243)--(6.663,8.243)--(6.684,8.243)--(6.705,8.243)--(6.726,8.243)%
  --(6.746,8.243)--(6.767,8.243)--(6.788,8.243)--(6.809,8.243)--(6.830,8.243)--(6.851,8.243)%
  --(6.872,8.243)--(6.893,8.243)--(6.913,8.243)--(6.934,8.243)--(6.955,8.243)--(6.976,8.243)%
  --(6.997,8.243)--(7.018,8.243)--(7.039,8.243)--(7.060,8.243)--(7.081,8.243)--(7.101,8.243)%
  --(7.122,8.243)--(7.143,8.243)--(7.164,8.243)--(7.185,8.243)--(7.206,8.243)--(7.227,8.243)%
  --(7.248,8.243)--(7.269,8.243)--(7.289,8.243)--(7.310,8.243)--(7.331,8.243)--(7.352,8.243)%
  --(7.373,8.243)--(7.394,8.243)--(7.415,8.243)--(7.436,8.243)--(7.457,8.243)--(7.477,8.243)%
  --(7.498,8.243)--(7.519,8.243)--(7.540,8.243)--(7.561,8.243)--(7.582,8.243)--(7.603,8.243)%
  --(7.624,8.243)--(7.644,8.243)--(7.665,8.243)--(7.686,8.243)--(7.707,8.243)--(7.728,8.243)%
  --(7.749,8.243)--(7.770,8.243)--(7.791,8.243)--(7.812,8.243)--(7.832,8.243)--(7.853,8.243)%
  --(7.874,8.243)--(7.895,8.243)--(7.916,8.243)--(7.937,8.243)--(7.958,8.243)--(7.979,8.243)%
  --(8.000,8.243)--(8.020,8.243)--(8.041,8.243)--(8.062,8.243)--(8.083,8.243)--(8.104,8.243)%
  --(8.125,8.243)--(8.146,8.243)--(8.167,8.243)--(8.188,8.243)--(8.208,8.243)--(8.229,8.243)%
  --(8.250,8.243)--(8.271,8.243)--(8.292,8.243)--(8.313,8.243)--(8.334,8.243)--(8.355,8.243)%
  --(8.375,8.243)--(8.396,8.243)--(8.417,8.243)--(8.438,8.243)--(8.459,8.243)--(8.480,8.243)%
  --(8.501,8.243)--(8.522,8.243)--(8.543,8.243)--(8.563,8.243)--(8.584,8.243)--(8.605,8.243)%
  --(8.626,8.243)--(8.647,8.243)--(8.668,8.243)--(8.689,8.243)--(8.710,8.243)--(8.731,8.243)%
  --(8.751,8.243)--(8.772,8.243)--(8.793,8.243)--(8.814,8.243)--(8.835,8.243)--(8.856,8.243)%
  --(8.877,8.243)--(8.898,8.243)--(8.919,8.093)--(8.939,8.093)--(8.960,8.093)--(8.981,8.093)%
  --(9.002,8.093)--(9.023,8.093)--(9.044,8.093)--(9.065,8.093)--(9.086,8.093)--(9.107,8.093)%
  --(9.127,8.093)--(9.148,8.093)--(9.169,8.093)--(9.190,8.093)--(9.211,8.093)--(9.232,8.093)%
  --(9.253,8.093)--(9.274,8.093)--(9.294,8.093)--(9.315,8.093)--(9.336,8.093)--(9.357,8.093)%
  --(9.378,8.093)--(9.399,8.093)--(9.420,8.093)--(9.441,8.093)--(9.462,8.093)--(9.482,8.093)%
  --(9.503,8.093)--(9.524,8.093)--(9.545,8.093)--(9.566,8.093)--(9.587,8.093)--(9.608,8.093)%
  --(9.629,8.093)--(9.650,8.093)--(9.670,8.093)--(9.691,8.093)--(9.712,8.093)--(9.733,8.093)%
  --(9.754,8.093)--(9.775,8.093)--(9.796,8.093)--(9.817,8.093)--(9.838,8.093)--(9.858,8.093)%
  --(9.879,8.093)--(9.900,8.093)--(9.921,8.093)--(9.942,8.093)--(9.963,8.093)--(9.984,8.093)%
  --(10.005,8.093)--(10.025,8.093)--(10.046,8.093)--(10.067,8.093)--(10.088,8.093)--(10.109,8.093)%
  --(10.130,8.093)--(10.151,8.093)--(10.172,8.093)--(10.193,8.093)--(10.213,8.093)--(10.234,8.093)%
  --(10.255,8.093)--(10.276,8.093)--(10.297,8.093)--(10.318,8.093)--(10.339,8.093)--(10.360,8.093)%
  --(10.381,8.093)--(10.401,8.093)--(10.422,8.093)--(10.443,8.093)--(10.464,8.093)--(10.485,8.093)%
  --(10.506,8.093)--(10.527,8.093)--(10.548,8.093)--(10.569,8.093)--(10.589,8.093)--(10.610,8.093)%
  --(10.631,8.093)--(10.652,8.093)--(10.673,8.093)--(10.694,8.093)--(10.715,8.093)--(10.736,8.093)%
  --(10.756,8.093)--(10.777,8.093)--(10.798,8.093)--(10.819,8.093)--(10.840,8.093)--(10.861,8.093)%
  --(10.882,8.093)--(10.903,8.093)--(10.924,8.093)--(10.944,8.093)--(10.965,8.093)--(10.986,8.093)%
  --(11.007,8.093)--(11.028,8.093);
\gpcolor{color=gp lt color border}
\node[gp node right,font={\fontsize{9pt}{10.8pt}\selectfont}] at (10.479,7.431) {q-real-1};
\gpcolor{color=gp lt color 2}
\gpsetlinetype{gp lt plot 2}
\draw[gp path] (10.663,7.431)--(11.579,7.431);
\draw[gp path] (1.504,2.570)--(1.525,2.570)--(1.546,2.570)--(1.567,2.570)--(1.588,2.570)%
  --(1.608,2.570)--(1.629,2.570)--(1.650,2.570)--(1.671,2.570)--(1.692,2.570)--(1.713,2.570)%
  --(1.734,2.570)--(1.755,3.050)--(1.776,3.050)--(1.796,3.050)--(1.817,3.050)--(1.838,3.050)%
  --(1.859,3.050)--(1.880,3.050)--(1.901,3.050)--(1.922,3.050)--(1.943,3.050)--(1.963,3.050)%
  --(1.984,3.050)--(2.005,3.050)--(2.026,3.543)--(2.047,3.543)--(2.068,3.543)--(2.089,3.543)%
  --(2.110,3.543)--(2.131,3.543)--(2.151,3.543)--(2.172,3.543)--(2.193,3.543)--(2.214,3.543)%
  --(2.235,3.543)--(2.256,3.543)--(2.277,3.543)--(2.298,3.543)--(2.319,3.795)--(2.339,3.795)%
  --(2.360,3.795)--(2.381,3.795)--(2.402,3.795)--(2.423,3.795)--(2.444,3.795)--(2.465,3.795)%
  --(2.486,3.795)--(2.507,3.795)--(2.527,3.795)--(2.548,3.795)--(2.569,3.795)--(2.590,3.795)%
  --(2.611,3.795)--(2.632,3.795)--(2.653,3.795)--(2.674,3.795)--(2.695,3.795)--(2.715,3.795)%
  --(2.736,3.795)--(2.757,4.259)--(2.778,4.259)--(2.799,4.259)--(2.820,4.259)--(2.841,4.259)%
  --(2.862,4.259)--(2.882,4.259)--(2.903,4.259)--(2.924,4.259)--(2.945,4.259)--(2.966,4.259)%
  --(2.987,4.259)--(3.008,4.259)--(3.029,4.259)--(3.050,4.259)--(3.070,4.259)--(3.091,4.259)%
  --(3.112,4.259)--(3.133,4.259)--(3.154,4.259)--(3.175,4.259)--(3.196,4.259)--(3.217,4.259)%
  --(3.238,4.259)--(3.258,4.259)--(3.279,4.259)--(3.300,4.259)--(3.321,4.259)--(3.342,4.259)%
  --(3.363,4.259)--(3.384,4.259)--(3.405,4.259)--(3.426,4.259)--(3.446,4.259)--(3.467,4.259)%
  --(3.488,4.259)--(3.509,4.259)--(3.530,4.259)--(3.551,4.259)--(3.572,4.259)--(3.593,4.259)%
  --(3.613,4.259)--(3.634,4.259)--(3.655,4.259)--(3.676,4.259)--(3.697,4.259)--(3.718,4.259)%
  --(3.739,4.259)--(3.760,4.259)--(3.781,4.259)--(3.801,4.259)--(3.822,4.259)--(3.843,4.259)%
  --(3.864,4.259)--(3.885,4.259)--(3.906,4.259)--(3.927,4.259)--(3.948,4.259)--(3.969,4.259)%
  --(3.989,4.259)--(4.010,4.259)--(4.031,4.259)--(4.052,4.259)--(4.073,4.259)--(4.094,4.259)%
  --(4.115,4.259)--(4.136,4.259)--(4.157,4.259)--(4.177,4.259)--(4.198,4.259)--(4.219,4.259)%
  --(4.240,4.259)--(4.261,4.259)--(4.282,4.259)--(4.303,4.259)--(4.324,4.259)--(4.344,4.259)%
  --(4.365,4.259)--(4.386,4.259)--(4.407,4.259)--(4.428,4.259)--(4.449,4.259)--(4.470,4.259)%
  --(4.491,4.259)--(4.512,4.259)--(4.532,4.259)--(4.553,4.259)--(4.574,4.259)--(4.595,4.259)%
  --(4.616,4.259)--(4.637,4.259)--(4.658,4.259)--(4.679,4.259)--(4.700,4.259)--(4.720,4.259)%
  --(4.741,4.259)--(4.762,4.259)--(4.783,4.259)--(4.804,4.259)--(4.825,4.259)--(4.846,4.259)%
  --(4.867,4.259)--(4.888,4.259)--(4.908,4.259)--(4.929,4.259)--(4.950,4.259)--(4.971,4.259)%
  --(4.992,4.259)--(5.013,4.259)--(5.034,4.259)--(5.055,4.259)--(5.076,4.259)--(5.096,4.259)%
  --(5.117,4.259)--(5.138,4.259)--(5.159,4.259)--(5.180,4.259)--(5.201,4.259)--(5.222,4.259)%
  --(5.243,4.259)--(5.263,4.259)--(5.284,4.503)--(5.305,4.503)--(5.326,4.503)--(5.347,4.503)%
  --(5.368,4.503)--(5.389,4.503)--(5.410,4.503)--(5.431,4.503)--(5.451,4.503)--(5.472,4.503)%
  --(5.493,4.503)--(5.514,4.503)--(5.535,4.503)--(5.556,4.503)--(5.577,4.503)--(5.598,4.503)%
  --(5.619,4.503)--(5.639,4.503)--(5.660,4.503)--(5.681,4.503)--(5.702,4.503)--(5.723,4.651)%
  --(5.744,4.651)--(5.765,4.651)--(5.786,4.651)--(5.807,4.651)--(5.827,4.651)--(5.848,4.651)%
  --(5.869,4.651)--(5.890,4.651)--(5.911,4.651)--(5.932,4.651)--(5.953,4.651)--(5.974,4.651)%
  --(5.994,4.651)--(6.015,4.651)--(6.036,4.651)--(6.057,4.651)--(6.078,4.523)--(6.099,4.523)%
  --(6.120,4.523)--(6.141,4.523)--(6.162,4.523)--(6.182,4.523)--(6.203,4.523)--(6.224,4.523)%
  --(6.245,4.523)--(6.266,4.523)--(6.287,4.523)--(6.308,4.523)--(6.329,4.523)--(6.350,4.523)%
  --(6.370,4.523)--(6.391,4.311)--(6.412,4.311)--(6.433,4.311)--(6.454,4.311)--(6.475,4.311)%
  --(6.496,4.311)--(6.517,4.311)--(6.538,4.311)--(6.558,4.311)--(6.579,4.311)--(6.600,4.311)%
  --(6.621,4.311)--(6.642,4.311)--(6.663,4.311)--(6.684,4.311)--(6.705,4.311)--(6.726,4.311)%
  --(6.746,4.311)--(6.767,4.311)--(6.788,4.311)--(6.809,4.311)--(6.830,4.311)--(6.851,4.311)%
  --(6.872,4.311)--(6.893,4.311)--(6.913,4.311)--(6.934,4.311)--(6.955,4.311)--(6.976,4.311)%
  --(6.997,4.311)--(7.018,4.311)--(7.039,4.311)--(7.060,4.311)--(7.081,4.311)--(7.101,4.311)%
  --(7.122,4.311)--(7.143,4.311)--(7.164,4.311)--(7.185,4.311)--(7.206,4.311)--(7.227,4.311)%
  --(7.248,4.311)--(7.269,4.311)--(7.289,4.311)--(7.310,4.311)--(7.331,4.311)--(7.352,4.311)%
  --(7.373,4.311)--(7.394,4.311)--(7.415,4.311)--(7.436,4.311)--(7.457,4.311)--(7.477,4.311)%
  --(7.498,4.311)--(7.519,4.311)--(7.540,4.311)--(7.561,4.311)--(7.582,4.311)--(7.603,4.311)%
  --(7.624,4.311)--(7.644,4.311)--(7.665,4.311)--(7.686,4.311)--(7.707,4.311)--(7.728,4.311)%
  --(7.749,4.311)--(7.770,4.311)--(7.791,4.311)--(7.812,4.311)--(7.832,4.311)--(7.853,4.311)%
  --(7.874,4.311)--(7.895,4.311)--(7.916,4.311)--(7.937,4.311)--(7.958,4.311)--(7.979,4.311)%
  --(8.000,4.311)--(8.020,4.311)--(8.041,4.311)--(8.062,4.311)--(8.083,4.311)--(8.104,4.311)%
  --(8.125,4.311)--(8.146,4.311)--(8.167,4.311)--(8.188,4.311)--(8.208,4.311)--(8.229,4.311)%
  --(8.250,4.311)--(8.271,4.311)--(8.292,4.311)--(8.313,4.311)--(8.334,4.311)--(8.355,4.311)%
  --(8.375,4.311)--(8.396,4.311)--(8.417,4.311)--(8.438,4.311)--(8.459,4.311)--(8.480,4.311)%
  --(8.501,4.311)--(8.522,4.311)--(8.543,4.311)--(8.563,4.311)--(8.584,4.311)--(8.605,4.311)%
  --(8.626,4.311)--(8.647,4.311)--(8.668,4.311)--(8.689,4.311)--(8.710,4.311)--(8.731,4.311)%
  --(8.751,4.311)--(8.772,4.311)--(8.793,4.311)--(8.814,4.311)--(8.835,4.311)--(8.856,4.311)%
  --(8.877,4.311)--(8.898,4.311)--(8.919,3.832)--(8.939,3.832)--(8.960,3.832)--(8.981,3.832)%
  --(9.002,3.832)--(9.023,3.832)--(9.044,3.832)--(9.065,3.832)--(9.086,3.832)--(9.107,3.832)%
  --(9.127,3.832)--(9.148,3.832)--(9.169,3.832)--(9.190,3.832)--(9.211,3.832)--(9.232,3.832)%
  --(9.253,3.832)--(9.274,3.832)--(9.294,3.832)--(9.315,3.832)--(9.336,3.832)--(9.357,3.832)%
  --(9.378,3.832)--(9.399,3.832)--(9.420,3.832)--(9.441,3.832)--(9.462,3.832)--(9.482,3.832)%
  --(9.503,3.832)--(9.524,3.832)--(9.545,3.832)--(9.566,3.832)--(9.587,3.832)--(9.608,3.832)%
  --(9.629,3.832)--(9.650,3.832)--(9.670,3.832)--(9.691,3.832)--(9.712,3.832)--(9.733,3.832)%
  --(9.754,3.832)--(9.775,3.832)--(9.796,3.832)--(9.817,3.832)--(9.838,3.832)--(9.858,3.832)%
  --(9.879,3.832)--(9.900,3.832)--(9.921,3.832)--(9.942,3.832)--(9.963,3.832)--(9.984,3.832)%
  --(10.005,3.832)--(10.025,3.832)--(10.046,3.832)--(10.067,3.832)--(10.088,3.832)--(10.109,3.832)%
  --(10.130,3.832)--(10.151,3.832)--(10.172,3.832)--(10.193,3.832)--(10.213,3.832)--(10.234,3.832)%
  --(10.255,3.832)--(10.276,3.832)--(10.297,3.832)--(10.318,3.832)--(10.339,3.832)--(10.360,3.832)%
  --(10.381,3.832)--(10.401,3.832)--(10.422,3.832)--(10.443,3.832)--(10.464,3.832)--(10.485,3.832)%
  --(10.506,3.832)--(10.527,3.832)--(10.548,3.832)--(10.569,3.832)--(10.589,3.832)--(10.610,3.832)%
  --(10.631,3.832)--(10.652,3.832)--(10.673,3.832)--(10.694,3.832)--(10.715,3.832)--(10.736,3.832)%
  --(10.756,3.832)--(10.777,3.832)--(10.798,3.832)--(10.819,3.832)--(10.840,3.832)--(10.861,3.832)%
  --(10.882,3.832)--(10.903,3.832)--(10.924,3.832)--(10.944,3.832)--(10.965,3.832)--(10.986,3.832)%
  --(11.007,3.832)--(11.028,3.832);
\gpcolor{color=gp lt color border}
\node[gp node right,font={\fontsize{9pt}{10.8pt}\selectfont}] at (10.479,7.123) {q-desired-1};
\gpcolor{color=gp lt color 3}
\gpsetlinetype{gp lt plot 3}
\draw[gp path] (10.663,7.123)--(11.579,7.123);
\draw[gp path] (1.504,2.473)--(1.525,2.473)--(1.546,2.473)--(1.567,2.473)--(1.588,2.473)%
  --(1.608,2.473)--(1.629,2.473)--(1.650,2.473)--(1.671,2.473)--(1.692,2.473)--(1.713,2.473)%
  --(1.734,3.294)--(1.755,3.294)--(1.776,3.294)--(1.796,3.294)--(1.817,3.294)--(1.838,3.294)%
  --(1.859,3.294)--(1.880,3.294)--(1.901,3.294)--(1.922,3.294)--(1.943,3.294)--(1.963,3.294)%
  --(1.984,3.294)--(2.005,3.551)--(2.026,3.551)--(2.047,3.551)--(2.068,3.551)--(2.089,3.551)%
  --(2.110,3.551)--(2.131,3.551)--(2.151,3.551)--(2.172,3.551)--(2.193,3.551)--(2.214,3.551)%
  --(2.235,3.551)--(2.256,3.551)--(2.277,3.551)--(2.298,3.551)--(2.319,3.854)--(2.339,3.854)%
  --(2.360,3.854)--(2.381,3.854)--(2.402,3.854)--(2.423,3.854)--(2.444,3.854)--(2.465,3.854)%
  --(2.486,3.854)--(2.507,3.854)--(2.527,3.854)--(2.548,3.854)--(2.569,3.854)--(2.590,3.854)%
  --(2.611,3.854)--(2.632,3.854)--(2.653,3.854)--(2.674,3.854)--(2.695,3.854)--(2.715,3.854)%
  --(2.736,3.854)--(2.757,4.672)--(2.778,4.672)--(2.799,4.672)--(2.820,4.672)--(2.841,4.672)%
  --(2.862,4.672)--(2.882,4.672)--(2.903,4.672)--(2.924,4.672)--(2.945,4.672)--(2.966,4.672)%
  --(2.987,4.672)--(3.008,4.672)--(3.029,4.672)--(3.050,4.672)--(3.070,4.672)--(3.091,4.672)%
  --(3.112,4.672)--(3.133,4.672)--(3.154,4.672)--(3.175,4.672)--(3.196,4.672)--(3.217,4.672)%
  --(3.238,4.672)--(3.258,4.672)--(3.279,4.672)--(3.300,4.672)--(3.321,4.672)--(3.342,4.672)%
  --(3.363,4.672)--(3.384,4.672)--(3.405,4.672)--(3.426,4.672)--(3.446,4.672)--(3.467,4.672)%
  --(3.488,4.672)--(3.509,4.672)--(3.530,4.672)--(3.551,4.672)--(3.572,4.672)--(3.593,4.672)%
  --(3.613,4.672)--(3.634,4.672)--(3.655,4.672)--(3.676,4.672)--(3.697,4.672)--(3.718,4.672)%
  --(3.739,4.672)--(3.760,4.672)--(3.781,4.672)--(3.801,4.672)--(3.822,4.672)--(3.843,4.672)%
  --(3.864,4.672)--(3.885,4.672)--(3.906,4.672)--(3.927,4.672)--(3.948,4.672)--(3.969,4.672)%
  --(3.989,4.672)--(4.010,4.672)--(4.031,4.672)--(4.052,4.672)--(4.073,4.672)--(4.094,4.672)%
  --(4.115,4.672)--(4.136,4.672)--(4.157,4.672)--(4.177,4.672)--(4.198,4.672)--(4.219,4.672)%
  --(4.240,4.672)--(4.261,4.672)--(4.282,4.672)--(4.303,4.672)--(4.324,4.672)--(4.344,4.672)%
  --(4.365,4.672)--(4.386,4.672)--(4.407,4.672)--(4.428,4.672)--(4.449,4.672)--(4.470,4.672)%
  --(4.491,4.672)--(4.512,4.672)--(4.532,4.672)--(4.553,4.672)--(4.574,4.672)--(4.595,4.672)%
  --(4.616,4.672)--(4.637,4.672)--(4.658,4.672)--(4.679,4.672)--(4.700,4.672)--(4.720,4.672)%
  --(4.741,4.672)--(4.762,4.672)--(4.783,4.672)--(4.804,4.672)--(4.825,4.672)--(4.846,4.672)%
  --(4.867,4.672)--(4.888,4.672)--(4.908,4.672)--(4.929,4.672)--(4.950,4.672)--(4.971,4.672)%
  --(4.992,4.672)--(5.013,4.672)--(5.034,4.672)--(5.055,4.672)--(5.076,4.672)--(5.096,4.672)%
  --(5.117,4.672)--(5.138,4.672)--(5.159,4.672)--(5.180,4.672)--(5.201,4.672)--(5.222,4.672)%
  --(5.243,4.672)--(5.263,4.485)--(5.284,4.485)--(5.305,4.485)--(5.326,4.485)--(5.347,4.485)%
  --(5.368,4.485)--(5.389,4.485)--(5.410,4.485)--(5.431,4.485)--(5.451,4.485)--(5.472,4.485)%
  --(5.493,4.485)--(5.514,4.485)--(5.535,4.485)--(5.556,4.485)--(5.577,4.485)--(5.598,4.485)%
  --(5.619,4.485)--(5.639,4.485)--(5.660,4.485)--(5.681,4.485)--(5.702,4.694)--(5.723,4.694)%
  --(5.744,4.694)--(5.765,4.694)--(5.786,4.694)--(5.807,4.694)--(5.827,4.694)--(5.848,4.694)%
  --(5.869,4.694)--(5.890,4.694)--(5.911,4.694)--(5.932,4.694)--(5.953,4.694)--(5.974,4.694)%
  --(5.994,4.694)--(6.015,4.694)--(6.036,4.694)--(6.057,4.512)--(6.078,4.512)--(6.099,4.512)%
  --(6.120,4.512)--(6.141,4.512)--(6.162,4.512)--(6.182,4.512)--(6.203,4.512)--(6.224,4.512)%
  --(6.245,4.512)--(6.266,4.512)--(6.287,4.512)--(6.308,4.512)--(6.329,4.512)--(6.350,4.512)%
  --(6.370,4.512)--(6.391,3.723)--(6.412,3.723)--(6.433,3.723)--(6.454,3.723)--(6.475,3.723)%
  --(6.496,3.723)--(6.517,3.723)--(6.538,3.723)--(6.558,3.723)--(6.579,3.723)--(6.600,3.723)%
  --(6.621,3.723)--(6.642,3.723)--(6.663,3.723)--(6.684,3.723)--(6.705,3.723)--(6.726,3.723)%
  --(6.746,3.723)--(6.767,3.723)--(6.788,3.723)--(6.809,3.723)--(6.830,3.723)--(6.851,3.723)%
  --(6.872,3.723)--(6.893,3.723)--(6.913,3.723)--(6.934,3.723)--(6.955,3.723)--(6.976,3.723)%
  --(6.997,3.723)--(7.018,3.723)--(7.039,3.723)--(7.060,3.723)--(7.081,3.723)--(7.101,3.723)%
  --(7.122,3.723)--(7.143,3.723)--(7.164,3.723)--(7.185,3.723)--(7.206,3.723)--(7.227,3.723)%
  --(7.248,3.723)--(7.269,3.723)--(7.289,3.723)--(7.310,3.723)--(7.331,3.723)--(7.352,3.723)%
  --(7.373,3.723)--(7.394,3.723)--(7.415,3.723)--(7.436,3.723)--(7.457,3.723)--(7.477,3.723)%
  --(7.498,3.723)--(7.519,3.723)--(7.540,3.723)--(7.561,3.723)--(7.582,3.723)--(7.603,3.723)%
  --(7.624,3.723)--(7.644,3.723)--(7.665,3.723)--(7.686,3.723)--(7.707,3.723)--(7.728,3.723)%
  --(7.749,3.723)--(7.770,3.723)--(7.791,3.723)--(7.812,3.723)--(7.832,3.723)--(7.853,3.723)%
  --(7.874,3.723)--(7.895,3.723)--(7.916,3.723)--(7.937,3.723)--(7.958,3.723)--(7.979,3.723)%
  --(8.000,3.723)--(8.020,3.723)--(8.041,3.723)--(8.062,3.723)--(8.083,3.723)--(8.104,3.723)%
  --(8.125,3.723)--(8.146,3.723)--(8.167,3.723)--(8.188,3.723)--(8.208,3.723)--(8.229,3.723)%
  --(8.250,3.723)--(8.271,3.723)--(8.292,3.723)--(8.313,3.723)--(8.334,3.723)--(8.355,3.723)%
  --(8.375,3.723)--(8.396,3.723)--(8.417,3.723)--(8.438,3.723)--(8.459,3.723)--(8.480,3.723)%
  --(8.501,3.723)--(8.522,3.723)--(8.543,3.723)--(8.563,3.723)--(8.584,3.723)--(8.605,3.723)%
  --(8.626,3.723)--(8.647,3.723)--(8.668,3.723)--(8.689,3.723)--(8.710,3.723)--(8.731,3.723)%
  --(8.751,3.723)--(8.772,3.723)--(8.793,3.723)--(8.814,3.723)--(8.835,3.723)--(8.856,3.723)%
  --(8.877,3.723)--(8.898,3.723)--(8.919,3.838)--(8.939,3.838)--(8.960,3.838)--(8.981,3.838)%
  --(9.002,3.838)--(9.023,3.838)--(9.044,3.838)--(9.065,3.838)--(9.086,3.838)--(9.107,3.838)%
  --(9.127,3.838)--(9.148,3.838)--(9.169,3.838)--(9.190,3.838)--(9.211,3.838)--(9.232,3.838)%
  --(9.253,3.838)--(9.274,3.838)--(9.294,3.838)--(9.315,3.838)--(9.336,3.838)--(9.357,3.838)%
  --(9.378,3.838)--(9.399,3.838)--(9.420,3.838)--(9.441,3.838)--(9.462,3.838)--(9.482,3.838)%
  --(9.503,3.838)--(9.524,3.838)--(9.545,3.838)--(9.566,3.838)--(9.587,3.838)--(9.608,3.838)%
  --(9.629,3.838)--(9.650,3.838)--(9.670,3.838)--(9.691,3.838)--(9.712,3.838)--(9.733,3.838)%
  --(9.754,3.838)--(9.775,3.838)--(9.796,3.838)--(9.817,3.838)--(9.838,3.838)--(9.858,3.838)%
  --(9.879,3.838)--(9.900,3.838)--(9.921,3.838)--(9.942,3.838)--(9.963,3.838)--(9.984,3.838)%
  --(10.005,3.838)--(10.025,3.838)--(10.046,3.838)--(10.067,3.838)--(10.088,3.838)--(10.109,3.838)%
  --(10.130,3.838)--(10.151,3.838)--(10.172,3.838)--(10.193,3.838)--(10.213,3.838)--(10.234,3.838)%
  --(10.255,3.838)--(10.276,3.838)--(10.297,3.838)--(10.318,3.838)--(10.339,3.838)--(10.360,3.838)%
  --(10.381,3.838)--(10.401,3.838)--(10.422,3.838)--(10.443,3.838)--(10.464,3.838)--(10.485,3.838)%
  --(10.506,3.838)--(10.527,3.838)--(10.548,3.838)--(10.569,3.838)--(10.589,3.838)--(10.610,3.838)%
  --(10.631,3.838)--(10.652,3.838)--(10.673,3.838)--(10.694,3.838)--(10.715,3.838)--(10.736,3.838)%
  --(10.756,3.838)--(10.777,3.838)--(10.798,3.838)--(10.819,3.838)--(10.840,3.838)--(10.861,3.838)%
  --(10.882,3.838)--(10.903,3.838)--(10.924,3.838)--(10.944,3.838)--(10.965,3.838)--(10.986,3.838)%
  --(11.007,3.838)--(11.028,3.838);
\gpcolor{color=gp lt color border}
\node[gp node right,font={\fontsize{9pt}{10.8pt}\selectfont}] at (10.479,6.815) {q-real-2};
\gpcolor{color=gp lt color 4}
\gpsetlinetype{gp lt plot 4}
\draw[gp path] (10.663,6.815)--(11.579,6.815);
\draw[gp path] (1.504,3.972)--(1.525,3.972)--(1.546,3.972)--(1.567,3.972)--(1.588,3.972)%
  --(1.608,3.972)--(1.629,3.972)--(1.650,3.972)--(1.671,3.972)--(1.692,3.972)--(1.713,3.972)%
  --(1.734,3.972)--(1.755,3.331)--(1.776,3.331)--(1.796,3.331)--(1.817,3.331)--(1.838,3.331)%
  --(1.859,3.331)--(1.880,3.331)--(1.901,3.331)--(1.922,3.331)--(1.943,3.331)--(1.963,3.331)%
  --(1.984,3.331)--(2.005,3.331)--(2.026,3.057)--(2.047,3.057)--(2.068,3.057)--(2.089,3.057)%
  --(2.110,3.057)--(2.131,3.057)--(2.151,3.057)--(2.172,3.057)--(2.193,3.057)--(2.214,3.057)%
  --(2.235,3.057)--(2.256,3.057)--(2.277,3.057)--(2.298,3.057)--(2.319,2.948)--(2.339,2.948)%
  --(2.360,2.948)--(2.381,2.948)--(2.402,2.948)--(2.423,2.948)--(2.444,2.948)--(2.465,2.948)%
  --(2.486,2.948)--(2.507,2.948)--(2.527,2.948)--(2.548,2.948)--(2.569,2.948)--(2.590,2.948)%
  --(2.611,2.948)--(2.632,2.948)--(2.653,2.948)--(2.674,2.948)--(2.695,2.948)--(2.715,2.948)%
  --(2.736,2.948)--(2.757,2.659)--(2.778,2.659)--(2.799,2.659)--(2.820,2.659)--(2.841,2.659)%
  --(2.862,2.659)--(2.882,2.659)--(2.903,2.659)--(2.924,2.659)--(2.945,2.659)--(2.966,2.659)%
  --(2.987,2.659)--(3.008,2.659)--(3.029,2.659)--(3.050,2.659)--(3.070,2.659)--(3.091,2.659)%
  --(3.112,2.659)--(3.133,2.659)--(3.154,2.659)--(3.175,2.659)--(3.196,2.659)--(3.217,2.659)%
  --(3.238,2.659)--(3.258,2.659)--(3.279,2.659)--(3.300,2.659)--(3.321,2.659)--(3.342,2.659)%
  --(3.363,2.659)--(3.384,2.659)--(3.405,2.659)--(3.426,2.659)--(3.446,2.659)--(3.467,2.659)%
  --(3.488,2.659)--(3.509,2.659)--(3.530,2.659)--(3.551,2.659)--(3.572,2.659)--(3.593,2.659)%
  --(3.613,2.659)--(3.634,2.659)--(3.655,2.659)--(3.676,2.659)--(3.697,2.659)--(3.718,2.659)%
  --(3.739,2.659)--(3.760,2.659)--(3.781,2.659)--(3.801,2.659)--(3.822,2.659)--(3.843,2.659)%
  --(3.864,2.659)--(3.885,2.659)--(3.906,2.659)--(3.927,2.659)--(3.948,2.659)--(3.969,2.659)%
  --(3.989,2.659)--(4.010,2.659)--(4.031,2.659)--(4.052,2.659)--(4.073,2.659)--(4.094,2.659)%
  --(4.115,2.659)--(4.136,2.659)--(4.157,2.659)--(4.177,2.659)--(4.198,2.659)--(4.219,2.659)%
  --(4.240,2.659)--(4.261,2.659)--(4.282,2.659)--(4.303,2.659)--(4.324,2.659)--(4.344,2.659)%
  --(4.365,2.659)--(4.386,2.659)--(4.407,2.659)--(4.428,2.659)--(4.449,2.659)--(4.470,2.659)%
  --(4.491,2.659)--(4.512,2.659)--(4.532,2.659)--(4.553,2.659)--(4.574,2.659)--(4.595,2.659)%
  --(4.616,2.659)--(4.637,2.659)--(4.658,2.659)--(4.679,2.659)--(4.700,2.659)--(4.720,2.659)%
  --(4.741,2.659)--(4.762,2.659)--(4.783,2.659)--(4.804,2.659)--(4.825,2.659)--(4.846,2.659)%
  --(4.867,2.659)--(4.888,2.659)--(4.908,2.659)--(4.929,2.659)--(4.950,2.659)--(4.971,2.659)%
  --(4.992,2.659)--(5.013,2.659)--(5.034,2.659)--(5.055,2.659)--(5.076,2.659)--(5.096,2.659)%
  --(5.117,2.659)--(5.138,2.659)--(5.159,2.659)--(5.180,2.659)--(5.201,2.659)--(5.222,2.659)%
  --(5.243,2.659)--(5.263,2.659)--(5.284,2.479)--(5.305,2.479)--(5.326,2.479)--(5.347,2.479)%
  --(5.368,2.479)--(5.389,2.479)--(5.410,2.479)--(5.431,2.479)--(5.451,2.479)--(5.472,2.479)%
  --(5.493,2.479)--(5.514,2.479)--(5.535,2.479)--(5.556,2.479)--(5.577,2.479)--(5.598,2.479)%
  --(5.619,2.479)--(5.639,2.479)--(5.660,2.479)--(5.681,2.479)--(5.702,2.479)--(5.723,2.631)%
  --(5.744,2.631)--(5.765,2.631)--(5.786,2.631)--(5.807,2.631)--(5.827,2.631)--(5.848,2.631)%
  --(5.869,2.631)--(5.890,2.631)--(5.911,2.631)--(5.932,2.631)--(5.953,2.631)--(5.974,2.631)%
  --(5.994,2.631)--(6.015,2.631)--(6.036,2.631)--(6.057,2.631)--(6.078,2.552)--(6.099,2.552)%
  --(6.120,2.552)--(6.141,2.552)--(6.162,2.552)--(6.182,2.552)--(6.203,2.552)--(6.224,2.552)%
  --(6.245,2.552)--(6.266,2.552)--(6.287,2.552)--(6.308,2.552)--(6.329,2.552)--(6.350,2.552)%
  --(6.370,2.552)--(6.391,2.317)--(6.412,2.317)--(6.433,2.317)--(6.454,2.317)--(6.475,2.317)%
  --(6.496,2.317)--(6.517,2.317)--(6.538,2.317)--(6.558,2.317)--(6.579,2.317)--(6.600,2.317)%
  --(6.621,2.317)--(6.642,2.317)--(6.663,2.317)--(6.684,2.317)--(6.705,2.317)--(6.726,2.317)%
  --(6.746,2.317)--(6.767,2.317)--(6.788,2.317)--(6.809,2.317)--(6.830,2.317)--(6.851,2.317)%
  --(6.872,2.317)--(6.893,2.317)--(6.913,2.317)--(6.934,2.317)--(6.955,2.317)--(6.976,2.317)%
  --(6.997,2.317)--(7.018,2.317)--(7.039,2.317)--(7.060,2.317)--(7.081,2.317)--(7.101,2.317)%
  --(7.122,2.317)--(7.143,2.317)--(7.164,2.317)--(7.185,2.317)--(7.206,2.317)--(7.227,2.317)%
  --(7.248,2.317)--(7.269,2.317)--(7.289,2.317)--(7.310,2.317)--(7.331,2.317)--(7.352,2.317)%
  --(7.373,2.317)--(7.394,2.317)--(7.415,2.317)--(7.436,2.317)--(7.457,2.317)--(7.477,2.317)%
  --(7.498,2.317)--(7.519,2.317)--(7.540,2.317)--(7.561,2.317)--(7.582,2.317)--(7.603,2.317)%
  --(7.624,2.317)--(7.644,2.317)--(7.665,2.317)--(7.686,2.317)--(7.707,2.317)--(7.728,2.317)%
  --(7.749,2.317)--(7.770,2.317)--(7.791,2.317)--(7.812,2.317)--(7.832,2.317)--(7.853,2.317)%
  --(7.874,2.317)--(7.895,2.317)--(7.916,2.317)--(7.937,2.317)--(7.958,2.317)--(7.979,2.317)%
  --(8.000,2.317)--(8.020,2.317)--(8.041,2.317)--(8.062,2.317)--(8.083,2.317)--(8.104,2.317)%
  --(8.125,2.317)--(8.146,2.317)--(8.167,2.317)--(8.188,2.317)--(8.208,2.317)--(8.229,2.317)%
  --(8.250,2.317)--(8.271,2.317)--(8.292,2.317)--(8.313,2.317)--(8.334,2.317)--(8.355,2.317)%
  --(8.375,2.317)--(8.396,2.317)--(8.417,2.317)--(8.438,2.317)--(8.459,2.317)--(8.480,2.317)%
  --(8.501,2.317)--(8.522,2.317)--(8.543,2.317)--(8.563,2.317)--(8.584,2.317)--(8.605,2.317)%
  --(8.626,2.317)--(8.647,2.317)--(8.668,2.317)--(8.689,2.317)--(8.710,2.317)--(8.731,2.317)%
  --(8.751,2.317)--(8.772,2.317)--(8.793,2.317)--(8.814,2.317)--(8.835,2.317)--(8.856,2.317)%
  --(8.877,2.317)--(8.898,2.317)--(8.919,1.731)--(8.939,1.731)--(8.960,1.731)--(8.981,1.731)%
  --(9.002,1.731)--(9.023,1.731)--(9.044,1.731)--(9.065,1.731)--(9.086,1.731)--(9.107,1.731)%
  --(9.127,1.731)--(9.148,1.731)--(9.169,1.731)--(9.190,1.731)--(9.211,1.731)--(9.232,1.731)%
  --(9.253,1.731)--(9.274,1.731)--(9.294,1.731)--(9.315,1.731)--(9.336,1.731)--(9.357,1.731)%
  --(9.378,1.731)--(9.399,1.731)--(9.420,1.731)--(9.441,1.731)--(9.462,1.731)--(9.482,1.731)%
  --(9.503,1.731)--(9.524,1.731)--(9.545,1.731)--(9.566,1.731)--(9.587,1.731)--(9.608,1.731)%
  --(9.629,1.731)--(9.650,1.731)--(9.670,1.731)--(9.691,1.731)--(9.712,1.731)--(9.733,1.731)%
  --(9.754,1.731)--(9.775,1.731)--(9.796,1.731)--(9.817,1.731)--(9.838,1.731)--(9.858,1.731)%
  --(9.879,1.731)--(9.900,1.731)--(9.921,1.731)--(9.942,1.731)--(9.963,1.731)--(9.984,1.731)%
  --(10.005,1.731)--(10.025,1.731)--(10.046,1.731)--(10.067,1.731)--(10.088,1.731)--(10.109,1.731)%
  --(10.130,1.731)--(10.151,1.731)--(10.172,1.731)--(10.193,1.731)--(10.213,1.731)--(10.234,1.731)%
  --(10.255,1.731)--(10.276,1.731)--(10.297,1.731)--(10.318,1.731)--(10.339,1.731)--(10.360,1.731)%
  --(10.381,1.731)--(10.401,1.731)--(10.422,1.731)--(10.443,1.731)--(10.464,1.731)--(10.485,1.731)%
  --(10.506,1.731)--(10.527,1.731)--(10.548,1.731)--(10.569,1.731)--(10.589,1.731)--(10.610,1.731)%
  --(10.631,1.731)--(10.652,1.731)--(10.673,1.731)--(10.694,1.731)--(10.715,1.731)--(10.736,1.731)%
  --(10.756,1.731)--(10.777,1.731)--(10.798,1.731)--(10.819,1.731)--(10.840,1.731)--(10.861,1.731)%
  --(10.882,1.731)--(10.903,1.731)--(10.924,1.731)--(10.944,1.731)--(10.965,1.731)--(10.986,1.731)%
  --(11.007,1.731)--(11.028,1.731);
\gpcolor{color=gp lt color border}
\node[gp node right,font={\fontsize{9pt}{10.8pt}\selectfont}] at (10.479,6.507) {q-desired-2};
\gpcolor{color=gp lt color 5}
\gpsetlinetype{gp lt plot 5}
\draw[gp path] (10.663,6.507)--(11.579,6.507);
\draw[gp path] (1.504,3.973)--(1.525,3.973)--(1.546,3.973)--(1.567,3.973)--(1.588,3.973)%
  --(1.608,3.973)--(1.629,3.973)--(1.650,3.973)--(1.671,3.973)--(1.692,3.973)--(1.713,3.973)%
  --(1.734,3.063)--(1.755,3.063)--(1.776,3.063)--(1.796,3.063)--(1.817,3.063)--(1.838,3.063)%
  --(1.859,3.063)--(1.880,3.063)--(1.901,3.063)--(1.922,3.063)--(1.943,3.063)--(1.963,3.063)%
  --(1.984,3.063)--(2.005,3.057)--(2.026,3.057)--(2.047,3.057)--(2.068,3.057)--(2.089,3.057)%
  --(2.110,3.057)--(2.131,3.057)--(2.151,3.057)--(2.172,3.057)--(2.193,3.057)--(2.214,3.057)%
  --(2.235,3.057)--(2.256,3.057)--(2.277,3.057)--(2.298,3.057)--(2.319,2.940)--(2.339,2.940)%
  --(2.360,2.940)--(2.381,2.940)--(2.402,2.940)--(2.423,2.940)--(2.444,2.940)--(2.465,2.940)%
  --(2.486,2.940)--(2.507,2.940)--(2.527,2.940)--(2.548,2.940)--(2.569,2.940)--(2.590,2.940)%
  --(2.611,2.940)--(2.632,2.940)--(2.653,2.940)--(2.674,2.940)--(2.695,2.940)--(2.715,2.940)%
  --(2.736,2.940)--(2.757,2.540)--(2.778,2.540)--(2.799,2.540)--(2.820,2.540)--(2.841,2.540)%
  --(2.862,2.540)--(2.882,2.540)--(2.903,2.540)--(2.924,2.540)--(2.945,2.540)--(2.966,2.540)%
  --(2.987,2.540)--(3.008,2.540)--(3.029,2.540)--(3.050,2.540)--(3.070,2.540)--(3.091,2.540)%
  --(3.112,2.540)--(3.133,2.540)--(3.154,2.540)--(3.175,2.540)--(3.196,2.540)--(3.217,2.540)%
  --(3.238,2.540)--(3.258,2.540)--(3.279,2.540)--(3.300,2.540)--(3.321,2.540)--(3.342,2.540)%
  --(3.363,2.540)--(3.384,2.540)--(3.405,2.540)--(3.426,2.540)--(3.446,2.540)--(3.467,2.540)%
  --(3.488,2.540)--(3.509,2.540)--(3.530,2.540)--(3.551,2.540)--(3.572,2.540)--(3.593,2.540)%
  --(3.613,2.540)--(3.634,2.540)--(3.655,2.540)--(3.676,2.540)--(3.697,2.540)--(3.718,2.540)%
  --(3.739,2.540)--(3.760,2.540)--(3.781,2.540)--(3.801,2.540)--(3.822,2.540)--(3.843,2.540)%
  --(3.864,2.540)--(3.885,2.540)--(3.906,2.540)--(3.927,2.540)--(3.948,2.540)--(3.969,2.540)%
  --(3.989,2.540)--(4.010,2.540)--(4.031,2.540)--(4.052,2.540)--(4.073,2.540)--(4.094,2.540)%
  --(4.115,2.540)--(4.136,2.540)--(4.157,2.540)--(4.177,2.540)--(4.198,2.540)--(4.219,2.540)%
  --(4.240,2.540)--(4.261,2.540)--(4.282,2.540)--(4.303,2.540)--(4.324,2.540)--(4.344,2.540)%
  --(4.365,2.540)--(4.386,2.540)--(4.407,2.540)--(4.428,2.540)--(4.449,2.540)--(4.470,2.540)%
  --(4.491,2.540)--(4.512,2.540)--(4.532,2.540)--(4.553,2.540)--(4.574,2.540)--(4.595,2.540)%
  --(4.616,2.540)--(4.637,2.540)--(4.658,2.540)--(4.679,2.540)--(4.700,2.540)--(4.720,2.540)%
  --(4.741,2.540)--(4.762,2.540)--(4.783,2.540)--(4.804,2.540)--(4.825,2.540)--(4.846,2.540)%
  --(4.867,2.540)--(4.888,2.540)--(4.908,2.540)--(4.929,2.540)--(4.950,2.540)--(4.971,2.540)%
  --(4.992,2.540)--(5.013,2.540)--(5.034,2.540)--(5.055,2.540)--(5.076,2.540)--(5.096,2.540)%
  --(5.117,2.540)--(5.138,2.540)--(5.159,2.540)--(5.180,2.540)--(5.201,2.540)--(5.222,2.540)%
  --(5.243,2.540)--(5.263,2.477)--(5.284,2.477)--(5.305,2.477)--(5.326,2.477)--(5.347,2.477)%
  --(5.368,2.477)--(5.389,2.477)--(5.410,2.477)--(5.431,2.477)--(5.451,2.477)--(5.472,2.477)%
  --(5.493,2.477)--(5.514,2.477)--(5.535,2.477)--(5.556,2.477)--(5.577,2.477)--(5.598,2.477)%
  --(5.619,2.477)--(5.639,2.477)--(5.660,2.477)--(5.681,2.477)--(5.702,2.652)--(5.723,2.652)%
  --(5.744,2.652)--(5.765,2.652)--(5.786,2.652)--(5.807,2.652)--(5.827,2.652)--(5.848,2.652)%
  --(5.869,2.652)--(5.890,2.652)--(5.911,2.652)--(5.932,2.652)--(5.953,2.652)--(5.974,2.652)%
  --(5.994,2.652)--(6.015,2.652)--(6.036,2.652)--(6.057,2.549)--(6.078,2.549)--(6.099,2.549)%
  --(6.120,2.549)--(6.141,2.549)--(6.162,2.549)--(6.182,2.549)--(6.203,2.549)--(6.224,2.549)%
  --(6.245,2.549)--(6.266,2.549)--(6.287,2.549)--(6.308,2.549)--(6.329,2.549)--(6.350,2.549)%
  --(6.370,2.549)--(6.391,1.648)--(6.412,1.648)--(6.433,1.648)--(6.454,1.648)--(6.475,1.648)%
  --(6.496,1.648)--(6.517,1.648)--(6.538,1.648)--(6.558,1.648)--(6.579,1.648)--(6.600,1.648)%
  --(6.621,1.648)--(6.642,1.648)--(6.663,1.648)--(6.684,1.648)--(6.705,1.648)--(6.726,1.648)%
  --(6.746,1.648)--(6.767,1.648)--(6.788,1.648)--(6.809,1.648)--(6.830,1.648)--(6.851,1.648)%
  --(6.872,1.648)--(6.893,1.648)--(6.913,1.648)--(6.934,1.648)--(6.955,1.648)--(6.976,1.648)%
  --(6.997,1.648)--(7.018,1.648)--(7.039,1.648)--(7.060,1.648)--(7.081,1.648)--(7.101,1.648)%
  --(7.122,1.648)--(7.143,1.648)--(7.164,1.648)--(7.185,1.648)--(7.206,1.648)--(7.227,1.648)%
  --(7.248,1.648)--(7.269,1.648)--(7.289,1.648)--(7.310,1.648)--(7.331,1.648)--(7.352,1.648)%
  --(7.373,1.648)--(7.394,1.648)--(7.415,1.648)--(7.436,1.648)--(7.457,1.648)--(7.477,1.648)%
  --(7.498,1.648)--(7.519,1.648)--(7.540,1.648)--(7.561,1.648)--(7.582,1.648)--(7.603,1.648)%
  --(7.624,1.648)--(7.644,1.648)--(7.665,1.648)--(7.686,1.648)--(7.707,1.648)--(7.728,1.648)%
  --(7.749,1.648)--(7.770,1.648)--(7.791,1.648)--(7.812,1.648)--(7.832,1.648)--(7.853,1.648)%
  --(7.874,1.648)--(7.895,1.648)--(7.916,1.648)--(7.937,1.648)--(7.958,1.648)--(7.979,1.648)%
  --(8.000,1.648)--(8.020,1.648)--(8.041,1.648)--(8.062,1.648)--(8.083,1.648)--(8.104,1.648)%
  --(8.125,1.648)--(8.146,1.648)--(8.167,1.648)--(8.188,1.648)--(8.208,1.648)--(8.229,1.648)%
  --(8.250,1.648)--(8.271,1.648)--(8.292,1.648)--(8.313,1.648)--(8.334,1.648)--(8.355,1.648)%
  --(8.375,1.648)--(8.396,1.648)--(8.417,1.648)--(8.438,1.648)--(8.459,1.648)--(8.480,1.648)%
  --(8.501,1.648)--(8.522,1.648)--(8.543,1.648)--(8.563,1.648)--(8.584,1.648)--(8.605,1.648)%
  --(8.626,1.648)--(8.647,1.648)--(8.668,1.648)--(8.689,1.648)--(8.710,1.648)--(8.731,1.648)%
  --(8.751,1.648)--(8.772,1.648)--(8.793,1.648)--(8.814,1.648)--(8.835,1.648)--(8.856,1.648)%
  --(8.877,1.648)--(8.898,1.648)--(8.919,1.734)--(8.939,1.734)--(8.960,1.734)--(8.981,1.734)%
  --(9.002,1.734)--(9.023,1.734)--(9.044,1.734)--(9.065,1.734)--(9.086,1.734)--(9.107,1.734)%
  --(9.127,1.734)--(9.148,1.734)--(9.169,1.734)--(9.190,1.734)--(9.211,1.734)--(9.232,1.734)%
  --(9.253,1.734)--(9.274,1.734)--(9.294,1.734)--(9.315,1.734)--(9.336,1.734)--(9.357,1.734)%
  --(9.378,1.734)--(9.399,1.734)--(9.420,1.734)--(9.441,1.734)--(9.462,1.734)--(9.482,1.734)%
  --(9.503,1.734)--(9.524,1.734)--(9.545,1.734)--(9.566,1.734)--(9.587,1.734)--(9.608,1.734)%
  --(9.629,1.734)--(9.650,1.734)--(9.670,1.734)--(9.691,1.734)--(9.712,1.734)--(9.733,1.734)%
  --(9.754,1.734)--(9.775,1.734)--(9.796,1.734)--(9.817,1.734)--(9.838,1.734)--(9.858,1.734)%
  --(9.879,1.734)--(9.900,1.734)--(9.921,1.734)--(9.942,1.734)--(9.963,1.734)--(9.984,1.734)%
  --(10.005,1.734)--(10.025,1.734)--(10.046,1.734)--(10.067,1.734)--(10.088,1.734)--(10.109,1.734)%
  --(10.130,1.734)--(10.151,1.734)--(10.172,1.734)--(10.193,1.734)--(10.213,1.734)--(10.234,1.734)%
  --(10.255,1.734)--(10.276,1.734)--(10.297,1.734)--(10.318,1.734)--(10.339,1.734)--(10.360,1.734)%
  --(10.381,1.734)--(10.401,1.734)--(10.422,1.734)--(10.443,1.734)--(10.464,1.734)--(10.485,1.734)%
  --(10.506,1.734)--(10.527,1.734)--(10.548,1.734)--(10.569,1.734)--(10.589,1.734)--(10.610,1.734)%
  --(10.631,1.734)--(10.652,1.734)--(10.673,1.734)--(10.694,1.734)--(10.715,1.734)--(10.736,1.734)%
  --(10.756,1.734)--(10.777,1.734)--(10.798,1.734)--(10.819,1.734)--(10.840,1.734)--(10.861,1.734)%
  --(10.882,1.734)--(10.903,1.734)--(10.924,1.734)--(10.944,1.734)--(10.965,1.734)--(10.986,1.734)%
  --(11.007,1.734)--(11.028,1.734);
\gpcolor{color=gp lt color border}
\gpsetlinetype{gp lt border}
\draw[gp path] (1.504,8.381)--(1.504,0.985)--(11.947,0.985)--(11.947,8.381)--cycle;
%% coordinates of the plot area
\gpdefrectangularnode{gp plot 1}{\pgfpoint{1.504cm}{0.985cm}}{\pgfpoint{11.947cm}{8.381cm}}
\end{tikzpicture}
%% gnuplot variables

    }
    \caption{Real configuration as reported by the PA10 controller versus the desired configuration for the first 3 joints.}
    \label{fig:q_real_desired}
\end{figure}


An optimal performance of the system would yield a minimal or none difference between current and desired configurations. 
However, the results shows otherwise.
It can be seen that for big steps between the input and current joint positions the actuators do not behave as intended.
This loss of the expected path is translated into small disruptions in the robot movements that can be seen when following the target. 
The mistakes in the behavior can be partially due to errors in the mechanical part, but more likely to constraints in the PA10 controller

Further probing reveals that this problem is not common in PA10 models, but the particular one used in this project has an error.
At the time of writing, no measures have been taken to solve the problem, but a suggested solution could be manually implement a limit to how much a joint may be moved in on step.
If the desired move is greater than the limit, the system should split the move into several smaller ones.


% At the time of writing, the reason for the joint positional error is yet undetermined.

% Variation between the true and desired configuration might be caused by mechanical play of a particular joint.
% The play will have to be larger than the encoder resolution to be detected and thus significant.


\section{Precision of object detection}
Backprojection?


\section{Detected object position and Kalman filter estimate}
To show that the Kalman filter is working as expected, predicting the ball position even in cases where the detection algorithm fails momentarily, the following experiment is conducted.

The PA10 robot configuration and thus the camera position is kept constant.
The target to be tracked, described in section \ref{chap:feature_extraction}, is then moved slowly inside the field of view of the cameras, to trigger the detection algorithm.
At some point, the ball is occluded, so that the tracking algorithm no longer succeeds in detecting it.
That is the point at which the Kalman filter predictions become particularly important to keep track of the ball.
Especially so in cases where the ball would have otherwise moved out of sight of the cameras while occluded.

\begin{figure}[htb]
    \centering
    \begin{subfigure}[b]{0.49\textwidth}
        \resizebox{\columnwidth}{!}{%
            \begin{tikzpicture}[gnuplot]
%% generated with GNUPLOT 4.6p4 (Lua 5.1; terminal rev. 99, script rev. 100)
%% Thu 28 May 2015 09:06:58 AM CEST
\path (0.000,0.000) rectangle (12.500,8.750);
\gpcolor{color=gp lt color border}
\gpsetlinetype{gp lt border}
\gpsetlinewidth{1.00}
\draw[gp path] (1.800,2.277)--(5.058,4.137);
\draw[gp path] (10.700,3.063)--(5.058,4.137);
\draw[gp path] (1.800,2.277)--(1.800,5.995);
\draw[gp path] (1.800,2.277)--(1.937,2.355);
\node[gp node center,font={\fontsize{9pt}{10.8pt}\selectfont}] at (1.660,2.144) {-0.35};
\draw[gp path] (5.058,4.137)--(4.921,4.059);
\draw[gp path] (2.427,2.158)--(2.564,2.236);
\node[gp node center,font={\fontsize{9pt}{10.8pt}\selectfont}] at (2.287,2.025) {-0.3};
\draw[gp path] (5.685,4.017)--(5.548,3.939);
\draw[gp path] (3.054,2.039)--(3.191,2.117);
\node[gp node center,font={\fontsize{9pt}{10.8pt}\selectfont}] at (2.914,1.905) {-0.25};
\draw[gp path] (6.311,3.898)--(6.175,3.820);
\draw[gp path] (3.681,1.919)--(3.818,1.997);
\node[gp node center,font={\fontsize{9pt}{10.8pt}\selectfont}] at (3.541,1.786) {-0.2};
\draw[gp path] (6.938,3.779)--(6.801,3.701);
\draw[gp path] (4.308,1.800)--(4.445,1.878);
\node[gp node center,font={\fontsize{9pt}{10.8pt}\selectfont}] at (4.168,1.667) {-0.15};
\draw[gp path] (7.565,3.659)--(7.428,3.582);
\draw[gp path] (4.935,1.681)--(5.072,1.759);
\node[gp node center,font={\fontsize{9pt}{10.8pt}\selectfont}] at (4.795,1.547) {-0.1};
\draw[gp path] (8.192,3.540)--(8.055,3.462);
\draw[gp path] (5.562,1.562)--(5.699,1.639);
\node[gp node center,font={\fontsize{9pt}{10.8pt}\selectfont}] at (5.422,1.428) {-0.05};
\draw[gp path] (8.819,3.421)--(8.682,3.343);
\draw[gp path] (6.189,1.442)--(6.325,1.520);
\node[gp node center,font={\fontsize{9pt}{10.8pt}\selectfont}] at (6.049,1.309) { 0};
\draw[gp path] (9.446,3.302)--(9.309,3.224);
\draw[gp path] (6.815,1.323)--(6.952,1.401);
\node[gp node center,font={\fontsize{9pt}{10.8pt}\selectfont}] at (6.675,1.190) { 0.05};
\draw[gp path] (10.073,3.182)--(9.936,3.104);
\draw[gp path] (7.442,1.204)--(7.579,1.282);
\node[gp node center,font={\fontsize{9pt}{10.8pt}\selectfont}] at (7.302,1.070) { 0.1};
\draw[gp path] (10.700,3.063)--(10.563,2.985);
\draw[gp path] (7.442,1.204)--(7.286,1.233);
\node[gp node center,font={\fontsize{9pt}{10.8pt}\selectfont}] at (7.601,1.153) {-0.15};
\draw[gp path] (1.800,2.277)--(1.956,2.248);
\draw[gp path] (7.907,1.469)--(7.752,1.499);
\node[gp node center,font={\fontsize{9pt}{10.8pt}\selectfont}] at (8.067,1.419) {-0.1};
\draw[gp path] (2.265,2.543)--(2.421,2.513);
\draw[gp path] (8.373,1.735)--(8.217,1.765);
\node[gp node center,font={\fontsize{9pt}{10.8pt}\selectfont}] at (8.532,1.684) {-0.05};
\draw[gp path] (2.731,2.808)--(2.887,2.779);
\draw[gp path] (8.838,2.001)--(8.682,2.030);
\node[gp node center,font={\fontsize{9pt}{10.8pt}\selectfont}] at (8.998,1.950) { 0};
\draw[gp path] (3.196,3.074)--(3.352,3.044);
\draw[gp path] (9.304,2.266)--(9.148,2.296);
\node[gp node center,font={\fontsize{9pt}{10.8pt}\selectfont}] at (9.463,2.215) { 0.05};
\draw[gp path] (3.662,3.340)--(3.818,3.310);
\draw[gp path] (9.769,2.532)--(9.613,2.561);
\node[gp node center,font={\fontsize{9pt}{10.8pt}\selectfont}] at (9.929,2.481) { 0.1};
\draw[gp path] (4.127,3.605)--(4.283,3.576);
\draw[gp path] (10.235,2.797)--(10.079,2.827);
\node[gp node center,font={\fontsize{9pt}{10.8pt}\selectfont}] at (10.394,2.747) { 0.15};
\draw[gp path] (4.593,3.871)--(4.748,3.841);
\draw[gp path] (10.700,3.063)--(10.544,3.093);
\node[gp node center,font={\fontsize{9pt}{10.8pt}\selectfont}] at (10.859,3.012) { 0.2};
\draw[gp path] (5.058,4.137)--(5.214,4.107);
\draw[gp path] (1.800,3.517)--(1.980,3.517);
\node[gp node right,font={\fontsize{9pt}{10.8pt}\selectfont}] at (1.440,3.517) { 0.5};
\draw[gp path] (1.800,3.792)--(1.980,3.792);
\node[gp node right,font={\fontsize{9pt}{10.8pt}\selectfont}] at (1.440,3.792) { 0.6};
\draw[gp path] (1.800,4.068)--(1.980,4.068);
\node[gp node right,font={\fontsize{9pt}{10.8pt}\selectfont}] at (1.440,4.068) { 0.7};
\draw[gp path] (1.800,4.343)--(1.980,4.343);
\node[gp node right,font={\fontsize{9pt}{10.8pt}\selectfont}] at (1.440,4.343) { 0.8};
\draw[gp path] (1.800,4.618)--(1.980,4.618);
\node[gp node right,font={\fontsize{9pt}{10.8pt}\selectfont}] at (1.440,4.618) { 0.9};
\draw[gp path] (1.800,4.893)--(1.980,4.893);
\node[gp node right,font={\fontsize{9pt}{10.8pt}\selectfont}] at (1.440,4.893) { 1};
\draw[gp path] (1.800,5.169)--(1.980,5.169);
\node[gp node right,font={\fontsize{9pt}{10.8pt}\selectfont}] at (1.440,5.169) { 1.1};
\draw[gp path] (1.800,5.444)--(1.980,5.444);
\node[gp node right,font={\fontsize{9pt}{10.8pt}\selectfont}] at (1.440,5.444) { 1.2};
\draw[gp path] (1.800,5.719)--(1.980,5.719);
\node[gp node right,font={\fontsize{9pt}{10.8pt}\selectfont}] at (1.440,5.719) { 1.3};
\draw[gp path] (1.800,5.995)--(1.980,5.995);
\node[gp node right,font={\fontsize{9pt}{10.8pt}\selectfont}] at (1.440,5.995) { 1.4};
\node[gp node center,font={\fontsize{9pt}{10.8pt}\selectfont}] at (0.512,4.755) {z};
\node[gp node right,font={\fontsize{9pt}{10.8pt}\selectfont}] at (9.416,7.645) {detected ball position};
\gpcolor{rgb color={1.000,0.822,0.000}}
\gpsetlinetype{gp lt plot 0}
\draw[gp path] (9.600,7.645)--(9.638,7.645);
\gpcolor{rgb color={0.956,1.000,0.000}}
\draw[gp path] (9.638,7.645)--(9.676,7.645);
\gpcolor{rgb color={0.733,1.000,0.000}}
\draw[gp path] (9.676,7.645)--(9.715,7.645);
\gpcolor{rgb color={0.511,1.000,0.000}}
\draw[gp path] (9.715,7.645)--(9.753,7.645);
\gpcolor{rgb color={0.289,1.000,0.000}}
\draw[gp path] (9.753,7.645)--(9.791,7.645);
\gpcolor{rgb color={0.067,1.000,0.000}}
\draw[gp path] (9.791,7.645)--(9.829,7.645);
\gpcolor{rgb color={0.000,1.000,0.156}}
\draw[gp path] (9.829,7.645)--(9.867,7.645);
\gpcolor{rgb color={0.000,1.000,0.378}}
\draw[gp path] (9.867,7.645)--(9.905,7.645);
\gpcolor{rgb color={0.000,1.000,0.600}}
\draw[gp path] (9.905,7.645)--(9.944,7.645);
\gpcolor{rgb color={0.000,1.000,0.822}}
\draw[gp path] (9.944,7.645)--(9.982,7.645);
\gpcolor{rgb color={0.000,0.956,1.000}}
\draw[gp path] (9.982,7.645)--(10.020,7.645);
\gpcolor{rgb color={0.000,0.733,1.000}}
\draw[gp path] (10.020,7.645)--(10.058,7.645);
\gpcolor{rgb color={0.000,0.511,1.000}}
\draw[gp path] (10.058,7.645)--(10.096,7.645);
\gpcolor{rgb color={0.000,0.289,1.000}}
\draw[gp path] (10.096,7.645)--(10.134,7.645);
\gpcolor{rgb color={0.000,0.067,1.000}}
\draw[gp path] (10.134,7.645)--(10.173,7.645);
\gpcolor{rgb color={0.156,0.000,1.000}}
\draw[gp path] (10.173,7.645)--(10.211,7.645);
\gpcolor{rgb color={0.378,0.000,1.000}}
\draw[gp path] (10.211,7.645)--(10.249,7.645);
\gpcolor{rgb color={0.600,0.000,1.000}}
\draw[gp path] (10.249,7.645)--(10.287,7.645);
\gpcolor{rgb color={0.822,0.000,1.000}}
\draw[gp path] (10.287,7.645)--(10.325,7.645);
\gpcolor{rgb color={1.000,0.000,0.956}}
\draw[gp path] (10.325,7.645)--(10.363,7.645);
\gpcolor{rgb color={1.000,0.000,0.733}}
\draw[gp path] (10.363,7.645)--(10.402,7.645);
\gpcolor{rgb color={1.000,0.000,0.511}}
\draw[gp path] (10.402,7.645)--(10.440,7.645);
\gpcolor{rgb color={1.000,0.000,0.289}}
\draw[gp path] (10.440,7.645)--(10.478,7.645);
\gpcolor{rgb color={1.000,0.000,0.067}}
\draw[gp path] (10.478,7.645)--(10.516,7.645);
\gpcolor{rgb color={1.000,0.767,0.000}}
\draw[gp path] (9.276,3.968)--(9.097,4.165);
\gpcolor{rgb color={1.000,0.900,0.000}}
\draw[gp path] (9.097,4.165)--(8.791,4.658);
\gpcolor{rgb color={0.833,1.000,0.000}}
\draw[gp path] (8.791,4.658)--(8.697,4.734);
\gpcolor{rgb color={0.633,1.000,0.000}}
\draw[gp path] (8.697,4.734)--(8.786,4.931);
\gpcolor{rgb color={0.300,1.000,0.000}}
\draw[gp path] (8.786,4.931)--(8.890,5.363);
\gpcolor{rgb color={0.000,1.000,0.033}}
\draw[gp path] (8.890,5.363)--(8.638,5.646);
\gpcolor{rgb color={0.000,1.000,0.233}}
\draw[gp path] (8.638,5.646)--(8.199,5.893);
\gpcolor{rgb color={0.000,1.000,0.433}}
\draw[gp path] (8.199,5.893)--(7.674,6.233);
\gpcolor{rgb color={0.000,1.000,0.567}}
\draw[gp path] (7.674,6.233)--(6.997,6.384);
\gpcolor{rgb color={0.000,1.000,0.700}}
\draw[gp path] (6.997,6.384)--(6.164,6.612);
\gpcolor{rgb color={0.000,0.167,1.000}}
\draw[gp path] (6.164,6.612)--(3.995,5.467);
\gpcolor{rgb color={0.100,0.000,1.000}}
\draw[gp path] (3.995,5.467)--(4.309,5.284);
\gpcolor{rgb color={0.233,0.000,1.000}}
\draw[gp path] (4.309,5.284)--(4.537,5.115);
\gpcolor{rgb color={0.500,0.000,1.000}}
\draw[gp path] (4.537,5.115)--(4.734,5.075);
\gpcolor{rgb color={0.967,0.000,1.000}}
\draw[gp path] (4.734,5.075)--(5.496,4.315);
\gpcolor{rgb color={1.000,0.000,0.500}}
\draw[gp path] (5.496,4.315)--(6.430,3.492);
\gpcolor{rgb color={1.000,0.000,0.167}}
\draw[gp path] (6.430,3.492)--(7.142,3.356);
\gpcolor{rgb color={1.000,0.600,0.000}}
\gpsetpointsize{4.00}
\gppoint{gp mark 4}{(9.600,7.645)}
\gpcolor{rgb color={0.000,0.733,1.000}}
\gppoint{gp mark 4}{(10.058,7.645)}
\gpcolor{rgb color={1.000,0.000,0.067}}
\gppoint{gp mark 4}{(10.516,7.645)}
\gpcolor{rgb color={1.000,0.600,0.000}}
\gppoint{gp mark 4}{(9.276,3.968)}
\gpcolor{rgb color={1.000,0.667,0.000}}
\gppoint{gp mark 4}{(9.276,3.968)}
\gpcolor{rgb color={1.000,0.733,0.000}}
\gppoint{gp mark 4}{(9.276,3.968)}
\gpcolor{rgb color={1.000,0.800,0.000}}
\gppoint{gp mark 4}{(9.097,4.165)}
\gpcolor{rgb color={1.000,0.867,0.000}}
\gppoint{gp mark 4}{(9.097,4.165)}
\gpcolor{rgb color={1.000,0.933,0.000}}
\gppoint{gp mark 4}{(8.791,4.658)}
\gpcolor{rgb color={1.000,1.000,0.000}}
\gppoint{gp mark 4}{(8.791,4.658)}
\gpcolor{rgb color={0.933,1.000,0.000}}
\gppoint{gp mark 4}{(8.791,4.658)}
\gpcolor{rgb color={0.867,1.000,0.000}}
\gppoint{gp mark 4}{(8.791,4.658)}
\gpcolor{rgb color={0.800,1.000,0.000}}
\gppoint{gp mark 4}{(8.697,4.734)}
\gpcolor{rgb color={0.733,1.000,0.000}}
\gppoint{gp mark 4}{(8.697,4.734)}
\gpcolor{rgb color={0.667,1.000,0.000}}
\gppoint{gp mark 4}{(8.697,4.734)}
\gpcolor{rgb color={0.600,1.000,0.000}}
\gppoint{gp mark 4}{(8.786,4.931)}
\gpcolor{rgb color={0.533,1.000,0.000}}
\gppoint{gp mark 4}{(8.786,4.931)}
\gpcolor{rgb color={0.467,1.000,0.000}}
\gppoint{gp mark 4}{(8.786,4.931)}
\gpcolor{rgb color={0.400,1.000,0.000}}
\gppoint{gp mark 4}{(8.786,4.931)}
\gpcolor{rgb color={0.333,1.000,0.000}}
\gppoint{gp mark 4}{(8.786,4.931)}
\gpcolor{rgb color={0.267,1.000,0.000}}
\gppoint{gp mark 4}{(8.890,5.363)}
\gpcolor{rgb color={0.200,1.000,0.000}}
\gppoint{gp mark 4}{(8.890,5.363)}
\gpcolor{rgb color={0.133,1.000,0.000}}
\gppoint{gp mark 4}{(8.890,5.363)}
\gpcolor{rgb color={0.067,1.000,0.000}}
\gppoint{gp mark 4}{(8.890,5.363)}
\gpcolor{rgb color={0.000,1.000,0.000}}
\gppoint{gp mark 4}{(8.890,5.363)}
\gpcolor{rgb color={0.000,1.000,0.067}}
\gppoint{gp mark 4}{(8.638,5.646)}
\gpcolor{rgb color={0.000,1.000,0.133}}
\gppoint{gp mark 4}{(8.638,5.646)}
\gpcolor{rgb color={0.000,1.000,0.200}}
\gppoint{gp mark 4}{(8.638,5.646)}
\gpcolor{rgb color={0.000,1.000,0.267}}
\gppoint{gp mark 4}{(8.199,5.893)}
\gpcolor{rgb color={0.000,1.000,0.333}}
\gppoint{gp mark 4}{(8.199,5.893)}
\gpcolor{rgb color={0.000,1.000,0.400}}
\gppoint{gp mark 4}{(8.199,5.893)}
\gpcolor{rgb color={0.000,1.000,0.467}}
\gppoint{gp mark 4}{(7.674,6.233)}
\gpcolor{rgb color={0.000,1.000,0.533}}
\gppoint{gp mark 4}{(7.674,6.233)}
\gpcolor{rgb color={0.000,1.000,0.600}}
\gppoint{gp mark 4}{(6.997,6.384)}
\gpcolor{rgb color={0.000,1.000,0.667}}
\gppoint{gp mark 4}{(6.997,6.384)}
\gpcolor{rgb color={0.000,1.000,0.733}}
\gppoint{gp mark 4}{(6.164,6.612)}
\gpcolor{rgb color={0.000,1.000,0.800}}
\gppoint{gp mark 4}{(6.164,6.612)}
\gpcolor{rgb color={0.000,1.000,0.867}}
\gppoint{gp mark 4}{(6.164,6.612)}
\gpcolor{rgb color={0.000,1.000,0.933}}
\gppoint{gp mark 4}{(6.164,6.612)}
\gpcolor{rgb color={0.000,1.000,1.000}}
\gppoint{gp mark 4}{(6.164,6.612)}
\gpcolor{rgb color={0.000,0.933,1.000}}
\gppoint{gp mark 4}{(6.164,6.612)}
\gpcolor{rgb color={0.000,0.867,1.000}}
\gppoint{gp mark 4}{(6.164,6.612)}
\gpcolor{rgb color={0.000,0.800,1.000}}
\gppoint{gp mark 4}{(6.164,6.612)}
\gpcolor{rgb color={0.000,0.733,1.000}}
\gppoint{gp mark 4}{(6.164,6.612)}
\gpcolor{rgb color={0.000,0.667,1.000}}
\gppoint{gp mark 4}{(6.164,6.612)}
\gpcolor{rgb color={0.000,0.600,1.000}}
\gppoint{gp mark 4}{(6.164,6.612)}
\gpcolor{rgb color={0.000,0.533,1.000}}
\gppoint{gp mark 4}{(6.164,6.612)}
\gpcolor{rgb color={0.000,0.467,1.000}}
\gppoint{gp mark 4}{(6.164,6.612)}
\gpcolor{rgb color={0.000,0.400,1.000}}
\gppoint{gp mark 4}{(6.164,6.612)}
\gpcolor{rgb color={0.000,0.333,1.000}}
\gppoint{gp mark 4}{(6.164,6.612)}
\gpcolor{rgb color={0.000,0.267,1.000}}
\gppoint{gp mark 4}{(6.164,6.612)}
\gpcolor{rgb color={0.000,0.200,1.000}}
\gppoint{gp mark 4}{(6.164,6.612)}
\gpcolor{rgb color={0.000,0.133,1.000}}
\gppoint{gp mark 4}{(3.995,5.467)}
\gpcolor{rgb color={0.000,0.067,1.000}}
\gppoint{gp mark 4}{(3.995,5.467)}
\gpcolor{rgb color={0.000,0.000,1.000}}
\gppoint{gp mark 4}{(3.995,5.467)}
\gpcolor{rgb color={0.067,0.000,1.000}}
\gppoint{gp mark 4}{(3.995,5.467)}
\gpcolor{rgb color={0.133,0.000,1.000}}
\gppoint{gp mark 4}{(4.309,5.284)}
\gpcolor{rgb color={0.200,0.000,1.000}}
\gppoint{gp mark 4}{(4.309,5.284)}
\gpcolor{rgb color={0.267,0.000,1.000}}
\gppoint{gp mark 4}{(4.537,5.115)}
\gpcolor{rgb color={0.333,0.000,1.000}}
\gppoint{gp mark 4}{(4.537,5.115)}
\gpcolor{rgb color={0.400,0.000,1.000}}
\gppoint{gp mark 4}{(4.537,5.115)}
\gpcolor{rgb color={0.467,0.000,1.000}}
\gppoint{gp mark 4}{(4.537,5.115)}
\gpcolor{rgb color={0.533,0.000,1.000}}
\gppoint{gp mark 4}{(4.734,5.075)}
\gpcolor{rgb color={0.600,0.000,1.000}}
\gppoint{gp mark 4}{(4.734,5.075)}
\gpcolor{rgb color={0.667,0.000,1.000}}
\gppoint{gp mark 4}{(4.734,5.075)}
\gpcolor{rgb color={0.733,0.000,1.000}}
\gppoint{gp mark 4}{(4.734,5.075)}
\gpcolor{rgb color={0.800,0.000,1.000}}
\gppoint{gp mark 4}{(4.734,5.075)}
\gpcolor{rgb color={0.867,0.000,1.000}}
\gppoint{gp mark 4}{(4.734,5.075)}
\gpcolor{rgb color={0.933,0.000,1.000}}
\gppoint{gp mark 4}{(4.734,5.075)}
\gpcolor{rgb color={1.000,0.000,1.000}}
\gppoint{gp mark 4}{(5.496,4.315)}
\gpcolor{rgb color={1.000,0.000,0.933}}
\gppoint{gp mark 4}{(5.496,4.315)}
\gpcolor{rgb color={1.000,0.000,0.867}}
\gppoint{gp mark 4}{(5.496,4.315)}
\gpcolor{rgb color={1.000,0.000,0.800}}
\gppoint{gp mark 4}{(5.496,4.315)}
\gpcolor{rgb color={1.000,0.000,0.733}}
\gppoint{gp mark 4}{(5.496,4.315)}
\gpcolor{rgb color={1.000,0.000,0.667}}
\gppoint{gp mark 4}{(5.496,4.315)}
\gpcolor{rgb color={1.000,0.000,0.600}}
\gppoint{gp mark 4}{(5.496,4.315)}
\gpcolor{rgb color={1.000,0.000,0.533}}
\gppoint{gp mark 4}{(5.496,4.315)}
\gpcolor{rgb color={1.000,0.000,0.467}}
\gppoint{gp mark 4}{(6.430,3.492)}
\gpcolor{rgb color={1.000,0.000,0.400}}
\gppoint{gp mark 4}{(6.430,3.492)}
\gpcolor{rgb color={1.000,0.000,0.333}}
\gppoint{gp mark 4}{(6.430,3.492)}
\gpcolor{rgb color={1.000,0.000,0.267}}
\gppoint{gp mark 4}{(6.430,3.492)}
\gpcolor{rgb color={1.000,0.000,0.200}}
\gppoint{gp mark 4}{(6.430,3.492)}
\gpcolor{rgb color={1.000,0.000,0.133}}
\gppoint{gp mark 4}{(7.142,3.356)}
\gpcolor{rgb color={1.000,0.000,0.067}}
\gppoint{gp mark 4}{(7.142,3.356)}
\gpcolor{color=gp lt color border}
\node[gp node right,font={\fontsize{9pt}{10.8pt}\selectfont}] at (9.416,7.337) {ball pos. kalman estimate};
\gpcolor{rgb color={1.000,0.822,0.000}}
\gpsetlinetype{gp lt plot 1}
\draw[gp path] (9.600,7.337)--(9.638,7.337);
\gpcolor{rgb color={0.956,1.000,0.000}}
\draw[gp path] (9.638,7.337)--(9.676,7.337);
\gpcolor{rgb color={0.733,1.000,0.000}}
\draw[gp path] (9.676,7.337)--(9.715,7.337);
\gpcolor{rgb color={0.511,1.000,0.000}}
\draw[gp path] (9.715,7.337)--(9.753,7.337);
\gpcolor{rgb color={0.289,1.000,0.000}}
\draw[gp path] (9.753,7.337)--(9.791,7.337);
\gpcolor{rgb color={0.067,1.000,0.000}}
\draw[gp path] (9.791,7.337)--(9.829,7.337);
\gpcolor{rgb color={0.000,1.000,0.156}}
\draw[gp path] (9.829,7.337)--(9.867,7.337);
\gpcolor{rgb color={0.000,1.000,0.378}}
\draw[gp path] (9.867,7.337)--(9.905,7.337);
\gpcolor{rgb color={0.000,1.000,0.600}}
\draw[gp path] (9.905,7.337)--(9.944,7.337);
\gpcolor{rgb color={0.000,1.000,0.822}}
\draw[gp path] (9.944,7.337)--(9.982,7.337);
\gpcolor{rgb color={0.000,0.956,1.000}}
\draw[gp path] (9.982,7.337)--(10.020,7.337);
\gpcolor{rgb color={0.000,0.733,1.000}}
\draw[gp path] (10.020,7.337)--(10.058,7.337);
\gpcolor{rgb color={0.000,0.511,1.000}}
\draw[gp path] (10.058,7.337)--(10.096,7.337);
\gpcolor{rgb color={0.000,0.289,1.000}}
\draw[gp path] (10.096,7.337)--(10.134,7.337);
\gpcolor{rgb color={0.000,0.067,1.000}}
\draw[gp path] (10.134,7.337)--(10.173,7.337);
\gpcolor{rgb color={0.156,0.000,1.000}}
\draw[gp path] (10.173,7.337)--(10.211,7.337);
\gpcolor{rgb color={0.378,0.000,1.000}}
\draw[gp path] (10.211,7.337)--(10.249,7.337);
\gpcolor{rgb color={0.600,0.000,1.000}}
\draw[gp path] (10.249,7.337)--(10.287,7.337);
\gpcolor{rgb color={0.822,0.000,1.000}}
\draw[gp path] (10.287,7.337)--(10.325,7.337);
\gpcolor{rgb color={1.000,0.000,0.956}}
\draw[gp path] (10.325,7.337)--(10.363,7.337);
\gpcolor{rgb color={1.000,0.000,0.733}}
\draw[gp path] (10.363,7.337)--(10.402,7.337);
\gpcolor{rgb color={1.000,0.000,0.511}}
\draw[gp path] (10.402,7.337)--(10.440,7.337);
\gpcolor{rgb color={1.000,0.000,0.289}}
\draw[gp path] (10.440,7.337)--(10.478,7.337);
\gpcolor{rgb color={1.000,0.000,0.067}}
\draw[gp path] (10.478,7.337)--(10.516,7.337);
\gpcolor{rgb color={1.000,0.633,0.000}}
\draw[gp path] (9.316,3.807)--(9.389,3.871);
\gpcolor{rgb color={1.000,0.700,0.000}}
\draw[gp path] (9.389,3.871)--(9.373,4.020);
\gpcolor{rgb color={1.000,0.767,0.000}}
\draw[gp path] (9.373,4.020)--(9.445,4.090);
\gpcolor{rgb color={1.000,0.833,0.000}}
\draw[gp path] (9.445,4.090)--(9.520,4.164);
\gpcolor{rgb color={1.000,0.900,0.000}}
\draw[gp path] (9.520,4.164)--(9.597,4.239);
\gpcolor{rgb color={1.000,0.967,0.000}}
\draw[gp path] (9.597,4.239)--(9.668,4.309);
\gpcolor{rgb color={0.967,1.000,0.000}}
\draw[gp path] (9.668,4.309)--(9.182,4.216);
\gpcolor{rgb color={0.900,1.000,0.000}}
\draw[gp path] (9.182,4.216)--(9.212,4.268);
\gpcolor{rgb color={0.833,1.000,0.000}}
\draw[gp path] (9.212,4.268)--(8.987,4.636);
\gpcolor{rgb color={0.767,1.000,0.000}}
\draw[gp path] (8.987,4.636)--(9.009,4.714);
\gpcolor{rgb color={0.700,1.000,0.000}}
\draw[gp path] (9.009,4.714)--(9.031,4.793);
\gpcolor{rgb color={0.633,1.000,0.000}}
\draw[gp path] (9.031,4.793)--(8.735,4.802);
\gpcolor{rgb color={0.500,1.000,0.000}}
\draw[gp path] (8.735,4.802)--(8.739,4.943);
\gpcolor{rgb color={0.433,1.000,0.000}}
\draw[gp path] (8.739,4.943)--(8.740,5.012);
\gpcolor{rgb color={0.300,1.000,0.000}}
\draw[gp path] (8.740,5.012)--(8.769,5.072);
\gpcolor{rgb color={0.233,1.000,0.000}}
\draw[gp path] (8.769,5.072)--(8.772,5.135);
\gpcolor{rgb color={0.167,1.000,0.000}}
\draw[gp path] (8.772,5.135)--(8.909,5.416);
\gpcolor{rgb color={0.100,1.000,0.000}}
\draw[gp path] (8.909,5.416)--(8.935,5.501);
\gpcolor{rgb color={0.033,1.000,0.000}}
\draw[gp path] (8.935,5.501)--(8.960,5.586);
\gpcolor{rgb color={0.000,1.000,0.033}}
\draw[gp path] (8.960,5.586)--(8.985,5.671);
\gpcolor{rgb color={0.000,1.000,0.100}}
\draw[gp path] (8.985,5.671)--(9.010,5.755);
\gpcolor{rgb color={0.000,1.000,0.167}}
\draw[gp path] (9.010,5.755)--(9.037,5.846);
\gpcolor{rgb color={0.000,1.000,0.233}}
\draw[gp path] (9.037,5.846)--(8.657,5.720);
\gpcolor{rgb color={0.000,1.000,0.300}}
\draw[gp path] (8.657,5.720)--(8.646,5.785);
\gpcolor{rgb color={0.000,1.000,0.367}}
\draw[gp path] (8.646,5.785)--(8.635,5.849);
\gpcolor{rgb color={0.000,1.000,0.433}}
\draw[gp path] (8.635,5.849)--(8.624,5.913);
\gpcolor{rgb color={0.000,1.000,0.500}}
\draw[gp path] (8.624,5.913)--(8.218,5.940);
\gpcolor{rgb color={0.000,1.000,0.567}}
\draw[gp path] (8.218,5.940)--(8.181,5.998);
\gpcolor{rgb color={0.000,1.000,0.633}}
\draw[gp path] (8.181,5.998)--(7.748,6.234);
\gpcolor{rgb color={0.000,1.000,0.700}}
\draw[gp path] (7.748,6.234)--(7.697,6.304);
\gpcolor{rgb color={0.000,1.000,0.767}}
\draw[gp path] (7.697,6.304)--(7.646,6.374);
\gpcolor{rgb color={0.000,1.000,0.833}}
\draw[gp path] (7.646,6.374)--(7.026,6.425);
\gpcolor{rgb color={0.000,1.000,0.900}}
\draw[gp path] (7.026,6.425)--(6.928,6.491);
\gpcolor{rgb color={0.000,1.000,0.967}}
\draw[gp path] (6.928,6.491)--(6.244,6.618);
\gpcolor{rgb color={0.000,0.967,1.000}}
\draw[gp path] (6.244,6.618)--(6.110,6.679);
\gpcolor{rgb color={0.000,0.900,1.000}}
\draw[gp path] (6.110,6.679)--(5.972,6.742);
\gpcolor{rgb color={0.000,0.833,1.000}}
\draw[gp path] (5.972,6.742)--(5.837,6.803);
\gpcolor{rgb color={0.000,0.767,1.000}}
\draw[gp path] (5.837,6.803)--(5.702,6.865);
\gpcolor{rgb color={0.000,0.700,1.000}}
\draw[gp path] (5.702,6.865)--(5.568,6.926);
\gpcolor{rgb color={0.000,0.633,1.000}}
\draw[gp path] (5.568,6.926)--(5.434,6.986);
\gpcolor{rgb color={0.000,0.567,1.000}}
\draw[gp path] (5.434,6.986)--(5.300,7.047);
\gpcolor{rgb color={0.000,0.500,1.000}}
\draw[gp path] (5.300,7.047)--(5.159,7.111);
\gpcolor{rgb color={0.000,0.433,1.000}}
\draw[gp path] (5.159,7.111)--(5.015,7.176);
\gpcolor{rgb color={0.033,0.000,1.000}}
\draw[gp path] (5.015,7.176)--(4.881,7.237);
\gpcolor{rgb color={0.100,0.000,1.000}}
\draw[gp path] (4.881,7.237)--(3.995,5.467);
\gpcolor{rgb color={0.433,0.000,1.000}}
\draw[gp path] (3.995,5.467)--(4.319,5.272);
\gpcolor{rgb color={0.500,0.000,1.000}}
\draw[gp path] (4.319,5.272)--(4.471,5.137);
\gpcolor{rgb color={0.567,0.000,1.000}}
\draw[gp path] (4.471,5.137)--(4.489,5.119);
\gpcolor{rgb color={0.633,0.000,1.000}}
\draw[gp path] (4.489,5.119)--(4.507,5.100);
\gpcolor{rgb color={0.700,0.000,1.000}}
\draw[gp path] (4.507,5.100)--(4.713,5.071);
\gpcolor{rgb color={0.767,0.000,1.000}}
\draw[gp path] (4.713,5.071)--(4.748,5.053);
\gpcolor{rgb color={0.833,0.000,1.000}}
\draw[gp path] (4.748,5.053)--(4.784,5.035);
\gpcolor{rgb color={0.900,0.000,1.000}}
\draw[gp path] (4.784,5.035)--(4.818,5.018);
\gpcolor{rgb color={0.967,0.000,1.000}}
\draw[gp path] (4.818,5.018)--(4.852,5.002);
\gpcolor{rgb color={1.000,0.000,0.967}}
\draw[gp path] (4.852,5.002)--(4.886,4.985);
\gpcolor{rgb color={1.000,0.000,0.900}}
\draw[gp path] (4.886,4.985)--(4.920,4.968);
\gpcolor{rgb color={1.000,0.000,0.833}}
\draw[gp path] (4.920,4.968)--(4.954,4.951);
\gpcolor{rgb color={1.000,0.000,0.767}}
\draw[gp path] (4.954,4.951)--(5.463,4.299);
\gpcolor{rgb color={1.000,0.000,0.700}}
\draw[gp path] (5.463,4.299)--(5.508,4.239);
\gpcolor{rgb color={1.000,0.000,0.633}}
\draw[gp path] (5.508,4.239)--(5.557,4.175);
\gpcolor{rgb color={1.000,0.000,0.567}}
\draw[gp path] (5.557,4.175)--(5.606,4.111);
\gpcolor{rgb color={1.000,0.000,0.500}}
\draw[gp path] (5.606,4.111)--(5.651,4.052);
\gpcolor{rgb color={1.000,0.000,0.433}}
\draw[gp path] (5.651,4.052)--(5.697,3.992);
\gpcolor{rgb color={1.000,0.000,0.367}}
\draw[gp path] (5.697,3.992)--(5.742,3.932);
\gpcolor{rgb color={1.000,0.000,0.300}}
\draw[gp path] (5.742,3.932)--(5.788,3.872);
\gpcolor{rgb color={1.000,0.000,0.233}}
\draw[gp path] (5.788,3.872)--(6.437,3.437);
\gpcolor{rgb color={1.000,0.000,0.167}}
\draw[gp path] (6.437,3.437)--(6.557,3.268);
\gpcolor{rgb color={1.000,0.600,0.000}}
\gppoint{gp mark 2}{(9.600,7.337)}
\gpcolor{rgb color={0.000,0.733,1.000}}
\gppoint{gp mark 2}{(10.058,7.337)}
\gpcolor{rgb color={1.000,0.000,0.067}}
\gppoint{gp mark 2}{(10.516,7.337)}
\gpcolor{rgb color={1.000,0.600,0.000}}
\gppoint{gp mark 2}{(9.316,3.807)}
\gpcolor{rgb color={1.000,0.667,0.000}}
\gppoint{gp mark 2}{(9.389,3.871)}
\gpcolor{rgb color={1.000,0.733,0.000}}
\gppoint{gp mark 2}{(9.373,4.020)}
\gpcolor{rgb color={1.000,0.800,0.000}}
\gppoint{gp mark 2}{(9.445,4.090)}
\gpcolor{rgb color={1.000,0.867,0.000}}
\gppoint{gp mark 2}{(9.520,4.164)}
\gpcolor{rgb color={1.000,0.933,0.000}}
\gppoint{gp mark 2}{(9.597,4.239)}
\gpcolor{rgb color={1.000,1.000,0.000}}
\gppoint{gp mark 2}{(9.668,4.309)}
\gpcolor{rgb color={0.933,1.000,0.000}}
\gppoint{gp mark 2}{(9.182,4.216)}
\gpcolor{rgb color={0.867,1.000,0.000}}
\gppoint{gp mark 2}{(9.212,4.268)}
\gpcolor{rgb color={0.800,1.000,0.000}}
\gppoint{gp mark 2}{(8.987,4.636)}
\gpcolor{rgb color={0.733,1.000,0.000}}
\gppoint{gp mark 2}{(9.009,4.714)}
\gpcolor{rgb color={0.667,1.000,0.000}}
\gppoint{gp mark 2}{(9.031,4.793)}
\gpcolor{rgb color={0.600,1.000,0.000}}
\gppoint{gp mark 2}{(8.735,4.802)}
\gpcolor{rgb color={0.533,1.000,0.000}}
\gppoint{gp mark 2}{(8.735,4.802)}
\gpcolor{rgb color={0.467,1.000,0.000}}
\gppoint{gp mark 2}{(8.739,4.943)}
\gpcolor{rgb color={0.400,1.000,0.000}}
\gppoint{gp mark 2}{(8.740,5.012)}
\gpcolor{rgb color={0.333,1.000,0.000}}
\gppoint{gp mark 2}{(8.740,5.012)}
\gpcolor{rgb color={0.267,1.000,0.000}}
\gppoint{gp mark 2}{(8.769,5.072)}
\gpcolor{rgb color={0.200,1.000,0.000}}
\gppoint{gp mark 2}{(8.772,5.135)}
\gpcolor{rgb color={0.133,1.000,0.000}}
\gppoint{gp mark 2}{(8.909,5.416)}
\gpcolor{rgb color={0.067,1.000,0.000}}
\gppoint{gp mark 2}{(8.935,5.501)}
\gpcolor{rgb color={0.000,1.000,0.000}}
\gppoint{gp mark 2}{(8.960,5.586)}
\gpcolor{rgb color={0.000,1.000,0.067}}
\gppoint{gp mark 2}{(8.985,5.671)}
\gpcolor{rgb color={0.000,1.000,0.133}}
\gppoint{gp mark 2}{(9.010,5.755)}
\gpcolor{rgb color={0.000,1.000,0.200}}
\gppoint{gp mark 2}{(9.037,5.846)}
\gpcolor{rgb color={0.000,1.000,0.267}}
\gppoint{gp mark 2}{(8.657,5.720)}
\gpcolor{rgb color={0.000,1.000,0.333}}
\gppoint{gp mark 2}{(8.646,5.785)}
\gpcolor{rgb color={0.000,1.000,0.400}}
\gppoint{gp mark 2}{(8.635,5.849)}
\gpcolor{rgb color={0.000,1.000,0.467}}
\gppoint{gp mark 2}{(8.624,5.913)}
\gpcolor{rgb color={0.000,1.000,0.533}}
\gppoint{gp mark 2}{(8.218,5.940)}
\gpcolor{rgb color={0.000,1.000,0.600}}
\gppoint{gp mark 2}{(8.181,5.998)}
\gpcolor{rgb color={0.000,1.000,0.667}}
\gppoint{gp mark 2}{(7.748,6.234)}
\gpcolor{rgb color={0.000,1.000,0.733}}
\gppoint{gp mark 2}{(7.697,6.304)}
\gpcolor{rgb color={0.000,1.000,0.800}}
\gppoint{gp mark 2}{(7.646,6.374)}
\gpcolor{rgb color={0.000,1.000,0.867}}
\gppoint{gp mark 2}{(7.026,6.425)}
\gpcolor{rgb color={0.000,1.000,0.933}}
\gppoint{gp mark 2}{(6.928,6.491)}
\gpcolor{rgb color={0.000,1.000,1.000}}
\gppoint{gp mark 2}{(6.244,6.618)}
\gpcolor{rgb color={0.000,0.933,1.000}}
\gppoint{gp mark 2}{(6.110,6.679)}
\gpcolor{rgb color={0.000,0.867,1.000}}
\gppoint{gp mark 2}{(5.972,6.742)}
\gpcolor{rgb color={0.000,0.800,1.000}}
\gppoint{gp mark 2}{(5.837,6.803)}
\gpcolor{rgb color={0.000,0.733,1.000}}
\gppoint{gp mark 2}{(5.702,6.865)}
\gpcolor{rgb color={0.000,0.667,1.000}}
\gppoint{gp mark 2}{(5.568,6.926)}
\gpcolor{rgb color={0.000,0.600,1.000}}
\gppoint{gp mark 2}{(5.434,6.986)}
\gpcolor{rgb color={0.000,0.533,1.000}}
\gppoint{gp mark 2}{(5.300,7.047)}
\gpcolor{rgb color={0.000,0.467,1.000}}
\gppoint{gp mark 2}{(5.159,7.111)}
\gpcolor{rgb color={0.000,0.400,1.000}}
\gppoint{gp mark 2}{(5.015,7.176)}
\gpcolor{rgb color={0.000,0.333,1.000}}
\gppoint{gp mark 2}{(5.015,7.176)}
\gpcolor{rgb color={0.000,0.267,1.000}}
\gppoint{gp mark 2}{(5.015,7.176)}
\gpcolor{rgb color={0.000,0.200,1.000}}
\gppoint{gp mark 2}{(5.015,7.176)}
\gpcolor{rgb color={0.000,0.133,1.000}}
\gppoint{gp mark 2}{(5.015,7.176)}
\gpcolor{rgb color={0.000,0.067,1.000}}
\gppoint{gp mark 2}{(5.015,7.176)}
\gpcolor{rgb color={0.000,0.000,1.000}}
\gppoint{gp mark 2}{(5.015,7.176)}
\gpcolor{rgb color={0.067,0.000,1.000}}
\gppoint{gp mark 2}{(4.881,7.237)}
\gpcolor{rgb color={0.133,0.000,1.000}}
\gppoint{gp mark 2}{(3.995,5.467)}
\gpcolor{rgb color={0.200,0.000,1.000}}
\gppoint{gp mark 2}{(3.995,5.467)}
\gpcolor{rgb color={0.267,0.000,1.000}}
\gppoint{gp mark 2}{(3.995,5.467)}
\gpcolor{rgb color={0.333,0.000,1.000}}
\gppoint{gp mark 2}{(3.995,5.467)}
\gpcolor{rgb color={0.400,0.000,1.000}}
\gppoint{gp mark 2}{(3.995,5.467)}
\gpcolor{rgb color={0.467,0.000,1.000}}
\gppoint{gp mark 2}{(4.319,5.272)}
\gpcolor{rgb color={0.533,0.000,1.000}}
\gppoint{gp mark 2}{(4.471,5.137)}
\gpcolor{rgb color={0.600,0.000,1.000}}
\gppoint{gp mark 2}{(4.489,5.119)}
\gpcolor{rgb color={0.667,0.000,1.000}}
\gppoint{gp mark 2}{(4.507,5.100)}
\gpcolor{rgb color={0.733,0.000,1.000}}
\gppoint{gp mark 2}{(4.713,5.071)}
\gpcolor{rgb color={0.800,0.000,1.000}}
\gppoint{gp mark 2}{(4.748,5.053)}
\gpcolor{rgb color={0.867,0.000,1.000}}
\gppoint{gp mark 2}{(4.784,5.035)}
\gpcolor{rgb color={0.933,0.000,1.000}}
\gppoint{gp mark 2}{(4.818,5.018)}
\gpcolor{rgb color={1.000,0.000,1.000}}
\gppoint{gp mark 2}{(4.852,5.002)}
\gpcolor{rgb color={1.000,0.000,0.933}}
\gppoint{gp mark 2}{(4.886,4.985)}
\gpcolor{rgb color={1.000,0.000,0.867}}
\gppoint{gp mark 2}{(4.920,4.968)}
\gpcolor{rgb color={1.000,0.000,0.800}}
\gppoint{gp mark 2}{(4.954,4.951)}
\gpcolor{rgb color={1.000,0.000,0.733}}
\gppoint{gp mark 2}{(5.463,4.299)}
\gpcolor{rgb color={1.000,0.000,0.667}}
\gppoint{gp mark 2}{(5.508,4.239)}
\gpcolor{rgb color={1.000,0.000,0.600}}
\gppoint{gp mark 2}{(5.557,4.175)}
\gpcolor{rgb color={1.000,0.000,0.533}}
\gppoint{gp mark 2}{(5.606,4.111)}
\gpcolor{rgb color={1.000,0.000,0.467}}
\gppoint{gp mark 2}{(5.651,4.052)}
\gpcolor{rgb color={1.000,0.000,0.400}}
\gppoint{gp mark 2}{(5.697,3.992)}
\gpcolor{rgb color={1.000,0.000,0.333}}
\gppoint{gp mark 2}{(5.742,3.932)}
\gpcolor{rgb color={1.000,0.000,0.267}}
\gppoint{gp mark 2}{(5.788,3.872)}
\gpcolor{rgb color={1.000,0.000,0.200}}
\gppoint{gp mark 2}{(6.437,3.437)}
\gpcolor{rgb color={1.000,0.000,0.133}}
\gppoint{gp mark 2}{(6.557,3.268)}
\gpcolor{rgb color={1.000,0.000,0.067}}
\gppoint{gp mark 2}{(6.557,3.268)}
\gpcolor{color=gp lt color border}
\gpsetlinetype{gp lt border}
\draw[gp path] (10.700,3.063)--(7.442,1.204);
\draw[gp path] (1.800,2.277)--(7.442,1.204);
\node[gp node center,font={\fontsize{9pt}{10.8pt}\selectfont}] at (3.807,1.276) {x};
\node[gp node center,font={\fontsize{9pt}{10.8pt}\selectfont}] at (10.482,1.865) {y};
\node[gp node center,font={\fontsize{9pt}{10.8pt}\selectfont}] at (0.512,4.755) {z};
\gpfill{rgb color={1.000,0.002,0.000}} (10.869,3.897)--(11.319,3.897)--(11.319,3.919)--(10.869,3.919)--cycle;
\gpfill{rgb color={1.000,0.048,0.000}} (10.869,3.918)--(11.319,3.918)--(11.319,3.941)--(10.869,3.941)--cycle;
\gpfill{rgb color={1.000,0.095,0.000}} (10.869,3.940)--(11.319,3.940)--(11.319,3.962)--(10.869,3.962)--cycle;
\gpfill{rgb color={1.000,0.141,0.000}} (10.869,3.961)--(11.319,3.961)--(11.319,3.984)--(10.869,3.984)--cycle;
\gpfill{rgb color={1.000,0.189,0.000}} (10.869,3.983)--(11.319,3.983)--(11.319,4.006)--(10.869,4.006)--cycle;
\gpfill{rgb color={1.000,0.236,0.000}} (10.869,4.005)--(11.319,4.005)--(11.319,4.027)--(10.869,4.027)--cycle;
\gpfill{rgb color={1.000,0.282,0.000}} (10.869,4.026)--(11.319,4.026)--(11.319,4.049)--(10.869,4.049)--cycle;
\gpfill{rgb color={1.000,0.330,0.000}} (10.869,4.048)--(11.319,4.048)--(11.319,4.070)--(10.869,4.070)--cycle;
\gpfill{rgb color={1.000,0.375,0.000}} (10.869,4.069)--(11.319,4.069)--(11.319,4.092)--(10.869,4.092)--cycle;
\gpfill{rgb color={1.000,0.423,0.000}} (10.869,4.091)--(11.319,4.091)--(11.319,4.114)--(10.869,4.114)--cycle;
\gpfill{rgb color={1.000,0.471,0.000}} (10.869,4.113)--(11.319,4.113)--(11.319,4.135)--(10.869,4.135)--cycle;
\gpfill{rgb color={1.000,0.516,0.000}} (10.869,4.134)--(11.319,4.134)--(11.319,4.157)--(10.869,4.157)--cycle;
\gpfill{rgb color={1.000,0.564,0.000}} (10.869,4.156)--(11.319,4.156)--(11.319,4.178)--(10.869,4.178)--cycle;
\gpfill{rgb color={1.000,0.610,0.000}} (10.869,4.177)--(11.319,4.177)--(11.319,4.200)--(10.869,4.200)--cycle;
\gpfill{rgb color={1.000,0.657,0.000}} (10.869,4.199)--(11.319,4.199)--(11.319,4.222)--(10.869,4.222)--cycle;
\gpfill{rgb color={1.000,0.705,0.000}} (10.869,4.221)--(11.319,4.221)--(11.319,4.243)--(10.869,4.243)--cycle;
\gpfill{rgb color={1.000,0.751,0.000}} (10.869,4.242)--(11.319,4.242)--(11.319,4.265)--(10.869,4.265)--cycle;
\gpfill{rgb color={1.000,0.798,0.000}} (10.869,4.264)--(11.319,4.264)--(11.319,4.286)--(10.869,4.286)--cycle;
\gpfill{rgb color={1.000,0.844,0.000}} (10.869,4.285)--(11.319,4.285)--(11.319,4.308)--(10.869,4.308)--cycle;
\gpfill{rgb color={1.000,0.892,0.000}} (10.869,4.307)--(11.319,4.307)--(11.319,4.330)--(10.869,4.330)--cycle;
\gpfill{rgb color={1.000,0.939,0.000}} (10.869,4.329)--(11.319,4.329)--(11.319,4.351)--(10.869,4.351)--cycle;
\gpfill{rgb color={1.000,0.985,0.000}} (10.869,4.350)--(11.319,4.350)--(11.319,4.373)--(10.869,4.373)--cycle;
\gpfill{rgb color={0.967,1.000,0.000}} (10.869,4.372)--(11.319,4.372)--(11.319,4.395)--(10.869,4.395)--cycle;
\gpfill{rgb color={0.920,1.000,0.000}} (10.869,4.394)--(11.319,4.394)--(11.319,4.416)--(10.869,4.416)--cycle;
\gpfill{rgb color={0.874,1.000,0.000}} (10.869,4.415)--(11.319,4.415)--(11.319,4.438)--(10.869,4.438)--cycle;
\gpfill{rgb color={0.826,1.000,0.000}} (10.869,4.437)--(11.319,4.437)--(11.319,4.459)--(10.869,4.459)--cycle;
\gpfill{rgb color={0.781,1.000,0.000}} (10.869,4.458)--(11.319,4.458)--(11.319,4.481)--(10.869,4.481)--cycle;
\gpfill{rgb color={0.733,1.000,0.000}} (10.869,4.480)--(11.319,4.480)--(11.319,4.503)--(10.869,4.503)--cycle;
\gpfill{rgb color={0.685,1.000,0.000}} (10.869,4.502)--(11.319,4.502)--(11.319,4.524)--(10.869,4.524)--cycle;
\gpfill{rgb color={0.640,1.000,0.000}} (10.869,4.523)--(11.319,4.523)--(11.319,4.546)--(10.869,4.546)--cycle;
\gpfill{rgb color={0.592,1.000,0.000}} (10.869,4.545)--(11.319,4.545)--(11.319,4.567)--(10.869,4.567)--cycle;
\gpfill{rgb color={0.547,1.000,0.000}} (10.869,4.566)--(11.319,4.566)--(11.319,4.589)--(10.869,4.589)--cycle;
\gpfill{rgb color={0.499,1.000,0.000}} (10.869,4.588)--(11.319,4.588)--(11.319,4.611)--(10.869,4.611)--cycle;
\gpfill{rgb color={0.451,1.000,0.000}} (10.869,4.610)--(11.319,4.610)--(11.319,4.632)--(10.869,4.632)--cycle;
\gpfill{rgb color={0.406,1.000,0.000}} (10.869,4.631)--(11.319,4.631)--(11.319,4.654)--(10.869,4.654)--cycle;
\gpfill{rgb color={0.358,1.000,0.000}} (10.869,4.653)--(11.319,4.653)--(11.319,4.675)--(10.869,4.675)--cycle;
\gpfill{rgb color={0.312,1.000,0.000}} (10.869,4.674)--(11.319,4.674)--(11.319,4.697)--(10.869,4.697)--cycle;
\gpfill{rgb color={0.265,1.000,0.000}} (10.869,4.696)--(11.319,4.696)--(11.319,4.719)--(10.869,4.719)--cycle;
\gpfill{rgb color={0.217,1.000,0.000}} (10.869,4.718)--(11.319,4.718)--(11.319,4.740)--(10.869,4.740)--cycle;
\gpfill{rgb color={0.171,1.000,0.000}} (10.869,4.739)--(11.319,4.739)--(11.319,4.762)--(10.869,4.762)--cycle;
\gpfill{rgb color={0.124,1.000,0.000}} (10.869,4.761)--(11.319,4.761)--(11.319,4.783)--(10.869,4.783)--cycle;
\gpfill{rgb color={0.078,1.000,0.000}} (10.869,4.782)--(11.319,4.782)--(11.319,4.805)--(10.869,4.805)--cycle;
\gpfill{rgb color={0.030,1.000,0.000}} (10.869,4.804)--(11.319,4.804)--(11.319,4.827)--(10.869,4.827)--cycle;
\gpfill{rgb color={0.000,1.000,0.017}} (10.869,4.826)--(11.319,4.826)--(11.319,4.848)--(10.869,4.848)--cycle;
\gpfill{rgb color={0.000,1.000,0.063}} (10.869,4.847)--(11.319,4.847)--(11.319,4.870)--(10.869,4.870)--cycle;
\gpfill{rgb color={0.000,1.000,0.111}} (10.869,4.869)--(11.319,4.869)--(11.319,4.892)--(10.869,4.892)--cycle;
\gpfill{rgb color={0.000,1.000,0.158}} (10.869,4.891)--(11.319,4.891)--(11.319,4.913)--(10.869,4.913)--cycle;
\gpfill{rgb color={0.000,1.000,0.204}} (10.869,4.912)--(11.319,4.912)--(11.319,4.935)--(10.869,4.935)--cycle;
\gpfill{rgb color={0.000,1.000,0.252}} (10.869,4.934)--(11.319,4.934)--(11.319,4.956)--(10.869,4.956)--cycle;
\gpfill{rgb color={0.000,1.000,0.297}} (10.869,4.955)--(11.319,4.955)--(11.319,4.978)--(10.869,4.978)--cycle;
\gpfill{rgb color={0.000,1.000,0.345}} (10.869,4.977)--(11.319,4.977)--(11.319,5.000)--(10.869,5.000)--cycle;
\gpfill{rgb color={0.000,1.000,0.393}} (10.869,4.999)--(11.319,4.999)--(11.319,5.021)--(10.869,5.021)--cycle;
\gpfill{rgb color={0.000,1.000,0.438}} (10.869,5.020)--(11.319,5.020)--(11.319,5.043)--(10.869,5.043)--cycle;
\gpfill{rgb color={0.000,1.000,0.486}} (10.869,5.042)--(11.319,5.042)--(11.319,5.064)--(10.869,5.064)--cycle;
\gpfill{rgb color={0.000,1.000,0.531}} (10.869,5.063)--(11.319,5.063)--(11.319,5.086)--(10.869,5.086)--cycle;
\gpfill{rgb color={0.000,1.000,0.579}} (10.869,5.085)--(11.319,5.085)--(11.319,5.108)--(10.869,5.108)--cycle;
\gpfill{rgb color={0.000,1.000,0.627}} (10.869,5.107)--(11.319,5.107)--(11.319,5.129)--(10.869,5.129)--cycle;
\gpfill{rgb color={0.000,1.000,0.672}} (10.869,5.128)--(11.319,5.128)--(11.319,5.151)--(10.869,5.151)--cycle;
\gpfill{rgb color={0.000,1.000,0.720}} (10.869,5.150)--(11.319,5.150)--(11.319,5.172)--(10.869,5.172)--cycle;
\gpfill{rgb color={0.000,1.000,0.766}} (10.869,5.171)--(11.319,5.171)--(11.319,5.194)--(10.869,5.194)--cycle;
\gpfill{rgb color={0.000,1.000,0.813}} (10.869,5.193)--(11.319,5.193)--(11.319,5.216)--(10.869,5.216)--cycle;
\gpfill{rgb color={0.000,1.000,0.861}} (10.869,5.215)--(11.319,5.215)--(11.319,5.237)--(10.869,5.237)--cycle;
\gpfill{rgb color={0.000,1.000,0.907}} (10.869,5.236)--(11.319,5.236)--(11.319,5.259)--(10.869,5.259)--cycle;
\gpfill{rgb color={0.000,1.000,0.954}} (10.869,5.258)--(11.319,5.258)--(11.319,5.281)--(10.869,5.281)--cycle;
\gpfill{rgb color={0.000,0.998,1.000}} (10.869,5.280)--(11.319,5.280)--(11.319,5.302)--(10.869,5.302)--cycle;
\gpfill{rgb color={0.000,0.952,1.000}} (10.869,5.301)--(11.319,5.301)--(11.319,5.324)--(10.869,5.324)--cycle;
\gpfill{rgb color={0.000,0.905,1.000}} (10.869,5.323)--(11.319,5.323)--(11.319,5.345)--(10.869,5.345)--cycle;
\gpfill{rgb color={0.000,0.859,1.000}} (10.869,5.344)--(11.319,5.344)--(11.319,5.367)--(10.869,5.367)--cycle;
\gpfill{rgb color={0.000,0.811,1.000}} (10.869,5.366)--(11.319,5.366)--(11.319,5.389)--(10.869,5.389)--cycle;
\gpfill{rgb color={0.000,0.764,1.000}} (10.869,5.388)--(11.319,5.388)--(11.319,5.410)--(10.869,5.410)--cycle;
\gpfill{rgb color={0.000,0.718,1.000}} (10.869,5.409)--(11.319,5.409)--(11.319,5.432)--(10.869,5.432)--cycle;
\gpfill{rgb color={0.000,0.670,1.000}} (10.869,5.431)--(11.319,5.431)--(11.319,5.453)--(10.869,5.453)--cycle;
\gpfill{rgb color={0.000,0.625,1.000}} (10.869,5.452)--(11.319,5.452)--(11.319,5.475)--(10.869,5.475)--cycle;
\gpfill{rgb color={0.000,0.577,1.000}} (10.869,5.474)--(11.319,5.474)--(11.319,5.497)--(10.869,5.497)--cycle;
\gpfill{rgb color={0.000,0.529,1.000}} (10.869,5.496)--(11.319,5.496)--(11.319,5.518)--(10.869,5.518)--cycle;
\gpfill{rgb color={0.000,0.484,1.000}} (10.869,5.517)--(11.319,5.517)--(11.319,5.540)--(10.869,5.540)--cycle;
\gpfill{rgb color={0.000,0.436,1.000}} (10.869,5.539)--(11.319,5.539)--(11.319,5.561)--(10.869,5.561)--cycle;
\gpfill{rgb color={0.000,0.390,1.000}} (10.869,5.560)--(11.319,5.560)--(11.319,5.583)--(10.869,5.583)--cycle;
\gpfill{rgb color={0.000,0.343,1.000}} (10.869,5.582)--(11.319,5.582)--(11.319,5.605)--(10.869,5.605)--cycle;
\gpfill{rgb color={0.000,0.295,1.000}} (10.869,5.604)--(11.319,5.604)--(11.319,5.626)--(10.869,5.626)--cycle;
\gpfill{rgb color={0.000,0.249,1.000}} (10.869,5.625)--(11.319,5.625)--(11.319,5.648)--(10.869,5.648)--cycle;
\gpfill{rgb color={0.000,0.202,1.000}} (10.869,5.647)--(11.319,5.647)--(11.319,5.669)--(10.869,5.669)--cycle;
\gpfill{rgb color={0.000,0.156,1.000}} (10.869,5.668)--(11.319,5.668)--(11.319,5.691)--(10.869,5.691)--cycle;
\gpfill{rgb color={0.000,0.108,1.000}} (10.869,5.690)--(11.319,5.690)--(11.319,5.713)--(10.869,5.713)--cycle;
\gpfill{rgb color={0.000,0.061,1.000}} (10.869,5.712)--(11.319,5.712)--(11.319,5.734)--(10.869,5.734)--cycle;
\gpfill{rgb color={0.000,0.015,1.000}} (10.869,5.733)--(11.319,5.733)--(11.319,5.756)--(10.869,5.756)--cycle;
\gpfill{rgb color={0.033,0.000,1.000}} (10.869,5.755)--(11.319,5.755)--(11.319,5.778)--(10.869,5.778)--cycle;
\gpfill{rgb color={0.080,0.000,1.000}} (10.869,5.777)--(11.319,5.777)--(11.319,5.799)--(10.869,5.799)--cycle;
\gpfill{rgb color={0.126,0.000,1.000}} (10.869,5.798)--(11.319,5.798)--(11.319,5.821)--(10.869,5.821)--cycle;
\gpfill{rgb color={0.174,0.000,1.000}} (10.869,5.820)--(11.319,5.820)--(11.319,5.842)--(10.869,5.842)--cycle;
\gpfill{rgb color={0.219,0.000,1.000}} (10.869,5.841)--(11.319,5.841)--(11.319,5.864)--(10.869,5.864)--cycle;
\gpfill{rgb color={0.267,0.000,1.000}} (10.869,5.863)--(11.319,5.863)--(11.319,5.886)--(10.869,5.886)--cycle;
\gpfill{rgb color={0.315,0.000,1.000}} (10.869,5.885)--(11.319,5.885)--(11.319,5.907)--(10.869,5.907)--cycle;
\gpfill{rgb color={0.360,0.000,1.000}} (10.869,5.906)--(11.319,5.906)--(11.319,5.929)--(10.869,5.929)--cycle;
\gpfill{rgb color={0.408,0.000,1.000}} (10.869,5.928)--(11.319,5.928)--(11.319,5.950)--(10.869,5.950)--cycle;
\gpfill{rgb color={0.453,0.000,1.000}} (10.869,5.949)--(11.319,5.949)--(11.319,5.972)--(10.869,5.972)--cycle;
\gpfill{rgb color={0.501,0.000,1.000}} (10.869,5.971)--(11.319,5.971)--(11.319,5.994)--(10.869,5.994)--cycle;
\gpfill{rgb color={0.549,0.000,1.000}} (10.869,5.993)--(11.319,5.993)--(11.319,6.015)--(10.869,6.015)--cycle;
\gpfill{rgb color={0.594,0.000,1.000}} (10.869,6.014)--(11.319,6.014)--(11.319,6.037)--(10.869,6.037)--cycle;
\gpfill{rgb color={0.642,0.000,1.000}} (10.869,6.036)--(11.319,6.036)--(11.319,6.058)--(10.869,6.058)--cycle;
\gpfill{rgb color={0.688,0.000,1.000}} (10.869,6.057)--(11.319,6.057)--(11.319,6.080)--(10.869,6.080)--cycle;
\gpfill{rgb color={0.735,0.000,1.000}} (10.869,6.079)--(11.319,6.079)--(11.319,6.102)--(10.869,6.102)--cycle;
\gpfill{rgb color={0.783,0.000,1.000}} (10.869,6.101)--(11.319,6.101)--(11.319,6.123)--(10.869,6.123)--cycle;
\gpfill{rgb color={0.829,0.000,1.000}} (10.869,6.122)--(11.319,6.122)--(11.319,6.145)--(10.869,6.145)--cycle;
\gpfill{rgb color={0.876,0.000,1.000}} (10.869,6.144)--(11.319,6.144)--(11.319,6.166)--(10.869,6.166)--cycle;
\gpfill{rgb color={0.922,0.000,1.000}} (10.869,6.165)--(11.319,6.165)--(11.319,6.188)--(10.869,6.188)--cycle;
\gpfill{rgb color={0.970,0.000,1.000}} (10.869,6.187)--(11.319,6.187)--(11.319,6.210)--(10.869,6.210)--cycle;
\gpfill{rgb color={1.000,0.000,0.983}} (10.869,6.209)--(11.319,6.209)--(11.319,6.231)--(10.869,6.231)--cycle;
\gpfill{rgb color={1.000,0.000,0.937}} (10.869,6.230)--(11.319,6.230)--(11.319,6.253)--(10.869,6.253)--cycle;
\gpfill{rgb color={1.000,0.000,0.889}} (10.869,6.252)--(11.319,6.252)--(11.319,6.275)--(10.869,6.275)--cycle;
\gpfill{rgb color={1.000,0.000,0.842}} (10.869,6.274)--(11.319,6.274)--(11.319,6.296)--(10.869,6.296)--cycle;
\gpfill{rgb color={1.000,0.000,0.796}} (10.869,6.295)--(11.319,6.295)--(11.319,6.318)--(10.869,6.318)--cycle;
\gpfill{rgb color={1.000,0.000,0.748}} (10.869,6.317)--(11.319,6.317)--(11.319,6.339)--(10.869,6.339)--cycle;
\gpfill{rgb color={1.000,0.000,0.703}} (10.869,6.338)--(11.319,6.338)--(11.319,6.361)--(10.869,6.361)--cycle;
\gpfill{rgb color={1.000,0.000,0.655}} (10.869,6.360)--(11.319,6.360)--(11.319,6.383)--(10.869,6.383)--cycle;
\gpfill{rgb color={1.000,0.000,0.607}} (10.869,6.382)--(11.319,6.382)--(11.319,6.404)--(10.869,6.404)--cycle;
\gpfill{rgb color={1.000,0.000,0.562}} (10.869,6.403)--(11.319,6.403)--(11.319,6.426)--(10.869,6.426)--cycle;
\gpfill{rgb color={1.000,0.000,0.514}} (10.869,6.425)--(11.319,6.425)--(11.319,6.447)--(10.869,6.447)--cycle;
\gpfill{rgb color={1.000,0.000,0.469}} (10.869,6.446)--(11.319,6.446)--(11.319,6.469)--(10.869,6.469)--cycle;
\gpfill{rgb color={1.000,0.000,0.421}} (10.869,6.468)--(11.319,6.468)--(11.319,6.491)--(10.869,6.491)--cycle;
\gpfill{rgb color={1.000,0.000,0.373}} (10.869,6.490)--(11.319,6.490)--(11.319,6.512)--(10.869,6.512)--cycle;
\gpfill{rgb color={1.000,0.000,0.328}} (10.869,6.511)--(11.319,6.511)--(11.319,6.534)--(10.869,6.534)--cycle;
\gpfill{rgb color={1.000,0.000,0.280}} (10.869,6.533)--(11.319,6.533)--(11.319,6.555)--(10.869,6.555)--cycle;
\gpfill{rgb color={1.000,0.000,0.234}} (10.869,6.554)--(11.319,6.554)--(11.319,6.577)--(10.869,6.577)--cycle;
\gpfill{rgb color={1.000,0.000,0.187}} (10.869,6.576)--(11.319,6.576)--(11.319,6.599)--(10.869,6.599)--cycle;
\gpfill{rgb color={1.000,0.000,0.139}} (10.869,6.598)--(11.319,6.598)--(11.319,6.620)--(10.869,6.620)--cycle;
\gpfill{rgb color={1.000,0.000,0.093}} (10.869,6.619)--(11.319,6.619)--(11.319,6.642)--(10.869,6.642)--cycle;
\gpfill{rgb color={1.000,0.000,0.046}} (10.869,6.641)--(11.319,6.641)--(11.319,6.663)--(10.869,6.663)--cycle;
\draw[gp path] (10.869,3.897)--(11.319,3.897)--(11.319,6.663)--(10.869,6.663)--cycle;
\draw[gp path] (11.319,3.897)--(11.139,3.897);
\node[gp node left,font={\fontsize{9pt}{10.8pt}\selectfont}] at (11.503,3.897) { 62};
\draw[gp path] (10.869,3.897)--(11.049,3.897);
\draw[gp path] (11.319,4.204)--(11.139,4.204);
\node[gp node left,font={\fontsize{9pt}{10.8pt}\selectfont}] at (11.503,4.204) { 64};
\draw[gp path] (10.869,4.204)--(11.049,4.204);
\draw[gp path] (11.319,4.511)--(11.139,4.511);
\node[gp node left,font={\fontsize{9pt}{10.8pt}\selectfont}] at (11.503,4.511) { 66};
\draw[gp path] (10.869,4.511)--(11.049,4.511);
\draw[gp path] (11.319,4.819)--(11.139,4.819);
\node[gp node left,font={\fontsize{9pt}{10.8pt}\selectfont}] at (11.503,4.819) { 68};
\draw[gp path] (10.869,4.819)--(11.049,4.819);
\draw[gp path] (11.319,5.126)--(11.139,5.126);
\node[gp node left,font={\fontsize{9pt}{10.8pt}\selectfont}] at (11.503,5.126) { 70};
\draw[gp path] (10.869,5.126)--(11.049,5.126);
\draw[gp path] (11.319,5.433)--(11.139,5.433);
\node[gp node left,font={\fontsize{9pt}{10.8pt}\selectfont}] at (11.503,5.433) { 72};
\draw[gp path] (10.869,5.433)--(11.049,5.433);
\draw[gp path] (11.319,5.741)--(11.139,5.741);
\node[gp node left,font={\fontsize{9pt}{10.8pt}\selectfont}] at (11.503,5.741) { 74};
\draw[gp path] (10.869,5.741)--(11.049,5.741);
\draw[gp path] (11.319,6.048)--(11.139,6.048);
\node[gp node left,font={\fontsize{9pt}{10.8pt}\selectfont}] at (11.503,6.048) { 76};
\draw[gp path] (10.869,6.048)--(11.049,6.048);
\draw[gp path] (11.319,6.355)--(11.139,6.355);
\node[gp node left,font={\fontsize{9pt}{10.8pt}\selectfont}] at (11.503,6.355) { 78};
\draw[gp path] (10.869,6.355)--(11.049,6.355);
\draw[gp path] (11.319,6.663)--(11.139,6.663);
\node[gp node left,font={\fontsize{9pt}{10.8pt}\selectfont}] at (11.503,6.663) { 80};
\draw[gp path] (10.869,6.663)--(11.049,6.663);
\node[gp node center,rotate=-270,font={\fontsize{9pt}{10.8pt}\selectfont}] at (12.147,5.280) {time (s)};
%% coordinates of the plot area
\gpdefrectangularnode{gp plot 1}{\pgfpoint{1.800cm}{0.771cm}}{\pgfpoint{10.700cm}{8.287cm}}
\end{tikzpicture}
%% gnuplot variables

        }
        \caption{Detected path of the ball and Kalman estimate.}
        \label{fig:ball_kalman_3d}
    \end{subfigure}~
    \begin{subfigure}[b]{0.49\textwidth}
        \resizebox{\columnwidth}{!}{%
            \begin{tikzpicture}[gnuplot]
%% generated with GNUPLOT 4.6p4 (Lua 5.1; terminal rev. 99, script rev. 100)
%% Thu 28 May 2015 09:06:58 AM CEST
\path (0.000,0.000) rectangle (12.500,8.750);
\gpcolor{color=gp lt color border}
\gpsetlinetype{gp lt border}
\gpsetlinewidth{1.00}
\draw[gp path] (1.688,0.985)--(1.868,0.985);
\draw[gp path] (11.947,0.985)--(11.767,0.985);
\node[gp node right,font={\fontsize{9pt}{10.8pt}\selectfont}] at (1.504,0.985) { 0};
\draw[gp path] (1.688,2.218)--(1.868,2.218);
\draw[gp path] (11.947,2.218)--(11.767,2.218);
\node[gp node right,font={\fontsize{9pt}{10.8pt}\selectfont}] at (1.504,2.218) { 0.05};
\draw[gp path] (1.688,3.450)--(1.868,3.450);
\draw[gp path] (11.947,3.450)--(11.767,3.450);
\node[gp node right,font={\fontsize{9pt}{10.8pt}\selectfont}] at (1.504,3.450) { 0.1};
\draw[gp path] (1.688,4.683)--(1.868,4.683);
\draw[gp path] (11.947,4.683)--(11.767,4.683);
\node[gp node right,font={\fontsize{9pt}{10.8pt}\selectfont}] at (1.504,4.683) { 0.15};
\draw[gp path] (1.688,5.916)--(1.868,5.916);
\draw[gp path] (11.947,5.916)--(11.767,5.916);
\node[gp node right,font={\fontsize{9pt}{10.8pt}\selectfont}] at (1.504,5.916) { 0.2};
\draw[gp path] (1.688,7.148)--(1.868,7.148);
\draw[gp path] (11.947,7.148)--(11.767,7.148);
\node[gp node right,font={\fontsize{9pt}{10.8pt}\selectfont}] at (1.504,7.148) { 0.25};
\draw[gp path] (1.688,8.381)--(1.868,8.381);
\draw[gp path] (11.947,8.381)--(11.767,8.381);
\node[gp node right,font={\fontsize{9pt}{10.8pt}\selectfont}] at (1.504,8.381) { 0.3};
\draw[gp path] (1.688,0.985)--(1.688,1.165);
\draw[gp path] (1.688,8.381)--(1.688,8.201);
\node[gp node center,font={\fontsize{9pt}{10.8pt}\selectfont}] at (1.688,0.677) { 64};
\draw[gp path] (2.970,0.985)--(2.970,1.165);
\draw[gp path] (2.970,8.381)--(2.970,8.201);
\node[gp node center,font={\fontsize{9pt}{10.8pt}\selectfont}] at (2.970,0.677) { 66};
\draw[gp path] (4.253,0.985)--(4.253,1.165);
\draw[gp path] (4.253,8.381)--(4.253,8.201);
\node[gp node center,font={\fontsize{9pt}{10.8pt}\selectfont}] at (4.253,0.677) { 68};
\draw[gp path] (5.535,0.985)--(5.535,1.165);
\draw[gp path] (5.535,8.381)--(5.535,8.201);
\node[gp node center,font={\fontsize{9pt}{10.8pt}\selectfont}] at (5.535,0.677) { 70};
\draw[gp path] (6.818,0.985)--(6.818,1.165);
\draw[gp path] (6.818,8.381)--(6.818,8.201);
\node[gp node center,font={\fontsize{9pt}{10.8pt}\selectfont}] at (6.818,0.677) { 72};
\draw[gp path] (8.100,0.985)--(8.100,1.165);
\draw[gp path] (8.100,8.381)--(8.100,8.201);
\node[gp node center,font={\fontsize{9pt}{10.8pt}\selectfont}] at (8.100,0.677) { 74};
\draw[gp path] (9.382,0.985)--(9.382,1.165);
\draw[gp path] (9.382,8.381)--(9.382,8.201);
\node[gp node center,font={\fontsize{9pt}{10.8pt}\selectfont}] at (9.382,0.677) { 76};
\draw[gp path] (10.665,0.985)--(10.665,1.165);
\draw[gp path] (10.665,8.381)--(10.665,8.201);
\node[gp node center,font={\fontsize{9pt}{10.8pt}\selectfont}] at (10.665,0.677) { 78};
\draw[gp path] (11.947,0.985)--(11.947,1.165);
\draw[gp path] (11.947,8.381)--(11.947,8.201);
\node[gp node center,font={\fontsize{9pt}{10.8pt}\selectfont}] at (11.947,0.677) { 80};
\draw[gp path] (1.688,8.381)--(1.688,0.985)--(11.947,0.985)--(11.947,8.381)--cycle;
\node[gp node center,rotate=-270,font={\fontsize{9pt}{10.8pt}\selectfont}] at (0.246,4.683) {distance (m)};
\node[gp node center,font={\fontsize{9pt}{10.8pt}\selectfont}] at (6.817,0.215) {time (s)};
\gpcolor{rgb color={1.000,0.075,0.000}}
\gpsetlinetype{gp lt plot 0}
\draw[gp path] (1.688,1.463)--(1.816,1.242);
\gpcolor{rgb color={1.000,0.150,0.000}}
\draw[gp path] (1.816,1.242)--(1.944,1.749);
\gpcolor{rgb color={1.000,0.225,0.000}}
\draw[gp path] (1.944,1.749)--(2.073,1.681);
\gpcolor{rgb color={1.000,0.300,0.000}}
\draw[gp path] (2.073,1.681)--(2.201,4.179);
\gpcolor{rgb color={1.000,0.375,0.000}}
\draw[gp path] (2.201,4.179)--(2.329,3.947);
\gpcolor{rgb color={1.000,0.450,0.000}}
\draw[gp path] (2.329,3.947)--(2.457,3.921);
\gpcolor{rgb color={1.000,0.525,0.000}}
\draw[gp path] (2.457,3.921)--(2.586,3.680);
\gpcolor{rgb color={1.000,0.600,0.000}}
\draw[gp path] (2.586,3.680)--(2.714,1.647);
\gpcolor{rgb color={1.000,0.675,0.000}}
\draw[gp path] (2.714,1.647)--(2.842,1.625);
\gpcolor{rgb color={1.000,0.750,0.000}}
\draw[gp path] (2.842,1.625)--(2.970,1.869);
\gpcolor{rgb color={1.000,0.825,0.000}}
\draw[gp path] (2.970,1.869)--(3.099,1.484);
\gpcolor{rgb color={1.000,0.900,0.000}}
\draw[gp path] (3.099,1.484)--(3.227,1.484);
\gpcolor{rgb color={1.000,0.975,0.000}}
\draw[gp path] (3.227,1.484)--(3.355,1.432);
\gpcolor{rgb color={0.950,1.000,0.000}}
\draw[gp path] (3.355,1.432)--(3.483,1.754);
\gpcolor{rgb color={0.875,1.000,0.000}}
\draw[gp path] (3.483,1.754)--(3.612,1.754);
\gpcolor{rgb color={0.800,1.000,0.000}}
\draw[gp path] (3.612,1.754)--(3.740,2.247);
\gpcolor{rgb color={0.725,1.000,0.000}}
\draw[gp path] (3.740,2.247)--(3.868,1.943);
\gpcolor{rgb color={0.650,1.000,0.000}}
\draw[gp path] (3.868,1.943)--(3.996,1.229);
\gpcolor{rgb color={0.575,1.000,0.000}}
\draw[gp path] (3.996,1.229)--(4.125,1.623);
\gpcolor{rgb color={0.500,1.000,0.000}}
\draw[gp path] (4.125,1.623)--(4.253,2.019);
\gpcolor{rgb color={0.425,1.000,0.000}}
\draw[gp path] (4.253,2.019)--(4.381,1.719);
\gpcolor{rgb color={0.350,1.000,0.000}}
\draw[gp path] (4.381,1.719)--(4.509,1.701);
\gpcolor{rgb color={0.275,1.000,0.000}}
\draw[gp path] (4.509,1.701)--(4.637,1.894);
\gpcolor{rgb color={0.200,1.000,0.000}}
\draw[gp path] (4.637,1.894)--(4.766,2.245);
\gpcolor{rgb color={0.125,1.000,0.000}}
\draw[gp path] (4.766,2.245)--(4.894,1.970);
\gpcolor{rgb color={0.050,1.000,0.000}}
\draw[gp path] (4.894,1.970)--(5.022,1.758);
\gpcolor{rgb color={0.000,1.000,0.025}}
\draw[gp path] (5.022,1.758)--(5.150,3.245);
\gpcolor{rgb color={0.000,1.000,0.100}}
\draw[gp path] (5.150,3.245)--(5.279,2.668);
\gpcolor{rgb color={0.000,1.000,0.175}}
\draw[gp path] (5.279,2.668)--(5.407,3.811);
\gpcolor{rgb color={0.000,1.000,0.250}}
\draw[gp path] (5.407,3.811)--(5.535,2.588);
\gpcolor{rgb color={0.000,1.000,0.325}}
\draw[gp path] (5.535,2.588)--(5.663,4.402);
\gpcolor{rgb color={0.000,1.000,0.400}}
\draw[gp path] (5.663,4.402)--(5.792,4.097);
\gpcolor{rgb color={0.000,1.000,0.475}}
\draw[gp path] (5.792,4.097)--(5.920,3.061);
\gpcolor{rgb color={0.000,1.000,0.550}}
\draw[gp path] (5.920,3.061)--(6.048,2.683);
\gpcolor{rgb color={0.000,1.000,0.625}}
\draw[gp path] (6.048,2.683)--(6.176,1.141);
\gpcolor{rgb color={0.000,1.000,0.700}}
\draw[gp path] (6.176,1.141)--(6.305,1.340);
\gpcolor{rgb color={0.000,1.000,0.775}}
\draw[gp path] (6.305,1.340)--(6.433,1.805);
\gpcolor{rgb color={0.000,1.000,0.850}}
\draw[gp path] (6.433,1.805)--(6.561,2.264);
\gpcolor{rgb color={0.000,1.000,0.925}}
\draw[gp path] (6.561,2.264)--(6.689,2.727);
\gpcolor{rgb color={0.000,1.000,1.000}}
\draw[gp path] (6.689,2.727)--(6.818,3.186);
\gpcolor{rgb color={0.000,0.925,1.000}}
\draw[gp path] (6.818,3.186)--(6.946,3.644);
\gpcolor{rgb color={0.000,0.850,1.000}}
\draw[gp path] (6.946,3.644)--(7.074,4.101);
\gpcolor{rgb color={0.000,0.775,1.000}}
\draw[gp path] (7.074,4.101)--(7.202,4.584);
\gpcolor{rgb color={0.000,0.700,1.000}}
\draw[gp path] (7.202,4.584)--(7.330,5.076);
\gpcolor{rgb color={0.000,0.625,1.000}}
\draw[gp path] (7.330,5.076)--(7.459,5.076);
\gpcolor{rgb color={0.000,0.550,1.000}}
\draw[gp path] (7.459,5.076)--(7.587,5.076);
\gpcolor{rgb color={0.000,0.475,1.000}}
\draw[gp path] (7.587,5.076)--(7.715,5.076);
\gpcolor{rgb color={0.000,0.400,1.000}}
\draw[gp path] (7.715,5.076)--(7.843,7.617);
\gpcolor{rgb color={0.000,0.325,1.000}}
\draw[gp path] (7.843,7.617)--(7.972,7.617);
\gpcolor{rgb color={0.000,0.250,1.000}}
\draw[gp path] (7.972,7.617)--(8.100,7.617);
\gpcolor{rgb color={0.000,0.175,1.000}}
\draw[gp path] (8.100,7.617)--(8.228,7.891);
\gpcolor{rgb color={0.000,0.100,1.000}}
\draw[gp path] (8.228,7.891)--(8.356,2.339);
\gpcolor{rgb color={0.000,0.025,1.000}}
\draw[gp path] (8.356,2.339)--(8.485,2.339);
\gpcolor{rgb color={0.050,0.000,1.000}}
\draw[gp path] (8.485,2.339)--(8.613,3.638);
\gpcolor{rgb color={0.125,0.000,1.000}}
\draw[gp path] (8.613,3.638)--(8.741,3.638);
\gpcolor{rgb color={0.200,0.000,1.000}}
\draw[gp path] (8.741,3.638)--(8.869,3.638);
\gpcolor{rgb color={0.275,0.000,1.000}}
\draw[gp path] (8.869,3.638)--(8.998,2.200);
\gpcolor{rgb color={0.350,0.000,1.000}}
\draw[gp path] (8.998,2.200)--(9.126,1.854);
\gpcolor{rgb color={0.425,0.000,1.000}}
\draw[gp path] (9.126,1.854)--(9.254,1.714);
\gpcolor{rgb color={0.500,0.000,1.000}}
\draw[gp path] (9.254,1.714)--(9.382,1.580);
\gpcolor{rgb color={0.575,0.000,1.000}}
\draw[gp path] (9.382,1.580)--(9.510,1.058);
\gpcolor{rgb color={0.650,0.000,1.000}}
\draw[gp path] (9.510,1.058)--(9.639,1.236);
\gpcolor{rgb color={0.725,0.000,1.000}}
\draw[gp path] (9.639,1.236)--(9.767,1.431);
\gpcolor{rgb color={0.800,0.000,1.000}}
\draw[gp path] (9.767,1.431)--(9.895,1.616);
\gpcolor{rgb color={0.875,0.000,1.000}}
\draw[gp path] (9.895,1.616)--(10.023,5.336);
\gpcolor{rgb color={0.950,0.000,1.000}}
\draw[gp path] (10.023,5.336)--(10.152,5.159);
\gpcolor{rgb color={1.000,0.000,0.975}}
\draw[gp path] (10.152,5.159)--(10.280,4.982);
\gpcolor{rgb color={1.000,0.000,0.900}}
\draw[gp path] (10.280,4.982)--(10.408,4.805);
\gpcolor{rgb color={1.000,0.000,0.825}}
\draw[gp path] (10.408,4.805)--(10.536,1.137);
\gpcolor{rgb color={1.000,0.000,0.750}}
\draw[gp path] (10.536,1.137)--(10.665,1.545);
\gpcolor{rgb color={1.000,0.000,0.675}}
\draw[gp path] (10.665,1.545)--(10.793,2.000);
\gpcolor{rgb color={1.000,0.000,0.600}}
\draw[gp path] (10.793,2.000)--(10.921,2.459);
\gpcolor{rgb color={1.000,0.000,0.525}}
\draw[gp path] (10.921,2.459)--(11.049,3.818);
\gpcolor{rgb color={1.000,0.000,0.450}}
\draw[gp path] (11.049,3.818)--(11.178,3.493);
\gpcolor{rgb color={1.000,0.000,0.375}}
\draw[gp path] (11.178,3.493)--(11.306,3.204);
\gpcolor{rgb color={1.000,0.000,0.300}}
\draw[gp path] (11.306,3.204)--(11.434,2.966);
\gpcolor{rgb color={1.000,0.000,0.225}}
\draw[gp path] (11.434,2.966)--(11.562,1.262);
\gpcolor{rgb color={1.000,0.000,0.150}}
\draw[gp path] (11.562,1.262)--(11.691,1.936);
\gpcolor{rgb color={1.000,0.000,0.075}}
\draw[gp path] (11.691,1.936)--(11.819,1.936);
\gpcolor{rgb color={1.000,0.000,0.000}}
\draw[gp path] (11.819,1.936)--(11.947,2.217);
\gpsetpointsize{4.00}
\gppoint{gp mark 7}{(1.688,1.463)}
\gpcolor{rgb color={1.000,0.075,0.000}}
\gppoint{gp mark 7}{(1.816,1.242)}
\gpcolor{rgb color={1.000,0.150,0.000}}
\gppoint{gp mark 7}{(1.944,1.749)}
\gpcolor{rgb color={1.000,0.225,0.000}}
\gppoint{gp mark 7}{(2.073,1.681)}
\gpcolor{rgb color={1.000,0.300,0.000}}
\gppoint{gp mark 7}{(2.201,4.179)}
\gpcolor{rgb color={1.000,0.375,0.000}}
\gppoint{gp mark 7}{(2.329,3.947)}
\gpcolor{rgb color={1.000,0.450,0.000}}
\gppoint{gp mark 7}{(2.457,3.921)}
\gpcolor{rgb color={1.000,0.525,0.000}}
\gppoint{gp mark 7}{(2.586,3.680)}
\gpcolor{rgb color={1.000,0.600,0.000}}
\gppoint{gp mark 7}{(2.714,1.647)}
\gpcolor{rgb color={1.000,0.675,0.000}}
\gppoint{gp mark 7}{(2.842,1.625)}
\gpcolor{rgb color={1.000,0.750,0.000}}
\gppoint{gp mark 7}{(2.970,1.869)}
\gpcolor{rgb color={1.000,0.825,0.000}}
\gppoint{gp mark 7}{(3.099,1.484)}
\gpcolor{rgb color={1.000,0.900,0.000}}
\gppoint{gp mark 7}{(3.227,1.484)}
\gpcolor{rgb color={1.000,0.975,0.000}}
\gppoint{gp mark 7}{(3.355,1.432)}
\gpcolor{rgb color={0.950,1.000,0.000}}
\gppoint{gp mark 7}{(3.483,1.754)}
\gpcolor{rgb color={0.875,1.000,0.000}}
\gppoint{gp mark 7}{(3.612,1.754)}
\gpcolor{rgb color={0.800,1.000,0.000}}
\gppoint{gp mark 7}{(3.740,2.247)}
\gpcolor{rgb color={0.725,1.000,0.000}}
\gppoint{gp mark 7}{(3.868,1.943)}
\gpcolor{rgb color={0.650,1.000,0.000}}
\gppoint{gp mark 7}{(3.996,1.229)}
\gpcolor{rgb color={0.575,1.000,0.000}}
\gppoint{gp mark 7}{(4.125,1.623)}
\gpcolor{rgb color={0.500,1.000,0.000}}
\gppoint{gp mark 7}{(4.253,2.019)}
\gpcolor{rgb color={0.425,1.000,0.000}}
\gppoint{gp mark 7}{(4.381,1.719)}
\gpcolor{rgb color={0.350,1.000,0.000}}
\gppoint{gp mark 7}{(4.509,1.701)}
\gpcolor{rgb color={0.275,1.000,0.000}}
\gppoint{gp mark 7}{(4.637,1.894)}
\gpcolor{rgb color={0.200,1.000,0.000}}
\gppoint{gp mark 7}{(4.766,2.245)}
\gpcolor{rgb color={0.125,1.000,0.000}}
\gppoint{gp mark 7}{(4.894,1.970)}
\gpcolor{rgb color={0.050,1.000,0.000}}
\gppoint{gp mark 7}{(5.022,1.758)}
\gpcolor{rgb color={0.000,1.000,0.025}}
\gppoint{gp mark 7}{(5.150,3.245)}
\gpcolor{rgb color={0.000,1.000,0.100}}
\gppoint{gp mark 7}{(5.279,2.668)}
\gpcolor{rgb color={0.000,1.000,0.175}}
\gppoint{gp mark 7}{(5.407,3.811)}
\gpcolor{rgb color={0.000,1.000,0.250}}
\gppoint{gp mark 7}{(5.535,2.588)}
\gpcolor{rgb color={0.000,1.000,0.325}}
\gppoint{gp mark 7}{(5.663,4.402)}
\gpcolor{rgb color={0.000,1.000,0.400}}
\gppoint{gp mark 7}{(5.792,4.097)}
\gpcolor{rgb color={0.000,1.000,0.475}}
\gppoint{gp mark 7}{(5.920,3.061)}
\gpcolor{rgb color={0.000,1.000,0.550}}
\gppoint{gp mark 7}{(6.048,2.683)}
\gpcolor{rgb color={0.000,1.000,0.625}}
\gppoint{gp mark 7}{(6.176,1.141)}
\gpcolor{rgb color={0.000,1.000,0.700}}
\gppoint{gp mark 7}{(6.305,1.340)}
\gpcolor{rgb color={0.000,1.000,0.775}}
\gppoint{gp mark 7}{(6.433,1.805)}
\gpcolor{rgb color={0.000,1.000,0.850}}
\gppoint{gp mark 7}{(6.561,2.264)}
\gpcolor{rgb color={0.000,1.000,0.925}}
\gppoint{gp mark 7}{(6.689,2.727)}
\gpcolor{rgb color={0.000,1.000,1.000}}
\gppoint{gp mark 7}{(6.818,3.186)}
\gpcolor{rgb color={0.000,0.925,1.000}}
\gppoint{gp mark 7}{(6.946,3.644)}
\gpcolor{rgb color={0.000,0.850,1.000}}
\gppoint{gp mark 7}{(7.074,4.101)}
\gpcolor{rgb color={0.000,0.775,1.000}}
\gppoint{gp mark 7}{(7.202,4.584)}
\gpcolor{rgb color={0.000,0.700,1.000}}
\gppoint{gp mark 7}{(7.330,5.076)}
\gpcolor{rgb color={0.000,0.625,1.000}}
\gppoint{gp mark 7}{(7.459,5.076)}
\gpcolor{rgb color={0.000,0.550,1.000}}
\gppoint{gp mark 7}{(7.587,5.076)}
\gpcolor{rgb color={0.000,0.475,1.000}}
\gppoint{gp mark 7}{(7.715,5.076)}
\gpcolor{rgb color={0.000,0.400,1.000}}
\gppoint{gp mark 7}{(7.843,7.617)}
\gpcolor{rgb color={0.000,0.325,1.000}}
\gppoint{gp mark 7}{(7.972,7.617)}
\gpcolor{rgb color={0.000,0.250,1.000}}
\gppoint{gp mark 7}{(8.100,7.617)}
\gpcolor{rgb color={0.000,0.175,1.000}}
\gppoint{gp mark 7}{(8.228,7.891)}
\gpcolor{rgb color={0.000,0.100,1.000}}
\gppoint{gp mark 7}{(8.356,2.339)}
\gpcolor{rgb color={0.000,0.025,1.000}}
\gppoint{gp mark 7}{(8.485,2.339)}
\gpcolor{rgb color={0.050,0.000,1.000}}
\gppoint{gp mark 7}{(8.613,3.638)}
\gpcolor{rgb color={0.125,0.000,1.000}}
\gppoint{gp mark 7}{(8.741,3.638)}
\gpcolor{rgb color={0.200,0.000,1.000}}
\gppoint{gp mark 7}{(8.869,3.638)}
\gpcolor{rgb color={0.275,0.000,1.000}}
\gppoint{gp mark 7}{(8.998,2.200)}
\gpcolor{rgb color={0.350,0.000,1.000}}
\gppoint{gp mark 7}{(9.126,1.854)}
\gpcolor{rgb color={0.425,0.000,1.000}}
\gppoint{gp mark 7}{(9.254,1.714)}
\gpcolor{rgb color={0.500,0.000,1.000}}
\gppoint{gp mark 7}{(9.382,1.580)}
\gpcolor{rgb color={0.575,0.000,1.000}}
\gppoint{gp mark 7}{(9.510,1.058)}
\gpcolor{rgb color={0.650,0.000,1.000}}
\gppoint{gp mark 7}{(9.639,1.236)}
\gpcolor{rgb color={0.725,0.000,1.000}}
\gppoint{gp mark 7}{(9.767,1.431)}
\gpcolor{rgb color={0.800,0.000,1.000}}
\gppoint{gp mark 7}{(9.895,1.616)}
\gpcolor{rgb color={0.875,0.000,1.000}}
\gppoint{gp mark 7}{(10.023,5.336)}
\gpcolor{rgb color={0.950,0.000,1.000}}
\gppoint{gp mark 7}{(10.152,5.159)}
\gpcolor{rgb color={1.000,0.000,0.975}}
\gppoint{gp mark 7}{(10.280,4.982)}
\gpcolor{rgb color={1.000,0.000,0.900}}
\gppoint{gp mark 7}{(10.408,4.805)}
\gpcolor{rgb color={1.000,0.000,0.825}}
\gppoint{gp mark 7}{(10.536,1.137)}
\gpcolor{rgb color={1.000,0.000,0.750}}
\gppoint{gp mark 7}{(10.665,1.545)}
\gpcolor{rgb color={1.000,0.000,0.675}}
\gppoint{gp mark 7}{(10.793,2.000)}
\gpcolor{rgb color={1.000,0.000,0.600}}
\gppoint{gp mark 7}{(10.921,2.459)}
\gpcolor{rgb color={1.000,0.000,0.525}}
\gppoint{gp mark 7}{(11.049,3.818)}
\gpcolor{rgb color={1.000,0.000,0.450}}
\gppoint{gp mark 7}{(11.178,3.493)}
\gpcolor{rgb color={1.000,0.000,0.375}}
\gppoint{gp mark 7}{(11.306,3.204)}
\gpcolor{rgb color={1.000,0.000,0.300}}
\gppoint{gp mark 7}{(11.434,2.966)}
\gpcolor{rgb color={1.000,0.000,0.225}}
\gppoint{gp mark 7}{(11.562,1.262)}
\gpcolor{rgb color={1.000,0.000,0.150}}
\gppoint{gp mark 7}{(11.691,1.936)}
\gpcolor{rgb color={1.000,0.000,0.075}}
\gppoint{gp mark 7}{(11.819,1.936)}
\gpcolor{rgb color={1.000,0.000,0.000}}
\gppoint{gp mark 7}{(11.947,2.217)}
\gpcolor{color=gp lt color border}
\gpsetlinetype{gp lt border}
\draw[gp path] (1.688,8.381)--(1.688,0.985)--(11.947,0.985)--(11.947,8.381)--cycle;
%% coordinates of the plot area
\gpdefrectangularnode{gp plot 1}{\pgfpoint{1.688cm}{0.985cm}}{\pgfpoint{11.947cm}{8.381cm}}
\end{tikzpicture}
%% gnuplot variables

        }
        \caption{Positional difference between detected ball position and Kalman estimate.}
        \label{fig:ball_kalman_error}
    \end{subfigure}
    \caption{Detected ball positions and associated Kalman filter estimates.}
    \label{fig:kalman_test}
\end{figure}

Figure~\ref{fig:kalman_test} shows a 16 second extract of positional data gathered by the ball detection algorithm and Kalman filter.
Figure~\ref{fig:ball_kalman_3d} depicts the 3D positions while \ref{fig:ball_kalman_error} shows the difference between the last detected ball position and Kalman filter estimate. Associated measurements (at the same time) are the same color.

At 71 seconds the ball is no longer detected for approximately 3 seconds, so the ``detected'' position remains constant. However, the Kalman filter continues predicting the ball position based on previous samples. This may be be seen in the error graph; the difference between detected ball position and the estimate grows quickly until around 74 seconds, where the ball is uncovered and detected correctly in the camera images.

In general, whenever the ball changes direction, velocity or acceleration the error will increase because the Kalman filter is yet uncorrected. If the tracked object is not detected, the filter update step is skipped entirely.


\section{Paths planning around obstacles}
Simulation only



% \begin{figure}[htb]
%     \centering
%     \resizebox{\columnwidth}{!}{%
%         \begin{tikzpicture}[gnuplot]
%% generated with GNUPLOT 4.6p4 (Lua 5.1; terminal rev. 99, script rev. 100)
%% Thu 28 May 2015 09:06:58 AM CEST
\path (0.000,0.000) rectangle (12.500,8.750);
\gpcolor{color=gp lt color border}
\gpsetlinetype{gp lt border}
\gpsetlinewidth{1.00}
\draw[gp path] (1.800,2.277)--(5.058,4.137);
\draw[gp path] (10.700,3.063)--(5.058,4.137);
\draw[gp path] (1.800,2.277)--(1.800,5.995);
\draw[gp path] (1.800,2.277)--(1.937,2.355);
\node[gp node center,font={\fontsize{9pt}{10.8pt}\selectfont}] at (1.660,2.144) {-0.35};
\draw[gp path] (5.058,4.137)--(4.921,4.059);
\draw[gp path] (2.427,2.158)--(2.564,2.236);
\node[gp node center,font={\fontsize{9pt}{10.8pt}\selectfont}] at (2.287,2.025) {-0.3};
\draw[gp path] (5.685,4.017)--(5.548,3.939);
\draw[gp path] (3.054,2.039)--(3.191,2.117);
\node[gp node center,font={\fontsize{9pt}{10.8pt}\selectfont}] at (2.914,1.905) {-0.25};
\draw[gp path] (6.311,3.898)--(6.175,3.820);
\draw[gp path] (3.681,1.919)--(3.818,1.997);
\node[gp node center,font={\fontsize{9pt}{10.8pt}\selectfont}] at (3.541,1.786) {-0.2};
\draw[gp path] (6.938,3.779)--(6.801,3.701);
\draw[gp path] (4.308,1.800)--(4.445,1.878);
\node[gp node center,font={\fontsize{9pt}{10.8pt}\selectfont}] at (4.168,1.667) {-0.15};
\draw[gp path] (7.565,3.659)--(7.428,3.582);
\draw[gp path] (4.935,1.681)--(5.072,1.759);
\node[gp node center,font={\fontsize{9pt}{10.8pt}\selectfont}] at (4.795,1.547) {-0.1};
\draw[gp path] (8.192,3.540)--(8.055,3.462);
\draw[gp path] (5.562,1.562)--(5.699,1.639);
\node[gp node center,font={\fontsize{9pt}{10.8pt}\selectfont}] at (5.422,1.428) {-0.05};
\draw[gp path] (8.819,3.421)--(8.682,3.343);
\draw[gp path] (6.189,1.442)--(6.325,1.520);
\node[gp node center,font={\fontsize{9pt}{10.8pt}\selectfont}] at (6.049,1.309) { 0};
\draw[gp path] (9.446,3.302)--(9.309,3.224);
\draw[gp path] (6.815,1.323)--(6.952,1.401);
\node[gp node center,font={\fontsize{9pt}{10.8pt}\selectfont}] at (6.675,1.190) { 0.05};
\draw[gp path] (10.073,3.182)--(9.936,3.104);
\draw[gp path] (7.442,1.204)--(7.579,1.282);
\node[gp node center,font={\fontsize{9pt}{10.8pt}\selectfont}] at (7.302,1.070) { 0.1};
\draw[gp path] (10.700,3.063)--(10.563,2.985);
\draw[gp path] (7.442,1.204)--(7.286,1.233);
\node[gp node center,font={\fontsize{9pt}{10.8pt}\selectfont}] at (7.601,1.153) {-0.15};
\draw[gp path] (1.800,2.277)--(1.956,2.248);
\draw[gp path] (7.907,1.469)--(7.752,1.499);
\node[gp node center,font={\fontsize{9pt}{10.8pt}\selectfont}] at (8.067,1.419) {-0.1};
\draw[gp path] (2.265,2.543)--(2.421,2.513);
\draw[gp path] (8.373,1.735)--(8.217,1.765);
\node[gp node center,font={\fontsize{9pt}{10.8pt}\selectfont}] at (8.532,1.684) {-0.05};
\draw[gp path] (2.731,2.808)--(2.887,2.779);
\draw[gp path] (8.838,2.001)--(8.682,2.030);
\node[gp node center,font={\fontsize{9pt}{10.8pt}\selectfont}] at (8.998,1.950) { 0};
\draw[gp path] (3.196,3.074)--(3.352,3.044);
\draw[gp path] (9.304,2.266)--(9.148,2.296);
\node[gp node center,font={\fontsize{9pt}{10.8pt}\selectfont}] at (9.463,2.215) { 0.05};
\draw[gp path] (3.662,3.340)--(3.818,3.310);
\draw[gp path] (9.769,2.532)--(9.613,2.561);
\node[gp node center,font={\fontsize{9pt}{10.8pt}\selectfont}] at (9.929,2.481) { 0.1};
\draw[gp path] (4.127,3.605)--(4.283,3.576);
\draw[gp path] (10.235,2.797)--(10.079,2.827);
\node[gp node center,font={\fontsize{9pt}{10.8pt}\selectfont}] at (10.394,2.747) { 0.15};
\draw[gp path] (4.593,3.871)--(4.748,3.841);
\draw[gp path] (10.700,3.063)--(10.544,3.093);
\node[gp node center,font={\fontsize{9pt}{10.8pt}\selectfont}] at (10.859,3.012) { 0.2};
\draw[gp path] (5.058,4.137)--(5.214,4.107);
\draw[gp path] (1.800,3.517)--(1.980,3.517);
\node[gp node right,font={\fontsize{9pt}{10.8pt}\selectfont}] at (1.440,3.517) { 0.5};
\draw[gp path] (1.800,3.792)--(1.980,3.792);
\node[gp node right,font={\fontsize{9pt}{10.8pt}\selectfont}] at (1.440,3.792) { 0.6};
\draw[gp path] (1.800,4.068)--(1.980,4.068);
\node[gp node right,font={\fontsize{9pt}{10.8pt}\selectfont}] at (1.440,4.068) { 0.7};
\draw[gp path] (1.800,4.343)--(1.980,4.343);
\node[gp node right,font={\fontsize{9pt}{10.8pt}\selectfont}] at (1.440,4.343) { 0.8};
\draw[gp path] (1.800,4.618)--(1.980,4.618);
\node[gp node right,font={\fontsize{9pt}{10.8pt}\selectfont}] at (1.440,4.618) { 0.9};
\draw[gp path] (1.800,4.893)--(1.980,4.893);
\node[gp node right,font={\fontsize{9pt}{10.8pt}\selectfont}] at (1.440,4.893) { 1};
\draw[gp path] (1.800,5.169)--(1.980,5.169);
\node[gp node right,font={\fontsize{9pt}{10.8pt}\selectfont}] at (1.440,5.169) { 1.1};
\draw[gp path] (1.800,5.444)--(1.980,5.444);
\node[gp node right,font={\fontsize{9pt}{10.8pt}\selectfont}] at (1.440,5.444) { 1.2};
\draw[gp path] (1.800,5.719)--(1.980,5.719);
\node[gp node right,font={\fontsize{9pt}{10.8pt}\selectfont}] at (1.440,5.719) { 1.3};
\draw[gp path] (1.800,5.995)--(1.980,5.995);
\node[gp node right,font={\fontsize{9pt}{10.8pt}\selectfont}] at (1.440,5.995) { 1.4};
\node[gp node center,font={\fontsize{9pt}{10.8pt}\selectfont}] at (0.512,4.755) {z};
\node[gp node right,font={\fontsize{9pt}{10.8pt}\selectfont}] at (9.416,7.645) {detected ball position};
\gpcolor{rgb color={1.000,0.822,0.000}}
\gpsetlinetype{gp lt plot 0}
\draw[gp path] (9.600,7.645)--(9.638,7.645);
\gpcolor{rgb color={0.956,1.000,0.000}}
\draw[gp path] (9.638,7.645)--(9.676,7.645);
\gpcolor{rgb color={0.733,1.000,0.000}}
\draw[gp path] (9.676,7.645)--(9.715,7.645);
\gpcolor{rgb color={0.511,1.000,0.000}}
\draw[gp path] (9.715,7.645)--(9.753,7.645);
\gpcolor{rgb color={0.289,1.000,0.000}}
\draw[gp path] (9.753,7.645)--(9.791,7.645);
\gpcolor{rgb color={0.067,1.000,0.000}}
\draw[gp path] (9.791,7.645)--(9.829,7.645);
\gpcolor{rgb color={0.000,1.000,0.156}}
\draw[gp path] (9.829,7.645)--(9.867,7.645);
\gpcolor{rgb color={0.000,1.000,0.378}}
\draw[gp path] (9.867,7.645)--(9.905,7.645);
\gpcolor{rgb color={0.000,1.000,0.600}}
\draw[gp path] (9.905,7.645)--(9.944,7.645);
\gpcolor{rgb color={0.000,1.000,0.822}}
\draw[gp path] (9.944,7.645)--(9.982,7.645);
\gpcolor{rgb color={0.000,0.956,1.000}}
\draw[gp path] (9.982,7.645)--(10.020,7.645);
\gpcolor{rgb color={0.000,0.733,1.000}}
\draw[gp path] (10.020,7.645)--(10.058,7.645);
\gpcolor{rgb color={0.000,0.511,1.000}}
\draw[gp path] (10.058,7.645)--(10.096,7.645);
\gpcolor{rgb color={0.000,0.289,1.000}}
\draw[gp path] (10.096,7.645)--(10.134,7.645);
\gpcolor{rgb color={0.000,0.067,1.000}}
\draw[gp path] (10.134,7.645)--(10.173,7.645);
\gpcolor{rgb color={0.156,0.000,1.000}}
\draw[gp path] (10.173,7.645)--(10.211,7.645);
\gpcolor{rgb color={0.378,0.000,1.000}}
\draw[gp path] (10.211,7.645)--(10.249,7.645);
\gpcolor{rgb color={0.600,0.000,1.000}}
\draw[gp path] (10.249,7.645)--(10.287,7.645);
\gpcolor{rgb color={0.822,0.000,1.000}}
\draw[gp path] (10.287,7.645)--(10.325,7.645);
\gpcolor{rgb color={1.000,0.000,0.956}}
\draw[gp path] (10.325,7.645)--(10.363,7.645);
\gpcolor{rgb color={1.000,0.000,0.733}}
\draw[gp path] (10.363,7.645)--(10.402,7.645);
\gpcolor{rgb color={1.000,0.000,0.511}}
\draw[gp path] (10.402,7.645)--(10.440,7.645);
\gpcolor{rgb color={1.000,0.000,0.289}}
\draw[gp path] (10.440,7.645)--(10.478,7.645);
\gpcolor{rgb color={1.000,0.000,0.067}}
\draw[gp path] (10.478,7.645)--(10.516,7.645);
\gpcolor{rgb color={1.000,0.767,0.000}}
\draw[gp path] (9.276,3.968)--(9.097,4.165);
\gpcolor{rgb color={1.000,0.900,0.000}}
\draw[gp path] (9.097,4.165)--(8.791,4.658);
\gpcolor{rgb color={0.833,1.000,0.000}}
\draw[gp path] (8.791,4.658)--(8.697,4.734);
\gpcolor{rgb color={0.633,1.000,0.000}}
\draw[gp path] (8.697,4.734)--(8.786,4.931);
\gpcolor{rgb color={0.300,1.000,0.000}}
\draw[gp path] (8.786,4.931)--(8.890,5.363);
\gpcolor{rgb color={0.000,1.000,0.033}}
\draw[gp path] (8.890,5.363)--(8.638,5.646);
\gpcolor{rgb color={0.000,1.000,0.233}}
\draw[gp path] (8.638,5.646)--(8.199,5.893);
\gpcolor{rgb color={0.000,1.000,0.433}}
\draw[gp path] (8.199,5.893)--(7.674,6.233);
\gpcolor{rgb color={0.000,1.000,0.567}}
\draw[gp path] (7.674,6.233)--(6.997,6.384);
\gpcolor{rgb color={0.000,1.000,0.700}}
\draw[gp path] (6.997,6.384)--(6.164,6.612);
\gpcolor{rgb color={0.000,0.167,1.000}}
\draw[gp path] (6.164,6.612)--(3.995,5.467);
\gpcolor{rgb color={0.100,0.000,1.000}}
\draw[gp path] (3.995,5.467)--(4.309,5.284);
\gpcolor{rgb color={0.233,0.000,1.000}}
\draw[gp path] (4.309,5.284)--(4.537,5.115);
\gpcolor{rgb color={0.500,0.000,1.000}}
\draw[gp path] (4.537,5.115)--(4.734,5.075);
\gpcolor{rgb color={0.967,0.000,1.000}}
\draw[gp path] (4.734,5.075)--(5.496,4.315);
\gpcolor{rgb color={1.000,0.000,0.500}}
\draw[gp path] (5.496,4.315)--(6.430,3.492);
\gpcolor{rgb color={1.000,0.000,0.167}}
\draw[gp path] (6.430,3.492)--(7.142,3.356);
\gpcolor{rgb color={1.000,0.600,0.000}}
\gpsetpointsize{4.00}
\gppoint{gp mark 4}{(9.600,7.645)}
\gpcolor{rgb color={0.000,0.733,1.000}}
\gppoint{gp mark 4}{(10.058,7.645)}
\gpcolor{rgb color={1.000,0.000,0.067}}
\gppoint{gp mark 4}{(10.516,7.645)}
\gpcolor{rgb color={1.000,0.600,0.000}}
\gppoint{gp mark 4}{(9.276,3.968)}
\gpcolor{rgb color={1.000,0.667,0.000}}
\gppoint{gp mark 4}{(9.276,3.968)}
\gpcolor{rgb color={1.000,0.733,0.000}}
\gppoint{gp mark 4}{(9.276,3.968)}
\gpcolor{rgb color={1.000,0.800,0.000}}
\gppoint{gp mark 4}{(9.097,4.165)}
\gpcolor{rgb color={1.000,0.867,0.000}}
\gppoint{gp mark 4}{(9.097,4.165)}
\gpcolor{rgb color={1.000,0.933,0.000}}
\gppoint{gp mark 4}{(8.791,4.658)}
\gpcolor{rgb color={1.000,1.000,0.000}}
\gppoint{gp mark 4}{(8.791,4.658)}
\gpcolor{rgb color={0.933,1.000,0.000}}
\gppoint{gp mark 4}{(8.791,4.658)}
\gpcolor{rgb color={0.867,1.000,0.000}}
\gppoint{gp mark 4}{(8.791,4.658)}
\gpcolor{rgb color={0.800,1.000,0.000}}
\gppoint{gp mark 4}{(8.697,4.734)}
\gpcolor{rgb color={0.733,1.000,0.000}}
\gppoint{gp mark 4}{(8.697,4.734)}
\gpcolor{rgb color={0.667,1.000,0.000}}
\gppoint{gp mark 4}{(8.697,4.734)}
\gpcolor{rgb color={0.600,1.000,0.000}}
\gppoint{gp mark 4}{(8.786,4.931)}
\gpcolor{rgb color={0.533,1.000,0.000}}
\gppoint{gp mark 4}{(8.786,4.931)}
\gpcolor{rgb color={0.467,1.000,0.000}}
\gppoint{gp mark 4}{(8.786,4.931)}
\gpcolor{rgb color={0.400,1.000,0.000}}
\gppoint{gp mark 4}{(8.786,4.931)}
\gpcolor{rgb color={0.333,1.000,0.000}}
\gppoint{gp mark 4}{(8.786,4.931)}
\gpcolor{rgb color={0.267,1.000,0.000}}
\gppoint{gp mark 4}{(8.890,5.363)}
\gpcolor{rgb color={0.200,1.000,0.000}}
\gppoint{gp mark 4}{(8.890,5.363)}
\gpcolor{rgb color={0.133,1.000,0.000}}
\gppoint{gp mark 4}{(8.890,5.363)}
\gpcolor{rgb color={0.067,1.000,0.000}}
\gppoint{gp mark 4}{(8.890,5.363)}
\gpcolor{rgb color={0.000,1.000,0.000}}
\gppoint{gp mark 4}{(8.890,5.363)}
\gpcolor{rgb color={0.000,1.000,0.067}}
\gppoint{gp mark 4}{(8.638,5.646)}
\gpcolor{rgb color={0.000,1.000,0.133}}
\gppoint{gp mark 4}{(8.638,5.646)}
\gpcolor{rgb color={0.000,1.000,0.200}}
\gppoint{gp mark 4}{(8.638,5.646)}
\gpcolor{rgb color={0.000,1.000,0.267}}
\gppoint{gp mark 4}{(8.199,5.893)}
\gpcolor{rgb color={0.000,1.000,0.333}}
\gppoint{gp mark 4}{(8.199,5.893)}
\gpcolor{rgb color={0.000,1.000,0.400}}
\gppoint{gp mark 4}{(8.199,5.893)}
\gpcolor{rgb color={0.000,1.000,0.467}}
\gppoint{gp mark 4}{(7.674,6.233)}
\gpcolor{rgb color={0.000,1.000,0.533}}
\gppoint{gp mark 4}{(7.674,6.233)}
\gpcolor{rgb color={0.000,1.000,0.600}}
\gppoint{gp mark 4}{(6.997,6.384)}
\gpcolor{rgb color={0.000,1.000,0.667}}
\gppoint{gp mark 4}{(6.997,6.384)}
\gpcolor{rgb color={0.000,1.000,0.733}}
\gppoint{gp mark 4}{(6.164,6.612)}
\gpcolor{rgb color={0.000,1.000,0.800}}
\gppoint{gp mark 4}{(6.164,6.612)}
\gpcolor{rgb color={0.000,1.000,0.867}}
\gppoint{gp mark 4}{(6.164,6.612)}
\gpcolor{rgb color={0.000,1.000,0.933}}
\gppoint{gp mark 4}{(6.164,6.612)}
\gpcolor{rgb color={0.000,1.000,1.000}}
\gppoint{gp mark 4}{(6.164,6.612)}
\gpcolor{rgb color={0.000,0.933,1.000}}
\gppoint{gp mark 4}{(6.164,6.612)}
\gpcolor{rgb color={0.000,0.867,1.000}}
\gppoint{gp mark 4}{(6.164,6.612)}
\gpcolor{rgb color={0.000,0.800,1.000}}
\gppoint{gp mark 4}{(6.164,6.612)}
\gpcolor{rgb color={0.000,0.733,1.000}}
\gppoint{gp mark 4}{(6.164,6.612)}
\gpcolor{rgb color={0.000,0.667,1.000}}
\gppoint{gp mark 4}{(6.164,6.612)}
\gpcolor{rgb color={0.000,0.600,1.000}}
\gppoint{gp mark 4}{(6.164,6.612)}
\gpcolor{rgb color={0.000,0.533,1.000}}
\gppoint{gp mark 4}{(6.164,6.612)}
\gpcolor{rgb color={0.000,0.467,1.000}}
\gppoint{gp mark 4}{(6.164,6.612)}
\gpcolor{rgb color={0.000,0.400,1.000}}
\gppoint{gp mark 4}{(6.164,6.612)}
\gpcolor{rgb color={0.000,0.333,1.000}}
\gppoint{gp mark 4}{(6.164,6.612)}
\gpcolor{rgb color={0.000,0.267,1.000}}
\gppoint{gp mark 4}{(6.164,6.612)}
\gpcolor{rgb color={0.000,0.200,1.000}}
\gppoint{gp mark 4}{(6.164,6.612)}
\gpcolor{rgb color={0.000,0.133,1.000}}
\gppoint{gp mark 4}{(3.995,5.467)}
\gpcolor{rgb color={0.000,0.067,1.000}}
\gppoint{gp mark 4}{(3.995,5.467)}
\gpcolor{rgb color={0.000,0.000,1.000}}
\gppoint{gp mark 4}{(3.995,5.467)}
\gpcolor{rgb color={0.067,0.000,1.000}}
\gppoint{gp mark 4}{(3.995,5.467)}
\gpcolor{rgb color={0.133,0.000,1.000}}
\gppoint{gp mark 4}{(4.309,5.284)}
\gpcolor{rgb color={0.200,0.000,1.000}}
\gppoint{gp mark 4}{(4.309,5.284)}
\gpcolor{rgb color={0.267,0.000,1.000}}
\gppoint{gp mark 4}{(4.537,5.115)}
\gpcolor{rgb color={0.333,0.000,1.000}}
\gppoint{gp mark 4}{(4.537,5.115)}
\gpcolor{rgb color={0.400,0.000,1.000}}
\gppoint{gp mark 4}{(4.537,5.115)}
\gpcolor{rgb color={0.467,0.000,1.000}}
\gppoint{gp mark 4}{(4.537,5.115)}
\gpcolor{rgb color={0.533,0.000,1.000}}
\gppoint{gp mark 4}{(4.734,5.075)}
\gpcolor{rgb color={0.600,0.000,1.000}}
\gppoint{gp mark 4}{(4.734,5.075)}
\gpcolor{rgb color={0.667,0.000,1.000}}
\gppoint{gp mark 4}{(4.734,5.075)}
\gpcolor{rgb color={0.733,0.000,1.000}}
\gppoint{gp mark 4}{(4.734,5.075)}
\gpcolor{rgb color={0.800,0.000,1.000}}
\gppoint{gp mark 4}{(4.734,5.075)}
\gpcolor{rgb color={0.867,0.000,1.000}}
\gppoint{gp mark 4}{(4.734,5.075)}
\gpcolor{rgb color={0.933,0.000,1.000}}
\gppoint{gp mark 4}{(4.734,5.075)}
\gpcolor{rgb color={1.000,0.000,1.000}}
\gppoint{gp mark 4}{(5.496,4.315)}
\gpcolor{rgb color={1.000,0.000,0.933}}
\gppoint{gp mark 4}{(5.496,4.315)}
\gpcolor{rgb color={1.000,0.000,0.867}}
\gppoint{gp mark 4}{(5.496,4.315)}
\gpcolor{rgb color={1.000,0.000,0.800}}
\gppoint{gp mark 4}{(5.496,4.315)}
\gpcolor{rgb color={1.000,0.000,0.733}}
\gppoint{gp mark 4}{(5.496,4.315)}
\gpcolor{rgb color={1.000,0.000,0.667}}
\gppoint{gp mark 4}{(5.496,4.315)}
\gpcolor{rgb color={1.000,0.000,0.600}}
\gppoint{gp mark 4}{(5.496,4.315)}
\gpcolor{rgb color={1.000,0.000,0.533}}
\gppoint{gp mark 4}{(5.496,4.315)}
\gpcolor{rgb color={1.000,0.000,0.467}}
\gppoint{gp mark 4}{(6.430,3.492)}
\gpcolor{rgb color={1.000,0.000,0.400}}
\gppoint{gp mark 4}{(6.430,3.492)}
\gpcolor{rgb color={1.000,0.000,0.333}}
\gppoint{gp mark 4}{(6.430,3.492)}
\gpcolor{rgb color={1.000,0.000,0.267}}
\gppoint{gp mark 4}{(6.430,3.492)}
\gpcolor{rgb color={1.000,0.000,0.200}}
\gppoint{gp mark 4}{(6.430,3.492)}
\gpcolor{rgb color={1.000,0.000,0.133}}
\gppoint{gp mark 4}{(7.142,3.356)}
\gpcolor{rgb color={1.000,0.000,0.067}}
\gppoint{gp mark 4}{(7.142,3.356)}
\gpcolor{color=gp lt color border}
\node[gp node right,font={\fontsize{9pt}{10.8pt}\selectfont}] at (9.416,7.337) {ball pos. kalman estimate};
\gpcolor{rgb color={1.000,0.822,0.000}}
\gpsetlinetype{gp lt plot 1}
\draw[gp path] (9.600,7.337)--(9.638,7.337);
\gpcolor{rgb color={0.956,1.000,0.000}}
\draw[gp path] (9.638,7.337)--(9.676,7.337);
\gpcolor{rgb color={0.733,1.000,0.000}}
\draw[gp path] (9.676,7.337)--(9.715,7.337);
\gpcolor{rgb color={0.511,1.000,0.000}}
\draw[gp path] (9.715,7.337)--(9.753,7.337);
\gpcolor{rgb color={0.289,1.000,0.000}}
\draw[gp path] (9.753,7.337)--(9.791,7.337);
\gpcolor{rgb color={0.067,1.000,0.000}}
\draw[gp path] (9.791,7.337)--(9.829,7.337);
\gpcolor{rgb color={0.000,1.000,0.156}}
\draw[gp path] (9.829,7.337)--(9.867,7.337);
\gpcolor{rgb color={0.000,1.000,0.378}}
\draw[gp path] (9.867,7.337)--(9.905,7.337);
\gpcolor{rgb color={0.000,1.000,0.600}}
\draw[gp path] (9.905,7.337)--(9.944,7.337);
\gpcolor{rgb color={0.000,1.000,0.822}}
\draw[gp path] (9.944,7.337)--(9.982,7.337);
\gpcolor{rgb color={0.000,0.956,1.000}}
\draw[gp path] (9.982,7.337)--(10.020,7.337);
\gpcolor{rgb color={0.000,0.733,1.000}}
\draw[gp path] (10.020,7.337)--(10.058,7.337);
\gpcolor{rgb color={0.000,0.511,1.000}}
\draw[gp path] (10.058,7.337)--(10.096,7.337);
\gpcolor{rgb color={0.000,0.289,1.000}}
\draw[gp path] (10.096,7.337)--(10.134,7.337);
\gpcolor{rgb color={0.000,0.067,1.000}}
\draw[gp path] (10.134,7.337)--(10.173,7.337);
\gpcolor{rgb color={0.156,0.000,1.000}}
\draw[gp path] (10.173,7.337)--(10.211,7.337);
\gpcolor{rgb color={0.378,0.000,1.000}}
\draw[gp path] (10.211,7.337)--(10.249,7.337);
\gpcolor{rgb color={0.600,0.000,1.000}}
\draw[gp path] (10.249,7.337)--(10.287,7.337);
\gpcolor{rgb color={0.822,0.000,1.000}}
\draw[gp path] (10.287,7.337)--(10.325,7.337);
\gpcolor{rgb color={1.000,0.000,0.956}}
\draw[gp path] (10.325,7.337)--(10.363,7.337);
\gpcolor{rgb color={1.000,0.000,0.733}}
\draw[gp path] (10.363,7.337)--(10.402,7.337);
\gpcolor{rgb color={1.000,0.000,0.511}}
\draw[gp path] (10.402,7.337)--(10.440,7.337);
\gpcolor{rgb color={1.000,0.000,0.289}}
\draw[gp path] (10.440,7.337)--(10.478,7.337);
\gpcolor{rgb color={1.000,0.000,0.067}}
\draw[gp path] (10.478,7.337)--(10.516,7.337);
\gpcolor{rgb color={1.000,0.633,0.000}}
\draw[gp path] (9.316,3.807)--(9.389,3.871);
\gpcolor{rgb color={1.000,0.700,0.000}}
\draw[gp path] (9.389,3.871)--(9.373,4.020);
\gpcolor{rgb color={1.000,0.767,0.000}}
\draw[gp path] (9.373,4.020)--(9.445,4.090);
\gpcolor{rgb color={1.000,0.833,0.000}}
\draw[gp path] (9.445,4.090)--(9.520,4.164);
\gpcolor{rgb color={1.000,0.900,0.000}}
\draw[gp path] (9.520,4.164)--(9.597,4.239);
\gpcolor{rgb color={1.000,0.967,0.000}}
\draw[gp path] (9.597,4.239)--(9.668,4.309);
\gpcolor{rgb color={0.967,1.000,0.000}}
\draw[gp path] (9.668,4.309)--(9.182,4.216);
\gpcolor{rgb color={0.900,1.000,0.000}}
\draw[gp path] (9.182,4.216)--(9.212,4.268);
\gpcolor{rgb color={0.833,1.000,0.000}}
\draw[gp path] (9.212,4.268)--(8.987,4.636);
\gpcolor{rgb color={0.767,1.000,0.000}}
\draw[gp path] (8.987,4.636)--(9.009,4.714);
\gpcolor{rgb color={0.700,1.000,0.000}}
\draw[gp path] (9.009,4.714)--(9.031,4.793);
\gpcolor{rgb color={0.633,1.000,0.000}}
\draw[gp path] (9.031,4.793)--(8.735,4.802);
\gpcolor{rgb color={0.500,1.000,0.000}}
\draw[gp path] (8.735,4.802)--(8.739,4.943);
\gpcolor{rgb color={0.433,1.000,0.000}}
\draw[gp path] (8.739,4.943)--(8.740,5.012);
\gpcolor{rgb color={0.300,1.000,0.000}}
\draw[gp path] (8.740,5.012)--(8.769,5.072);
\gpcolor{rgb color={0.233,1.000,0.000}}
\draw[gp path] (8.769,5.072)--(8.772,5.135);
\gpcolor{rgb color={0.167,1.000,0.000}}
\draw[gp path] (8.772,5.135)--(8.909,5.416);
\gpcolor{rgb color={0.100,1.000,0.000}}
\draw[gp path] (8.909,5.416)--(8.935,5.501);
\gpcolor{rgb color={0.033,1.000,0.000}}
\draw[gp path] (8.935,5.501)--(8.960,5.586);
\gpcolor{rgb color={0.000,1.000,0.033}}
\draw[gp path] (8.960,5.586)--(8.985,5.671);
\gpcolor{rgb color={0.000,1.000,0.100}}
\draw[gp path] (8.985,5.671)--(9.010,5.755);
\gpcolor{rgb color={0.000,1.000,0.167}}
\draw[gp path] (9.010,5.755)--(9.037,5.846);
\gpcolor{rgb color={0.000,1.000,0.233}}
\draw[gp path] (9.037,5.846)--(8.657,5.720);
\gpcolor{rgb color={0.000,1.000,0.300}}
\draw[gp path] (8.657,5.720)--(8.646,5.785);
\gpcolor{rgb color={0.000,1.000,0.367}}
\draw[gp path] (8.646,5.785)--(8.635,5.849);
\gpcolor{rgb color={0.000,1.000,0.433}}
\draw[gp path] (8.635,5.849)--(8.624,5.913);
\gpcolor{rgb color={0.000,1.000,0.500}}
\draw[gp path] (8.624,5.913)--(8.218,5.940);
\gpcolor{rgb color={0.000,1.000,0.567}}
\draw[gp path] (8.218,5.940)--(8.181,5.998);
\gpcolor{rgb color={0.000,1.000,0.633}}
\draw[gp path] (8.181,5.998)--(7.748,6.234);
\gpcolor{rgb color={0.000,1.000,0.700}}
\draw[gp path] (7.748,6.234)--(7.697,6.304);
\gpcolor{rgb color={0.000,1.000,0.767}}
\draw[gp path] (7.697,6.304)--(7.646,6.374);
\gpcolor{rgb color={0.000,1.000,0.833}}
\draw[gp path] (7.646,6.374)--(7.026,6.425);
\gpcolor{rgb color={0.000,1.000,0.900}}
\draw[gp path] (7.026,6.425)--(6.928,6.491);
\gpcolor{rgb color={0.000,1.000,0.967}}
\draw[gp path] (6.928,6.491)--(6.244,6.618);
\gpcolor{rgb color={0.000,0.967,1.000}}
\draw[gp path] (6.244,6.618)--(6.110,6.679);
\gpcolor{rgb color={0.000,0.900,1.000}}
\draw[gp path] (6.110,6.679)--(5.972,6.742);
\gpcolor{rgb color={0.000,0.833,1.000}}
\draw[gp path] (5.972,6.742)--(5.837,6.803);
\gpcolor{rgb color={0.000,0.767,1.000}}
\draw[gp path] (5.837,6.803)--(5.702,6.865);
\gpcolor{rgb color={0.000,0.700,1.000}}
\draw[gp path] (5.702,6.865)--(5.568,6.926);
\gpcolor{rgb color={0.000,0.633,1.000}}
\draw[gp path] (5.568,6.926)--(5.434,6.986);
\gpcolor{rgb color={0.000,0.567,1.000}}
\draw[gp path] (5.434,6.986)--(5.300,7.047);
\gpcolor{rgb color={0.000,0.500,1.000}}
\draw[gp path] (5.300,7.047)--(5.159,7.111);
\gpcolor{rgb color={0.000,0.433,1.000}}
\draw[gp path] (5.159,7.111)--(5.015,7.176);
\gpcolor{rgb color={0.033,0.000,1.000}}
\draw[gp path] (5.015,7.176)--(4.881,7.237);
\gpcolor{rgb color={0.100,0.000,1.000}}
\draw[gp path] (4.881,7.237)--(3.995,5.467);
\gpcolor{rgb color={0.433,0.000,1.000}}
\draw[gp path] (3.995,5.467)--(4.319,5.272);
\gpcolor{rgb color={0.500,0.000,1.000}}
\draw[gp path] (4.319,5.272)--(4.471,5.137);
\gpcolor{rgb color={0.567,0.000,1.000}}
\draw[gp path] (4.471,5.137)--(4.489,5.119);
\gpcolor{rgb color={0.633,0.000,1.000}}
\draw[gp path] (4.489,5.119)--(4.507,5.100);
\gpcolor{rgb color={0.700,0.000,1.000}}
\draw[gp path] (4.507,5.100)--(4.713,5.071);
\gpcolor{rgb color={0.767,0.000,1.000}}
\draw[gp path] (4.713,5.071)--(4.748,5.053);
\gpcolor{rgb color={0.833,0.000,1.000}}
\draw[gp path] (4.748,5.053)--(4.784,5.035);
\gpcolor{rgb color={0.900,0.000,1.000}}
\draw[gp path] (4.784,5.035)--(4.818,5.018);
\gpcolor{rgb color={0.967,0.000,1.000}}
\draw[gp path] (4.818,5.018)--(4.852,5.002);
\gpcolor{rgb color={1.000,0.000,0.967}}
\draw[gp path] (4.852,5.002)--(4.886,4.985);
\gpcolor{rgb color={1.000,0.000,0.900}}
\draw[gp path] (4.886,4.985)--(4.920,4.968);
\gpcolor{rgb color={1.000,0.000,0.833}}
\draw[gp path] (4.920,4.968)--(4.954,4.951);
\gpcolor{rgb color={1.000,0.000,0.767}}
\draw[gp path] (4.954,4.951)--(5.463,4.299);
\gpcolor{rgb color={1.000,0.000,0.700}}
\draw[gp path] (5.463,4.299)--(5.508,4.239);
\gpcolor{rgb color={1.000,0.000,0.633}}
\draw[gp path] (5.508,4.239)--(5.557,4.175);
\gpcolor{rgb color={1.000,0.000,0.567}}
\draw[gp path] (5.557,4.175)--(5.606,4.111);
\gpcolor{rgb color={1.000,0.000,0.500}}
\draw[gp path] (5.606,4.111)--(5.651,4.052);
\gpcolor{rgb color={1.000,0.000,0.433}}
\draw[gp path] (5.651,4.052)--(5.697,3.992);
\gpcolor{rgb color={1.000,0.000,0.367}}
\draw[gp path] (5.697,3.992)--(5.742,3.932);
\gpcolor{rgb color={1.000,0.000,0.300}}
\draw[gp path] (5.742,3.932)--(5.788,3.872);
\gpcolor{rgb color={1.000,0.000,0.233}}
\draw[gp path] (5.788,3.872)--(6.437,3.437);
\gpcolor{rgb color={1.000,0.000,0.167}}
\draw[gp path] (6.437,3.437)--(6.557,3.268);
\gpcolor{rgb color={1.000,0.600,0.000}}
\gppoint{gp mark 2}{(9.600,7.337)}
\gpcolor{rgb color={0.000,0.733,1.000}}
\gppoint{gp mark 2}{(10.058,7.337)}
\gpcolor{rgb color={1.000,0.000,0.067}}
\gppoint{gp mark 2}{(10.516,7.337)}
\gpcolor{rgb color={1.000,0.600,0.000}}
\gppoint{gp mark 2}{(9.316,3.807)}
\gpcolor{rgb color={1.000,0.667,0.000}}
\gppoint{gp mark 2}{(9.389,3.871)}
\gpcolor{rgb color={1.000,0.733,0.000}}
\gppoint{gp mark 2}{(9.373,4.020)}
\gpcolor{rgb color={1.000,0.800,0.000}}
\gppoint{gp mark 2}{(9.445,4.090)}
\gpcolor{rgb color={1.000,0.867,0.000}}
\gppoint{gp mark 2}{(9.520,4.164)}
\gpcolor{rgb color={1.000,0.933,0.000}}
\gppoint{gp mark 2}{(9.597,4.239)}
\gpcolor{rgb color={1.000,1.000,0.000}}
\gppoint{gp mark 2}{(9.668,4.309)}
\gpcolor{rgb color={0.933,1.000,0.000}}
\gppoint{gp mark 2}{(9.182,4.216)}
\gpcolor{rgb color={0.867,1.000,0.000}}
\gppoint{gp mark 2}{(9.212,4.268)}
\gpcolor{rgb color={0.800,1.000,0.000}}
\gppoint{gp mark 2}{(8.987,4.636)}
\gpcolor{rgb color={0.733,1.000,0.000}}
\gppoint{gp mark 2}{(9.009,4.714)}
\gpcolor{rgb color={0.667,1.000,0.000}}
\gppoint{gp mark 2}{(9.031,4.793)}
\gpcolor{rgb color={0.600,1.000,0.000}}
\gppoint{gp mark 2}{(8.735,4.802)}
\gpcolor{rgb color={0.533,1.000,0.000}}
\gppoint{gp mark 2}{(8.735,4.802)}
\gpcolor{rgb color={0.467,1.000,0.000}}
\gppoint{gp mark 2}{(8.739,4.943)}
\gpcolor{rgb color={0.400,1.000,0.000}}
\gppoint{gp mark 2}{(8.740,5.012)}
\gpcolor{rgb color={0.333,1.000,0.000}}
\gppoint{gp mark 2}{(8.740,5.012)}
\gpcolor{rgb color={0.267,1.000,0.000}}
\gppoint{gp mark 2}{(8.769,5.072)}
\gpcolor{rgb color={0.200,1.000,0.000}}
\gppoint{gp mark 2}{(8.772,5.135)}
\gpcolor{rgb color={0.133,1.000,0.000}}
\gppoint{gp mark 2}{(8.909,5.416)}
\gpcolor{rgb color={0.067,1.000,0.000}}
\gppoint{gp mark 2}{(8.935,5.501)}
\gpcolor{rgb color={0.000,1.000,0.000}}
\gppoint{gp mark 2}{(8.960,5.586)}
\gpcolor{rgb color={0.000,1.000,0.067}}
\gppoint{gp mark 2}{(8.985,5.671)}
\gpcolor{rgb color={0.000,1.000,0.133}}
\gppoint{gp mark 2}{(9.010,5.755)}
\gpcolor{rgb color={0.000,1.000,0.200}}
\gppoint{gp mark 2}{(9.037,5.846)}
\gpcolor{rgb color={0.000,1.000,0.267}}
\gppoint{gp mark 2}{(8.657,5.720)}
\gpcolor{rgb color={0.000,1.000,0.333}}
\gppoint{gp mark 2}{(8.646,5.785)}
\gpcolor{rgb color={0.000,1.000,0.400}}
\gppoint{gp mark 2}{(8.635,5.849)}
\gpcolor{rgb color={0.000,1.000,0.467}}
\gppoint{gp mark 2}{(8.624,5.913)}
\gpcolor{rgb color={0.000,1.000,0.533}}
\gppoint{gp mark 2}{(8.218,5.940)}
\gpcolor{rgb color={0.000,1.000,0.600}}
\gppoint{gp mark 2}{(8.181,5.998)}
\gpcolor{rgb color={0.000,1.000,0.667}}
\gppoint{gp mark 2}{(7.748,6.234)}
\gpcolor{rgb color={0.000,1.000,0.733}}
\gppoint{gp mark 2}{(7.697,6.304)}
\gpcolor{rgb color={0.000,1.000,0.800}}
\gppoint{gp mark 2}{(7.646,6.374)}
\gpcolor{rgb color={0.000,1.000,0.867}}
\gppoint{gp mark 2}{(7.026,6.425)}
\gpcolor{rgb color={0.000,1.000,0.933}}
\gppoint{gp mark 2}{(6.928,6.491)}
\gpcolor{rgb color={0.000,1.000,1.000}}
\gppoint{gp mark 2}{(6.244,6.618)}
\gpcolor{rgb color={0.000,0.933,1.000}}
\gppoint{gp mark 2}{(6.110,6.679)}
\gpcolor{rgb color={0.000,0.867,1.000}}
\gppoint{gp mark 2}{(5.972,6.742)}
\gpcolor{rgb color={0.000,0.800,1.000}}
\gppoint{gp mark 2}{(5.837,6.803)}
\gpcolor{rgb color={0.000,0.733,1.000}}
\gppoint{gp mark 2}{(5.702,6.865)}
\gpcolor{rgb color={0.000,0.667,1.000}}
\gppoint{gp mark 2}{(5.568,6.926)}
\gpcolor{rgb color={0.000,0.600,1.000}}
\gppoint{gp mark 2}{(5.434,6.986)}
\gpcolor{rgb color={0.000,0.533,1.000}}
\gppoint{gp mark 2}{(5.300,7.047)}
\gpcolor{rgb color={0.000,0.467,1.000}}
\gppoint{gp mark 2}{(5.159,7.111)}
\gpcolor{rgb color={0.000,0.400,1.000}}
\gppoint{gp mark 2}{(5.015,7.176)}
\gpcolor{rgb color={0.000,0.333,1.000}}
\gppoint{gp mark 2}{(5.015,7.176)}
\gpcolor{rgb color={0.000,0.267,1.000}}
\gppoint{gp mark 2}{(5.015,7.176)}
\gpcolor{rgb color={0.000,0.200,1.000}}
\gppoint{gp mark 2}{(5.015,7.176)}
\gpcolor{rgb color={0.000,0.133,1.000}}
\gppoint{gp mark 2}{(5.015,7.176)}
\gpcolor{rgb color={0.000,0.067,1.000}}
\gppoint{gp mark 2}{(5.015,7.176)}
\gpcolor{rgb color={0.000,0.000,1.000}}
\gppoint{gp mark 2}{(5.015,7.176)}
\gpcolor{rgb color={0.067,0.000,1.000}}
\gppoint{gp mark 2}{(4.881,7.237)}
\gpcolor{rgb color={0.133,0.000,1.000}}
\gppoint{gp mark 2}{(3.995,5.467)}
\gpcolor{rgb color={0.200,0.000,1.000}}
\gppoint{gp mark 2}{(3.995,5.467)}
\gpcolor{rgb color={0.267,0.000,1.000}}
\gppoint{gp mark 2}{(3.995,5.467)}
\gpcolor{rgb color={0.333,0.000,1.000}}
\gppoint{gp mark 2}{(3.995,5.467)}
\gpcolor{rgb color={0.400,0.000,1.000}}
\gppoint{gp mark 2}{(3.995,5.467)}
\gpcolor{rgb color={0.467,0.000,1.000}}
\gppoint{gp mark 2}{(4.319,5.272)}
\gpcolor{rgb color={0.533,0.000,1.000}}
\gppoint{gp mark 2}{(4.471,5.137)}
\gpcolor{rgb color={0.600,0.000,1.000}}
\gppoint{gp mark 2}{(4.489,5.119)}
\gpcolor{rgb color={0.667,0.000,1.000}}
\gppoint{gp mark 2}{(4.507,5.100)}
\gpcolor{rgb color={0.733,0.000,1.000}}
\gppoint{gp mark 2}{(4.713,5.071)}
\gpcolor{rgb color={0.800,0.000,1.000}}
\gppoint{gp mark 2}{(4.748,5.053)}
\gpcolor{rgb color={0.867,0.000,1.000}}
\gppoint{gp mark 2}{(4.784,5.035)}
\gpcolor{rgb color={0.933,0.000,1.000}}
\gppoint{gp mark 2}{(4.818,5.018)}
\gpcolor{rgb color={1.000,0.000,1.000}}
\gppoint{gp mark 2}{(4.852,5.002)}
\gpcolor{rgb color={1.000,0.000,0.933}}
\gppoint{gp mark 2}{(4.886,4.985)}
\gpcolor{rgb color={1.000,0.000,0.867}}
\gppoint{gp mark 2}{(4.920,4.968)}
\gpcolor{rgb color={1.000,0.000,0.800}}
\gppoint{gp mark 2}{(4.954,4.951)}
\gpcolor{rgb color={1.000,0.000,0.733}}
\gppoint{gp mark 2}{(5.463,4.299)}
\gpcolor{rgb color={1.000,0.000,0.667}}
\gppoint{gp mark 2}{(5.508,4.239)}
\gpcolor{rgb color={1.000,0.000,0.600}}
\gppoint{gp mark 2}{(5.557,4.175)}
\gpcolor{rgb color={1.000,0.000,0.533}}
\gppoint{gp mark 2}{(5.606,4.111)}
\gpcolor{rgb color={1.000,0.000,0.467}}
\gppoint{gp mark 2}{(5.651,4.052)}
\gpcolor{rgb color={1.000,0.000,0.400}}
\gppoint{gp mark 2}{(5.697,3.992)}
\gpcolor{rgb color={1.000,0.000,0.333}}
\gppoint{gp mark 2}{(5.742,3.932)}
\gpcolor{rgb color={1.000,0.000,0.267}}
\gppoint{gp mark 2}{(5.788,3.872)}
\gpcolor{rgb color={1.000,0.000,0.200}}
\gppoint{gp mark 2}{(6.437,3.437)}
\gpcolor{rgb color={1.000,0.000,0.133}}
\gppoint{gp mark 2}{(6.557,3.268)}
\gpcolor{rgb color={1.000,0.000,0.067}}
\gppoint{gp mark 2}{(6.557,3.268)}
\gpcolor{color=gp lt color border}
\gpsetlinetype{gp lt border}
\draw[gp path] (10.700,3.063)--(7.442,1.204);
\draw[gp path] (1.800,2.277)--(7.442,1.204);
\node[gp node center,font={\fontsize{9pt}{10.8pt}\selectfont}] at (3.807,1.276) {x};
\node[gp node center,font={\fontsize{9pt}{10.8pt}\selectfont}] at (10.482,1.865) {y};
\node[gp node center,font={\fontsize{9pt}{10.8pt}\selectfont}] at (0.512,4.755) {z};
\gpfill{rgb color={1.000,0.002,0.000}} (10.869,3.897)--(11.319,3.897)--(11.319,3.919)--(10.869,3.919)--cycle;
\gpfill{rgb color={1.000,0.048,0.000}} (10.869,3.918)--(11.319,3.918)--(11.319,3.941)--(10.869,3.941)--cycle;
\gpfill{rgb color={1.000,0.095,0.000}} (10.869,3.940)--(11.319,3.940)--(11.319,3.962)--(10.869,3.962)--cycle;
\gpfill{rgb color={1.000,0.141,0.000}} (10.869,3.961)--(11.319,3.961)--(11.319,3.984)--(10.869,3.984)--cycle;
\gpfill{rgb color={1.000,0.189,0.000}} (10.869,3.983)--(11.319,3.983)--(11.319,4.006)--(10.869,4.006)--cycle;
\gpfill{rgb color={1.000,0.236,0.000}} (10.869,4.005)--(11.319,4.005)--(11.319,4.027)--(10.869,4.027)--cycle;
\gpfill{rgb color={1.000,0.282,0.000}} (10.869,4.026)--(11.319,4.026)--(11.319,4.049)--(10.869,4.049)--cycle;
\gpfill{rgb color={1.000,0.330,0.000}} (10.869,4.048)--(11.319,4.048)--(11.319,4.070)--(10.869,4.070)--cycle;
\gpfill{rgb color={1.000,0.375,0.000}} (10.869,4.069)--(11.319,4.069)--(11.319,4.092)--(10.869,4.092)--cycle;
\gpfill{rgb color={1.000,0.423,0.000}} (10.869,4.091)--(11.319,4.091)--(11.319,4.114)--(10.869,4.114)--cycle;
\gpfill{rgb color={1.000,0.471,0.000}} (10.869,4.113)--(11.319,4.113)--(11.319,4.135)--(10.869,4.135)--cycle;
\gpfill{rgb color={1.000,0.516,0.000}} (10.869,4.134)--(11.319,4.134)--(11.319,4.157)--(10.869,4.157)--cycle;
\gpfill{rgb color={1.000,0.564,0.000}} (10.869,4.156)--(11.319,4.156)--(11.319,4.178)--(10.869,4.178)--cycle;
\gpfill{rgb color={1.000,0.610,0.000}} (10.869,4.177)--(11.319,4.177)--(11.319,4.200)--(10.869,4.200)--cycle;
\gpfill{rgb color={1.000,0.657,0.000}} (10.869,4.199)--(11.319,4.199)--(11.319,4.222)--(10.869,4.222)--cycle;
\gpfill{rgb color={1.000,0.705,0.000}} (10.869,4.221)--(11.319,4.221)--(11.319,4.243)--(10.869,4.243)--cycle;
\gpfill{rgb color={1.000,0.751,0.000}} (10.869,4.242)--(11.319,4.242)--(11.319,4.265)--(10.869,4.265)--cycle;
\gpfill{rgb color={1.000,0.798,0.000}} (10.869,4.264)--(11.319,4.264)--(11.319,4.286)--(10.869,4.286)--cycle;
\gpfill{rgb color={1.000,0.844,0.000}} (10.869,4.285)--(11.319,4.285)--(11.319,4.308)--(10.869,4.308)--cycle;
\gpfill{rgb color={1.000,0.892,0.000}} (10.869,4.307)--(11.319,4.307)--(11.319,4.330)--(10.869,4.330)--cycle;
\gpfill{rgb color={1.000,0.939,0.000}} (10.869,4.329)--(11.319,4.329)--(11.319,4.351)--(10.869,4.351)--cycle;
\gpfill{rgb color={1.000,0.985,0.000}} (10.869,4.350)--(11.319,4.350)--(11.319,4.373)--(10.869,4.373)--cycle;
\gpfill{rgb color={0.967,1.000,0.000}} (10.869,4.372)--(11.319,4.372)--(11.319,4.395)--(10.869,4.395)--cycle;
\gpfill{rgb color={0.920,1.000,0.000}} (10.869,4.394)--(11.319,4.394)--(11.319,4.416)--(10.869,4.416)--cycle;
\gpfill{rgb color={0.874,1.000,0.000}} (10.869,4.415)--(11.319,4.415)--(11.319,4.438)--(10.869,4.438)--cycle;
\gpfill{rgb color={0.826,1.000,0.000}} (10.869,4.437)--(11.319,4.437)--(11.319,4.459)--(10.869,4.459)--cycle;
\gpfill{rgb color={0.781,1.000,0.000}} (10.869,4.458)--(11.319,4.458)--(11.319,4.481)--(10.869,4.481)--cycle;
\gpfill{rgb color={0.733,1.000,0.000}} (10.869,4.480)--(11.319,4.480)--(11.319,4.503)--(10.869,4.503)--cycle;
\gpfill{rgb color={0.685,1.000,0.000}} (10.869,4.502)--(11.319,4.502)--(11.319,4.524)--(10.869,4.524)--cycle;
\gpfill{rgb color={0.640,1.000,0.000}} (10.869,4.523)--(11.319,4.523)--(11.319,4.546)--(10.869,4.546)--cycle;
\gpfill{rgb color={0.592,1.000,0.000}} (10.869,4.545)--(11.319,4.545)--(11.319,4.567)--(10.869,4.567)--cycle;
\gpfill{rgb color={0.547,1.000,0.000}} (10.869,4.566)--(11.319,4.566)--(11.319,4.589)--(10.869,4.589)--cycle;
\gpfill{rgb color={0.499,1.000,0.000}} (10.869,4.588)--(11.319,4.588)--(11.319,4.611)--(10.869,4.611)--cycle;
\gpfill{rgb color={0.451,1.000,0.000}} (10.869,4.610)--(11.319,4.610)--(11.319,4.632)--(10.869,4.632)--cycle;
\gpfill{rgb color={0.406,1.000,0.000}} (10.869,4.631)--(11.319,4.631)--(11.319,4.654)--(10.869,4.654)--cycle;
\gpfill{rgb color={0.358,1.000,0.000}} (10.869,4.653)--(11.319,4.653)--(11.319,4.675)--(10.869,4.675)--cycle;
\gpfill{rgb color={0.312,1.000,0.000}} (10.869,4.674)--(11.319,4.674)--(11.319,4.697)--(10.869,4.697)--cycle;
\gpfill{rgb color={0.265,1.000,0.000}} (10.869,4.696)--(11.319,4.696)--(11.319,4.719)--(10.869,4.719)--cycle;
\gpfill{rgb color={0.217,1.000,0.000}} (10.869,4.718)--(11.319,4.718)--(11.319,4.740)--(10.869,4.740)--cycle;
\gpfill{rgb color={0.171,1.000,0.000}} (10.869,4.739)--(11.319,4.739)--(11.319,4.762)--(10.869,4.762)--cycle;
\gpfill{rgb color={0.124,1.000,0.000}} (10.869,4.761)--(11.319,4.761)--(11.319,4.783)--(10.869,4.783)--cycle;
\gpfill{rgb color={0.078,1.000,0.000}} (10.869,4.782)--(11.319,4.782)--(11.319,4.805)--(10.869,4.805)--cycle;
\gpfill{rgb color={0.030,1.000,0.000}} (10.869,4.804)--(11.319,4.804)--(11.319,4.827)--(10.869,4.827)--cycle;
\gpfill{rgb color={0.000,1.000,0.017}} (10.869,4.826)--(11.319,4.826)--(11.319,4.848)--(10.869,4.848)--cycle;
\gpfill{rgb color={0.000,1.000,0.063}} (10.869,4.847)--(11.319,4.847)--(11.319,4.870)--(10.869,4.870)--cycle;
\gpfill{rgb color={0.000,1.000,0.111}} (10.869,4.869)--(11.319,4.869)--(11.319,4.892)--(10.869,4.892)--cycle;
\gpfill{rgb color={0.000,1.000,0.158}} (10.869,4.891)--(11.319,4.891)--(11.319,4.913)--(10.869,4.913)--cycle;
\gpfill{rgb color={0.000,1.000,0.204}} (10.869,4.912)--(11.319,4.912)--(11.319,4.935)--(10.869,4.935)--cycle;
\gpfill{rgb color={0.000,1.000,0.252}} (10.869,4.934)--(11.319,4.934)--(11.319,4.956)--(10.869,4.956)--cycle;
\gpfill{rgb color={0.000,1.000,0.297}} (10.869,4.955)--(11.319,4.955)--(11.319,4.978)--(10.869,4.978)--cycle;
\gpfill{rgb color={0.000,1.000,0.345}} (10.869,4.977)--(11.319,4.977)--(11.319,5.000)--(10.869,5.000)--cycle;
\gpfill{rgb color={0.000,1.000,0.393}} (10.869,4.999)--(11.319,4.999)--(11.319,5.021)--(10.869,5.021)--cycle;
\gpfill{rgb color={0.000,1.000,0.438}} (10.869,5.020)--(11.319,5.020)--(11.319,5.043)--(10.869,5.043)--cycle;
\gpfill{rgb color={0.000,1.000,0.486}} (10.869,5.042)--(11.319,5.042)--(11.319,5.064)--(10.869,5.064)--cycle;
\gpfill{rgb color={0.000,1.000,0.531}} (10.869,5.063)--(11.319,5.063)--(11.319,5.086)--(10.869,5.086)--cycle;
\gpfill{rgb color={0.000,1.000,0.579}} (10.869,5.085)--(11.319,5.085)--(11.319,5.108)--(10.869,5.108)--cycle;
\gpfill{rgb color={0.000,1.000,0.627}} (10.869,5.107)--(11.319,5.107)--(11.319,5.129)--(10.869,5.129)--cycle;
\gpfill{rgb color={0.000,1.000,0.672}} (10.869,5.128)--(11.319,5.128)--(11.319,5.151)--(10.869,5.151)--cycle;
\gpfill{rgb color={0.000,1.000,0.720}} (10.869,5.150)--(11.319,5.150)--(11.319,5.172)--(10.869,5.172)--cycle;
\gpfill{rgb color={0.000,1.000,0.766}} (10.869,5.171)--(11.319,5.171)--(11.319,5.194)--(10.869,5.194)--cycle;
\gpfill{rgb color={0.000,1.000,0.813}} (10.869,5.193)--(11.319,5.193)--(11.319,5.216)--(10.869,5.216)--cycle;
\gpfill{rgb color={0.000,1.000,0.861}} (10.869,5.215)--(11.319,5.215)--(11.319,5.237)--(10.869,5.237)--cycle;
\gpfill{rgb color={0.000,1.000,0.907}} (10.869,5.236)--(11.319,5.236)--(11.319,5.259)--(10.869,5.259)--cycle;
\gpfill{rgb color={0.000,1.000,0.954}} (10.869,5.258)--(11.319,5.258)--(11.319,5.281)--(10.869,5.281)--cycle;
\gpfill{rgb color={0.000,0.998,1.000}} (10.869,5.280)--(11.319,5.280)--(11.319,5.302)--(10.869,5.302)--cycle;
\gpfill{rgb color={0.000,0.952,1.000}} (10.869,5.301)--(11.319,5.301)--(11.319,5.324)--(10.869,5.324)--cycle;
\gpfill{rgb color={0.000,0.905,1.000}} (10.869,5.323)--(11.319,5.323)--(11.319,5.345)--(10.869,5.345)--cycle;
\gpfill{rgb color={0.000,0.859,1.000}} (10.869,5.344)--(11.319,5.344)--(11.319,5.367)--(10.869,5.367)--cycle;
\gpfill{rgb color={0.000,0.811,1.000}} (10.869,5.366)--(11.319,5.366)--(11.319,5.389)--(10.869,5.389)--cycle;
\gpfill{rgb color={0.000,0.764,1.000}} (10.869,5.388)--(11.319,5.388)--(11.319,5.410)--(10.869,5.410)--cycle;
\gpfill{rgb color={0.000,0.718,1.000}} (10.869,5.409)--(11.319,5.409)--(11.319,5.432)--(10.869,5.432)--cycle;
\gpfill{rgb color={0.000,0.670,1.000}} (10.869,5.431)--(11.319,5.431)--(11.319,5.453)--(10.869,5.453)--cycle;
\gpfill{rgb color={0.000,0.625,1.000}} (10.869,5.452)--(11.319,5.452)--(11.319,5.475)--(10.869,5.475)--cycle;
\gpfill{rgb color={0.000,0.577,1.000}} (10.869,5.474)--(11.319,5.474)--(11.319,5.497)--(10.869,5.497)--cycle;
\gpfill{rgb color={0.000,0.529,1.000}} (10.869,5.496)--(11.319,5.496)--(11.319,5.518)--(10.869,5.518)--cycle;
\gpfill{rgb color={0.000,0.484,1.000}} (10.869,5.517)--(11.319,5.517)--(11.319,5.540)--(10.869,5.540)--cycle;
\gpfill{rgb color={0.000,0.436,1.000}} (10.869,5.539)--(11.319,5.539)--(11.319,5.561)--(10.869,5.561)--cycle;
\gpfill{rgb color={0.000,0.390,1.000}} (10.869,5.560)--(11.319,5.560)--(11.319,5.583)--(10.869,5.583)--cycle;
\gpfill{rgb color={0.000,0.343,1.000}} (10.869,5.582)--(11.319,5.582)--(11.319,5.605)--(10.869,5.605)--cycle;
\gpfill{rgb color={0.000,0.295,1.000}} (10.869,5.604)--(11.319,5.604)--(11.319,5.626)--(10.869,5.626)--cycle;
\gpfill{rgb color={0.000,0.249,1.000}} (10.869,5.625)--(11.319,5.625)--(11.319,5.648)--(10.869,5.648)--cycle;
\gpfill{rgb color={0.000,0.202,1.000}} (10.869,5.647)--(11.319,5.647)--(11.319,5.669)--(10.869,5.669)--cycle;
\gpfill{rgb color={0.000,0.156,1.000}} (10.869,5.668)--(11.319,5.668)--(11.319,5.691)--(10.869,5.691)--cycle;
\gpfill{rgb color={0.000,0.108,1.000}} (10.869,5.690)--(11.319,5.690)--(11.319,5.713)--(10.869,5.713)--cycle;
\gpfill{rgb color={0.000,0.061,1.000}} (10.869,5.712)--(11.319,5.712)--(11.319,5.734)--(10.869,5.734)--cycle;
\gpfill{rgb color={0.000,0.015,1.000}} (10.869,5.733)--(11.319,5.733)--(11.319,5.756)--(10.869,5.756)--cycle;
\gpfill{rgb color={0.033,0.000,1.000}} (10.869,5.755)--(11.319,5.755)--(11.319,5.778)--(10.869,5.778)--cycle;
\gpfill{rgb color={0.080,0.000,1.000}} (10.869,5.777)--(11.319,5.777)--(11.319,5.799)--(10.869,5.799)--cycle;
\gpfill{rgb color={0.126,0.000,1.000}} (10.869,5.798)--(11.319,5.798)--(11.319,5.821)--(10.869,5.821)--cycle;
\gpfill{rgb color={0.174,0.000,1.000}} (10.869,5.820)--(11.319,5.820)--(11.319,5.842)--(10.869,5.842)--cycle;
\gpfill{rgb color={0.219,0.000,1.000}} (10.869,5.841)--(11.319,5.841)--(11.319,5.864)--(10.869,5.864)--cycle;
\gpfill{rgb color={0.267,0.000,1.000}} (10.869,5.863)--(11.319,5.863)--(11.319,5.886)--(10.869,5.886)--cycle;
\gpfill{rgb color={0.315,0.000,1.000}} (10.869,5.885)--(11.319,5.885)--(11.319,5.907)--(10.869,5.907)--cycle;
\gpfill{rgb color={0.360,0.000,1.000}} (10.869,5.906)--(11.319,5.906)--(11.319,5.929)--(10.869,5.929)--cycle;
\gpfill{rgb color={0.408,0.000,1.000}} (10.869,5.928)--(11.319,5.928)--(11.319,5.950)--(10.869,5.950)--cycle;
\gpfill{rgb color={0.453,0.000,1.000}} (10.869,5.949)--(11.319,5.949)--(11.319,5.972)--(10.869,5.972)--cycle;
\gpfill{rgb color={0.501,0.000,1.000}} (10.869,5.971)--(11.319,5.971)--(11.319,5.994)--(10.869,5.994)--cycle;
\gpfill{rgb color={0.549,0.000,1.000}} (10.869,5.993)--(11.319,5.993)--(11.319,6.015)--(10.869,6.015)--cycle;
\gpfill{rgb color={0.594,0.000,1.000}} (10.869,6.014)--(11.319,6.014)--(11.319,6.037)--(10.869,6.037)--cycle;
\gpfill{rgb color={0.642,0.000,1.000}} (10.869,6.036)--(11.319,6.036)--(11.319,6.058)--(10.869,6.058)--cycle;
\gpfill{rgb color={0.688,0.000,1.000}} (10.869,6.057)--(11.319,6.057)--(11.319,6.080)--(10.869,6.080)--cycle;
\gpfill{rgb color={0.735,0.000,1.000}} (10.869,6.079)--(11.319,6.079)--(11.319,6.102)--(10.869,6.102)--cycle;
\gpfill{rgb color={0.783,0.000,1.000}} (10.869,6.101)--(11.319,6.101)--(11.319,6.123)--(10.869,6.123)--cycle;
\gpfill{rgb color={0.829,0.000,1.000}} (10.869,6.122)--(11.319,6.122)--(11.319,6.145)--(10.869,6.145)--cycle;
\gpfill{rgb color={0.876,0.000,1.000}} (10.869,6.144)--(11.319,6.144)--(11.319,6.166)--(10.869,6.166)--cycle;
\gpfill{rgb color={0.922,0.000,1.000}} (10.869,6.165)--(11.319,6.165)--(11.319,6.188)--(10.869,6.188)--cycle;
\gpfill{rgb color={0.970,0.000,1.000}} (10.869,6.187)--(11.319,6.187)--(11.319,6.210)--(10.869,6.210)--cycle;
\gpfill{rgb color={1.000,0.000,0.983}} (10.869,6.209)--(11.319,6.209)--(11.319,6.231)--(10.869,6.231)--cycle;
\gpfill{rgb color={1.000,0.000,0.937}} (10.869,6.230)--(11.319,6.230)--(11.319,6.253)--(10.869,6.253)--cycle;
\gpfill{rgb color={1.000,0.000,0.889}} (10.869,6.252)--(11.319,6.252)--(11.319,6.275)--(10.869,6.275)--cycle;
\gpfill{rgb color={1.000,0.000,0.842}} (10.869,6.274)--(11.319,6.274)--(11.319,6.296)--(10.869,6.296)--cycle;
\gpfill{rgb color={1.000,0.000,0.796}} (10.869,6.295)--(11.319,6.295)--(11.319,6.318)--(10.869,6.318)--cycle;
\gpfill{rgb color={1.000,0.000,0.748}} (10.869,6.317)--(11.319,6.317)--(11.319,6.339)--(10.869,6.339)--cycle;
\gpfill{rgb color={1.000,0.000,0.703}} (10.869,6.338)--(11.319,6.338)--(11.319,6.361)--(10.869,6.361)--cycle;
\gpfill{rgb color={1.000,0.000,0.655}} (10.869,6.360)--(11.319,6.360)--(11.319,6.383)--(10.869,6.383)--cycle;
\gpfill{rgb color={1.000,0.000,0.607}} (10.869,6.382)--(11.319,6.382)--(11.319,6.404)--(10.869,6.404)--cycle;
\gpfill{rgb color={1.000,0.000,0.562}} (10.869,6.403)--(11.319,6.403)--(11.319,6.426)--(10.869,6.426)--cycle;
\gpfill{rgb color={1.000,0.000,0.514}} (10.869,6.425)--(11.319,6.425)--(11.319,6.447)--(10.869,6.447)--cycle;
\gpfill{rgb color={1.000,0.000,0.469}} (10.869,6.446)--(11.319,6.446)--(11.319,6.469)--(10.869,6.469)--cycle;
\gpfill{rgb color={1.000,0.000,0.421}} (10.869,6.468)--(11.319,6.468)--(11.319,6.491)--(10.869,6.491)--cycle;
\gpfill{rgb color={1.000,0.000,0.373}} (10.869,6.490)--(11.319,6.490)--(11.319,6.512)--(10.869,6.512)--cycle;
\gpfill{rgb color={1.000,0.000,0.328}} (10.869,6.511)--(11.319,6.511)--(11.319,6.534)--(10.869,6.534)--cycle;
\gpfill{rgb color={1.000,0.000,0.280}} (10.869,6.533)--(11.319,6.533)--(11.319,6.555)--(10.869,6.555)--cycle;
\gpfill{rgb color={1.000,0.000,0.234}} (10.869,6.554)--(11.319,6.554)--(11.319,6.577)--(10.869,6.577)--cycle;
\gpfill{rgb color={1.000,0.000,0.187}} (10.869,6.576)--(11.319,6.576)--(11.319,6.599)--(10.869,6.599)--cycle;
\gpfill{rgb color={1.000,0.000,0.139}} (10.869,6.598)--(11.319,6.598)--(11.319,6.620)--(10.869,6.620)--cycle;
\gpfill{rgb color={1.000,0.000,0.093}} (10.869,6.619)--(11.319,6.619)--(11.319,6.642)--(10.869,6.642)--cycle;
\gpfill{rgb color={1.000,0.000,0.046}} (10.869,6.641)--(11.319,6.641)--(11.319,6.663)--(10.869,6.663)--cycle;
\draw[gp path] (10.869,3.897)--(11.319,3.897)--(11.319,6.663)--(10.869,6.663)--cycle;
\draw[gp path] (11.319,3.897)--(11.139,3.897);
\node[gp node left,font={\fontsize{9pt}{10.8pt}\selectfont}] at (11.503,3.897) { 62};
\draw[gp path] (10.869,3.897)--(11.049,3.897);
\draw[gp path] (11.319,4.204)--(11.139,4.204);
\node[gp node left,font={\fontsize{9pt}{10.8pt}\selectfont}] at (11.503,4.204) { 64};
\draw[gp path] (10.869,4.204)--(11.049,4.204);
\draw[gp path] (11.319,4.511)--(11.139,4.511);
\node[gp node left,font={\fontsize{9pt}{10.8pt}\selectfont}] at (11.503,4.511) { 66};
\draw[gp path] (10.869,4.511)--(11.049,4.511);
\draw[gp path] (11.319,4.819)--(11.139,4.819);
\node[gp node left,font={\fontsize{9pt}{10.8pt}\selectfont}] at (11.503,4.819) { 68};
\draw[gp path] (10.869,4.819)--(11.049,4.819);
\draw[gp path] (11.319,5.126)--(11.139,5.126);
\node[gp node left,font={\fontsize{9pt}{10.8pt}\selectfont}] at (11.503,5.126) { 70};
\draw[gp path] (10.869,5.126)--(11.049,5.126);
\draw[gp path] (11.319,5.433)--(11.139,5.433);
\node[gp node left,font={\fontsize{9pt}{10.8pt}\selectfont}] at (11.503,5.433) { 72};
\draw[gp path] (10.869,5.433)--(11.049,5.433);
\draw[gp path] (11.319,5.741)--(11.139,5.741);
\node[gp node left,font={\fontsize{9pt}{10.8pt}\selectfont}] at (11.503,5.741) { 74};
\draw[gp path] (10.869,5.741)--(11.049,5.741);
\draw[gp path] (11.319,6.048)--(11.139,6.048);
\node[gp node left,font={\fontsize{9pt}{10.8pt}\selectfont}] at (11.503,6.048) { 76};
\draw[gp path] (10.869,6.048)--(11.049,6.048);
\draw[gp path] (11.319,6.355)--(11.139,6.355);
\node[gp node left,font={\fontsize{9pt}{10.8pt}\selectfont}] at (11.503,6.355) { 78};
\draw[gp path] (10.869,6.355)--(11.049,6.355);
\draw[gp path] (11.319,6.663)--(11.139,6.663);
\node[gp node left,font={\fontsize{9pt}{10.8pt}\selectfont}] at (11.503,6.663) { 80};
\draw[gp path] (10.869,6.663)--(11.049,6.663);
\node[gp node center,rotate=-270,font={\fontsize{9pt}{10.8pt}\selectfont}] at (12.147,5.280) {time (s)};
%% coordinates of the plot area
\gpdefrectangularnode{gp plot 1}{\pgfpoint{1.800cm}{0.771cm}}{\pgfpoint{10.700cm}{8.287cm}}
\end{tikzpicture}
%% gnuplot variables

%     }
%     \caption{Kalman.}
%     \label{fig:ball_kalman_3d}
% \end{figure}

% \begin{figure}[htb]
%     \centering
%     \resizebox{\columnwidth}{!}{%
%         \begin{tikzpicture}[gnuplot]
%% generated with GNUPLOT 4.6p4 (Lua 5.1; terminal rev. 99, script rev. 100)
%% Thu 28 May 2015 09:06:58 AM CEST
\path (0.000,0.000) rectangle (12.500,8.750);
\gpcolor{color=gp lt color border}
\gpsetlinetype{gp lt border}
\gpsetlinewidth{1.00}
\draw[gp path] (1.688,0.985)--(1.868,0.985);
\draw[gp path] (11.947,0.985)--(11.767,0.985);
\node[gp node right,font={\fontsize{9pt}{10.8pt}\selectfont}] at (1.504,0.985) { 0};
\draw[gp path] (1.688,2.218)--(1.868,2.218);
\draw[gp path] (11.947,2.218)--(11.767,2.218);
\node[gp node right,font={\fontsize{9pt}{10.8pt}\selectfont}] at (1.504,2.218) { 0.05};
\draw[gp path] (1.688,3.450)--(1.868,3.450);
\draw[gp path] (11.947,3.450)--(11.767,3.450);
\node[gp node right,font={\fontsize{9pt}{10.8pt}\selectfont}] at (1.504,3.450) { 0.1};
\draw[gp path] (1.688,4.683)--(1.868,4.683);
\draw[gp path] (11.947,4.683)--(11.767,4.683);
\node[gp node right,font={\fontsize{9pt}{10.8pt}\selectfont}] at (1.504,4.683) { 0.15};
\draw[gp path] (1.688,5.916)--(1.868,5.916);
\draw[gp path] (11.947,5.916)--(11.767,5.916);
\node[gp node right,font={\fontsize{9pt}{10.8pt}\selectfont}] at (1.504,5.916) { 0.2};
\draw[gp path] (1.688,7.148)--(1.868,7.148);
\draw[gp path] (11.947,7.148)--(11.767,7.148);
\node[gp node right,font={\fontsize{9pt}{10.8pt}\selectfont}] at (1.504,7.148) { 0.25};
\draw[gp path] (1.688,8.381)--(1.868,8.381);
\draw[gp path] (11.947,8.381)--(11.767,8.381);
\node[gp node right,font={\fontsize{9pt}{10.8pt}\selectfont}] at (1.504,8.381) { 0.3};
\draw[gp path] (1.688,0.985)--(1.688,1.165);
\draw[gp path] (1.688,8.381)--(1.688,8.201);
\node[gp node center,font={\fontsize{9pt}{10.8pt}\selectfont}] at (1.688,0.677) { 64};
\draw[gp path] (2.970,0.985)--(2.970,1.165);
\draw[gp path] (2.970,8.381)--(2.970,8.201);
\node[gp node center,font={\fontsize{9pt}{10.8pt}\selectfont}] at (2.970,0.677) { 66};
\draw[gp path] (4.253,0.985)--(4.253,1.165);
\draw[gp path] (4.253,8.381)--(4.253,8.201);
\node[gp node center,font={\fontsize{9pt}{10.8pt}\selectfont}] at (4.253,0.677) { 68};
\draw[gp path] (5.535,0.985)--(5.535,1.165);
\draw[gp path] (5.535,8.381)--(5.535,8.201);
\node[gp node center,font={\fontsize{9pt}{10.8pt}\selectfont}] at (5.535,0.677) { 70};
\draw[gp path] (6.818,0.985)--(6.818,1.165);
\draw[gp path] (6.818,8.381)--(6.818,8.201);
\node[gp node center,font={\fontsize{9pt}{10.8pt}\selectfont}] at (6.818,0.677) { 72};
\draw[gp path] (8.100,0.985)--(8.100,1.165);
\draw[gp path] (8.100,8.381)--(8.100,8.201);
\node[gp node center,font={\fontsize{9pt}{10.8pt}\selectfont}] at (8.100,0.677) { 74};
\draw[gp path] (9.382,0.985)--(9.382,1.165);
\draw[gp path] (9.382,8.381)--(9.382,8.201);
\node[gp node center,font={\fontsize{9pt}{10.8pt}\selectfont}] at (9.382,0.677) { 76};
\draw[gp path] (10.665,0.985)--(10.665,1.165);
\draw[gp path] (10.665,8.381)--(10.665,8.201);
\node[gp node center,font={\fontsize{9pt}{10.8pt}\selectfont}] at (10.665,0.677) { 78};
\draw[gp path] (11.947,0.985)--(11.947,1.165);
\draw[gp path] (11.947,8.381)--(11.947,8.201);
\node[gp node center,font={\fontsize{9pt}{10.8pt}\selectfont}] at (11.947,0.677) { 80};
\draw[gp path] (1.688,8.381)--(1.688,0.985)--(11.947,0.985)--(11.947,8.381)--cycle;
\node[gp node center,rotate=-270,font={\fontsize{9pt}{10.8pt}\selectfont}] at (0.246,4.683) {distance (m)};
\node[gp node center,font={\fontsize{9pt}{10.8pt}\selectfont}] at (6.817,0.215) {time (s)};
\gpcolor{rgb color={1.000,0.075,0.000}}
\gpsetlinetype{gp lt plot 0}
\draw[gp path] (1.688,1.463)--(1.816,1.242);
\gpcolor{rgb color={1.000,0.150,0.000}}
\draw[gp path] (1.816,1.242)--(1.944,1.749);
\gpcolor{rgb color={1.000,0.225,0.000}}
\draw[gp path] (1.944,1.749)--(2.073,1.681);
\gpcolor{rgb color={1.000,0.300,0.000}}
\draw[gp path] (2.073,1.681)--(2.201,4.179);
\gpcolor{rgb color={1.000,0.375,0.000}}
\draw[gp path] (2.201,4.179)--(2.329,3.947);
\gpcolor{rgb color={1.000,0.450,0.000}}
\draw[gp path] (2.329,3.947)--(2.457,3.921);
\gpcolor{rgb color={1.000,0.525,0.000}}
\draw[gp path] (2.457,3.921)--(2.586,3.680);
\gpcolor{rgb color={1.000,0.600,0.000}}
\draw[gp path] (2.586,3.680)--(2.714,1.647);
\gpcolor{rgb color={1.000,0.675,0.000}}
\draw[gp path] (2.714,1.647)--(2.842,1.625);
\gpcolor{rgb color={1.000,0.750,0.000}}
\draw[gp path] (2.842,1.625)--(2.970,1.869);
\gpcolor{rgb color={1.000,0.825,0.000}}
\draw[gp path] (2.970,1.869)--(3.099,1.484);
\gpcolor{rgb color={1.000,0.900,0.000}}
\draw[gp path] (3.099,1.484)--(3.227,1.484);
\gpcolor{rgb color={1.000,0.975,0.000}}
\draw[gp path] (3.227,1.484)--(3.355,1.432);
\gpcolor{rgb color={0.950,1.000,0.000}}
\draw[gp path] (3.355,1.432)--(3.483,1.754);
\gpcolor{rgb color={0.875,1.000,0.000}}
\draw[gp path] (3.483,1.754)--(3.612,1.754);
\gpcolor{rgb color={0.800,1.000,0.000}}
\draw[gp path] (3.612,1.754)--(3.740,2.247);
\gpcolor{rgb color={0.725,1.000,0.000}}
\draw[gp path] (3.740,2.247)--(3.868,1.943);
\gpcolor{rgb color={0.650,1.000,0.000}}
\draw[gp path] (3.868,1.943)--(3.996,1.229);
\gpcolor{rgb color={0.575,1.000,0.000}}
\draw[gp path] (3.996,1.229)--(4.125,1.623);
\gpcolor{rgb color={0.500,1.000,0.000}}
\draw[gp path] (4.125,1.623)--(4.253,2.019);
\gpcolor{rgb color={0.425,1.000,0.000}}
\draw[gp path] (4.253,2.019)--(4.381,1.719);
\gpcolor{rgb color={0.350,1.000,0.000}}
\draw[gp path] (4.381,1.719)--(4.509,1.701);
\gpcolor{rgb color={0.275,1.000,0.000}}
\draw[gp path] (4.509,1.701)--(4.637,1.894);
\gpcolor{rgb color={0.200,1.000,0.000}}
\draw[gp path] (4.637,1.894)--(4.766,2.245);
\gpcolor{rgb color={0.125,1.000,0.000}}
\draw[gp path] (4.766,2.245)--(4.894,1.970);
\gpcolor{rgb color={0.050,1.000,0.000}}
\draw[gp path] (4.894,1.970)--(5.022,1.758);
\gpcolor{rgb color={0.000,1.000,0.025}}
\draw[gp path] (5.022,1.758)--(5.150,3.245);
\gpcolor{rgb color={0.000,1.000,0.100}}
\draw[gp path] (5.150,3.245)--(5.279,2.668);
\gpcolor{rgb color={0.000,1.000,0.175}}
\draw[gp path] (5.279,2.668)--(5.407,3.811);
\gpcolor{rgb color={0.000,1.000,0.250}}
\draw[gp path] (5.407,3.811)--(5.535,2.588);
\gpcolor{rgb color={0.000,1.000,0.325}}
\draw[gp path] (5.535,2.588)--(5.663,4.402);
\gpcolor{rgb color={0.000,1.000,0.400}}
\draw[gp path] (5.663,4.402)--(5.792,4.097);
\gpcolor{rgb color={0.000,1.000,0.475}}
\draw[gp path] (5.792,4.097)--(5.920,3.061);
\gpcolor{rgb color={0.000,1.000,0.550}}
\draw[gp path] (5.920,3.061)--(6.048,2.683);
\gpcolor{rgb color={0.000,1.000,0.625}}
\draw[gp path] (6.048,2.683)--(6.176,1.141);
\gpcolor{rgb color={0.000,1.000,0.700}}
\draw[gp path] (6.176,1.141)--(6.305,1.340);
\gpcolor{rgb color={0.000,1.000,0.775}}
\draw[gp path] (6.305,1.340)--(6.433,1.805);
\gpcolor{rgb color={0.000,1.000,0.850}}
\draw[gp path] (6.433,1.805)--(6.561,2.264);
\gpcolor{rgb color={0.000,1.000,0.925}}
\draw[gp path] (6.561,2.264)--(6.689,2.727);
\gpcolor{rgb color={0.000,1.000,1.000}}
\draw[gp path] (6.689,2.727)--(6.818,3.186);
\gpcolor{rgb color={0.000,0.925,1.000}}
\draw[gp path] (6.818,3.186)--(6.946,3.644);
\gpcolor{rgb color={0.000,0.850,1.000}}
\draw[gp path] (6.946,3.644)--(7.074,4.101);
\gpcolor{rgb color={0.000,0.775,1.000}}
\draw[gp path] (7.074,4.101)--(7.202,4.584);
\gpcolor{rgb color={0.000,0.700,1.000}}
\draw[gp path] (7.202,4.584)--(7.330,5.076);
\gpcolor{rgb color={0.000,0.625,1.000}}
\draw[gp path] (7.330,5.076)--(7.459,5.076);
\gpcolor{rgb color={0.000,0.550,1.000}}
\draw[gp path] (7.459,5.076)--(7.587,5.076);
\gpcolor{rgb color={0.000,0.475,1.000}}
\draw[gp path] (7.587,5.076)--(7.715,5.076);
\gpcolor{rgb color={0.000,0.400,1.000}}
\draw[gp path] (7.715,5.076)--(7.843,7.617);
\gpcolor{rgb color={0.000,0.325,1.000}}
\draw[gp path] (7.843,7.617)--(7.972,7.617);
\gpcolor{rgb color={0.000,0.250,1.000}}
\draw[gp path] (7.972,7.617)--(8.100,7.617);
\gpcolor{rgb color={0.000,0.175,1.000}}
\draw[gp path] (8.100,7.617)--(8.228,7.891);
\gpcolor{rgb color={0.000,0.100,1.000}}
\draw[gp path] (8.228,7.891)--(8.356,2.339);
\gpcolor{rgb color={0.000,0.025,1.000}}
\draw[gp path] (8.356,2.339)--(8.485,2.339);
\gpcolor{rgb color={0.050,0.000,1.000}}
\draw[gp path] (8.485,2.339)--(8.613,3.638);
\gpcolor{rgb color={0.125,0.000,1.000}}
\draw[gp path] (8.613,3.638)--(8.741,3.638);
\gpcolor{rgb color={0.200,0.000,1.000}}
\draw[gp path] (8.741,3.638)--(8.869,3.638);
\gpcolor{rgb color={0.275,0.000,1.000}}
\draw[gp path] (8.869,3.638)--(8.998,2.200);
\gpcolor{rgb color={0.350,0.000,1.000}}
\draw[gp path] (8.998,2.200)--(9.126,1.854);
\gpcolor{rgb color={0.425,0.000,1.000}}
\draw[gp path] (9.126,1.854)--(9.254,1.714);
\gpcolor{rgb color={0.500,0.000,1.000}}
\draw[gp path] (9.254,1.714)--(9.382,1.580);
\gpcolor{rgb color={0.575,0.000,1.000}}
\draw[gp path] (9.382,1.580)--(9.510,1.058);
\gpcolor{rgb color={0.650,0.000,1.000}}
\draw[gp path] (9.510,1.058)--(9.639,1.236);
\gpcolor{rgb color={0.725,0.000,1.000}}
\draw[gp path] (9.639,1.236)--(9.767,1.431);
\gpcolor{rgb color={0.800,0.000,1.000}}
\draw[gp path] (9.767,1.431)--(9.895,1.616);
\gpcolor{rgb color={0.875,0.000,1.000}}
\draw[gp path] (9.895,1.616)--(10.023,5.336);
\gpcolor{rgb color={0.950,0.000,1.000}}
\draw[gp path] (10.023,5.336)--(10.152,5.159);
\gpcolor{rgb color={1.000,0.000,0.975}}
\draw[gp path] (10.152,5.159)--(10.280,4.982);
\gpcolor{rgb color={1.000,0.000,0.900}}
\draw[gp path] (10.280,4.982)--(10.408,4.805);
\gpcolor{rgb color={1.000,0.000,0.825}}
\draw[gp path] (10.408,4.805)--(10.536,1.137);
\gpcolor{rgb color={1.000,0.000,0.750}}
\draw[gp path] (10.536,1.137)--(10.665,1.545);
\gpcolor{rgb color={1.000,0.000,0.675}}
\draw[gp path] (10.665,1.545)--(10.793,2.000);
\gpcolor{rgb color={1.000,0.000,0.600}}
\draw[gp path] (10.793,2.000)--(10.921,2.459);
\gpcolor{rgb color={1.000,0.000,0.525}}
\draw[gp path] (10.921,2.459)--(11.049,3.818);
\gpcolor{rgb color={1.000,0.000,0.450}}
\draw[gp path] (11.049,3.818)--(11.178,3.493);
\gpcolor{rgb color={1.000,0.000,0.375}}
\draw[gp path] (11.178,3.493)--(11.306,3.204);
\gpcolor{rgb color={1.000,0.000,0.300}}
\draw[gp path] (11.306,3.204)--(11.434,2.966);
\gpcolor{rgb color={1.000,0.000,0.225}}
\draw[gp path] (11.434,2.966)--(11.562,1.262);
\gpcolor{rgb color={1.000,0.000,0.150}}
\draw[gp path] (11.562,1.262)--(11.691,1.936);
\gpcolor{rgb color={1.000,0.000,0.075}}
\draw[gp path] (11.691,1.936)--(11.819,1.936);
\gpcolor{rgb color={1.000,0.000,0.000}}
\draw[gp path] (11.819,1.936)--(11.947,2.217);
\gpsetpointsize{4.00}
\gppoint{gp mark 7}{(1.688,1.463)}
\gpcolor{rgb color={1.000,0.075,0.000}}
\gppoint{gp mark 7}{(1.816,1.242)}
\gpcolor{rgb color={1.000,0.150,0.000}}
\gppoint{gp mark 7}{(1.944,1.749)}
\gpcolor{rgb color={1.000,0.225,0.000}}
\gppoint{gp mark 7}{(2.073,1.681)}
\gpcolor{rgb color={1.000,0.300,0.000}}
\gppoint{gp mark 7}{(2.201,4.179)}
\gpcolor{rgb color={1.000,0.375,0.000}}
\gppoint{gp mark 7}{(2.329,3.947)}
\gpcolor{rgb color={1.000,0.450,0.000}}
\gppoint{gp mark 7}{(2.457,3.921)}
\gpcolor{rgb color={1.000,0.525,0.000}}
\gppoint{gp mark 7}{(2.586,3.680)}
\gpcolor{rgb color={1.000,0.600,0.000}}
\gppoint{gp mark 7}{(2.714,1.647)}
\gpcolor{rgb color={1.000,0.675,0.000}}
\gppoint{gp mark 7}{(2.842,1.625)}
\gpcolor{rgb color={1.000,0.750,0.000}}
\gppoint{gp mark 7}{(2.970,1.869)}
\gpcolor{rgb color={1.000,0.825,0.000}}
\gppoint{gp mark 7}{(3.099,1.484)}
\gpcolor{rgb color={1.000,0.900,0.000}}
\gppoint{gp mark 7}{(3.227,1.484)}
\gpcolor{rgb color={1.000,0.975,0.000}}
\gppoint{gp mark 7}{(3.355,1.432)}
\gpcolor{rgb color={0.950,1.000,0.000}}
\gppoint{gp mark 7}{(3.483,1.754)}
\gpcolor{rgb color={0.875,1.000,0.000}}
\gppoint{gp mark 7}{(3.612,1.754)}
\gpcolor{rgb color={0.800,1.000,0.000}}
\gppoint{gp mark 7}{(3.740,2.247)}
\gpcolor{rgb color={0.725,1.000,0.000}}
\gppoint{gp mark 7}{(3.868,1.943)}
\gpcolor{rgb color={0.650,1.000,0.000}}
\gppoint{gp mark 7}{(3.996,1.229)}
\gpcolor{rgb color={0.575,1.000,0.000}}
\gppoint{gp mark 7}{(4.125,1.623)}
\gpcolor{rgb color={0.500,1.000,0.000}}
\gppoint{gp mark 7}{(4.253,2.019)}
\gpcolor{rgb color={0.425,1.000,0.000}}
\gppoint{gp mark 7}{(4.381,1.719)}
\gpcolor{rgb color={0.350,1.000,0.000}}
\gppoint{gp mark 7}{(4.509,1.701)}
\gpcolor{rgb color={0.275,1.000,0.000}}
\gppoint{gp mark 7}{(4.637,1.894)}
\gpcolor{rgb color={0.200,1.000,0.000}}
\gppoint{gp mark 7}{(4.766,2.245)}
\gpcolor{rgb color={0.125,1.000,0.000}}
\gppoint{gp mark 7}{(4.894,1.970)}
\gpcolor{rgb color={0.050,1.000,0.000}}
\gppoint{gp mark 7}{(5.022,1.758)}
\gpcolor{rgb color={0.000,1.000,0.025}}
\gppoint{gp mark 7}{(5.150,3.245)}
\gpcolor{rgb color={0.000,1.000,0.100}}
\gppoint{gp mark 7}{(5.279,2.668)}
\gpcolor{rgb color={0.000,1.000,0.175}}
\gppoint{gp mark 7}{(5.407,3.811)}
\gpcolor{rgb color={0.000,1.000,0.250}}
\gppoint{gp mark 7}{(5.535,2.588)}
\gpcolor{rgb color={0.000,1.000,0.325}}
\gppoint{gp mark 7}{(5.663,4.402)}
\gpcolor{rgb color={0.000,1.000,0.400}}
\gppoint{gp mark 7}{(5.792,4.097)}
\gpcolor{rgb color={0.000,1.000,0.475}}
\gppoint{gp mark 7}{(5.920,3.061)}
\gpcolor{rgb color={0.000,1.000,0.550}}
\gppoint{gp mark 7}{(6.048,2.683)}
\gpcolor{rgb color={0.000,1.000,0.625}}
\gppoint{gp mark 7}{(6.176,1.141)}
\gpcolor{rgb color={0.000,1.000,0.700}}
\gppoint{gp mark 7}{(6.305,1.340)}
\gpcolor{rgb color={0.000,1.000,0.775}}
\gppoint{gp mark 7}{(6.433,1.805)}
\gpcolor{rgb color={0.000,1.000,0.850}}
\gppoint{gp mark 7}{(6.561,2.264)}
\gpcolor{rgb color={0.000,1.000,0.925}}
\gppoint{gp mark 7}{(6.689,2.727)}
\gpcolor{rgb color={0.000,1.000,1.000}}
\gppoint{gp mark 7}{(6.818,3.186)}
\gpcolor{rgb color={0.000,0.925,1.000}}
\gppoint{gp mark 7}{(6.946,3.644)}
\gpcolor{rgb color={0.000,0.850,1.000}}
\gppoint{gp mark 7}{(7.074,4.101)}
\gpcolor{rgb color={0.000,0.775,1.000}}
\gppoint{gp mark 7}{(7.202,4.584)}
\gpcolor{rgb color={0.000,0.700,1.000}}
\gppoint{gp mark 7}{(7.330,5.076)}
\gpcolor{rgb color={0.000,0.625,1.000}}
\gppoint{gp mark 7}{(7.459,5.076)}
\gpcolor{rgb color={0.000,0.550,1.000}}
\gppoint{gp mark 7}{(7.587,5.076)}
\gpcolor{rgb color={0.000,0.475,1.000}}
\gppoint{gp mark 7}{(7.715,5.076)}
\gpcolor{rgb color={0.000,0.400,1.000}}
\gppoint{gp mark 7}{(7.843,7.617)}
\gpcolor{rgb color={0.000,0.325,1.000}}
\gppoint{gp mark 7}{(7.972,7.617)}
\gpcolor{rgb color={0.000,0.250,1.000}}
\gppoint{gp mark 7}{(8.100,7.617)}
\gpcolor{rgb color={0.000,0.175,1.000}}
\gppoint{gp mark 7}{(8.228,7.891)}
\gpcolor{rgb color={0.000,0.100,1.000}}
\gppoint{gp mark 7}{(8.356,2.339)}
\gpcolor{rgb color={0.000,0.025,1.000}}
\gppoint{gp mark 7}{(8.485,2.339)}
\gpcolor{rgb color={0.050,0.000,1.000}}
\gppoint{gp mark 7}{(8.613,3.638)}
\gpcolor{rgb color={0.125,0.000,1.000}}
\gppoint{gp mark 7}{(8.741,3.638)}
\gpcolor{rgb color={0.200,0.000,1.000}}
\gppoint{gp mark 7}{(8.869,3.638)}
\gpcolor{rgb color={0.275,0.000,1.000}}
\gppoint{gp mark 7}{(8.998,2.200)}
\gpcolor{rgb color={0.350,0.000,1.000}}
\gppoint{gp mark 7}{(9.126,1.854)}
\gpcolor{rgb color={0.425,0.000,1.000}}
\gppoint{gp mark 7}{(9.254,1.714)}
\gpcolor{rgb color={0.500,0.000,1.000}}
\gppoint{gp mark 7}{(9.382,1.580)}
\gpcolor{rgb color={0.575,0.000,1.000}}
\gppoint{gp mark 7}{(9.510,1.058)}
\gpcolor{rgb color={0.650,0.000,1.000}}
\gppoint{gp mark 7}{(9.639,1.236)}
\gpcolor{rgb color={0.725,0.000,1.000}}
\gppoint{gp mark 7}{(9.767,1.431)}
\gpcolor{rgb color={0.800,0.000,1.000}}
\gppoint{gp mark 7}{(9.895,1.616)}
\gpcolor{rgb color={0.875,0.000,1.000}}
\gppoint{gp mark 7}{(10.023,5.336)}
\gpcolor{rgb color={0.950,0.000,1.000}}
\gppoint{gp mark 7}{(10.152,5.159)}
\gpcolor{rgb color={1.000,0.000,0.975}}
\gppoint{gp mark 7}{(10.280,4.982)}
\gpcolor{rgb color={1.000,0.000,0.900}}
\gppoint{gp mark 7}{(10.408,4.805)}
\gpcolor{rgb color={1.000,0.000,0.825}}
\gppoint{gp mark 7}{(10.536,1.137)}
\gpcolor{rgb color={1.000,0.000,0.750}}
\gppoint{gp mark 7}{(10.665,1.545)}
\gpcolor{rgb color={1.000,0.000,0.675}}
\gppoint{gp mark 7}{(10.793,2.000)}
\gpcolor{rgb color={1.000,0.000,0.600}}
\gppoint{gp mark 7}{(10.921,2.459)}
\gpcolor{rgb color={1.000,0.000,0.525}}
\gppoint{gp mark 7}{(11.049,3.818)}
\gpcolor{rgb color={1.000,0.000,0.450}}
\gppoint{gp mark 7}{(11.178,3.493)}
\gpcolor{rgb color={1.000,0.000,0.375}}
\gppoint{gp mark 7}{(11.306,3.204)}
\gpcolor{rgb color={1.000,0.000,0.300}}
\gppoint{gp mark 7}{(11.434,2.966)}
\gpcolor{rgb color={1.000,0.000,0.225}}
\gppoint{gp mark 7}{(11.562,1.262)}
\gpcolor{rgb color={1.000,0.000,0.150}}
\gppoint{gp mark 7}{(11.691,1.936)}
\gpcolor{rgb color={1.000,0.000,0.075}}
\gppoint{gp mark 7}{(11.819,1.936)}
\gpcolor{rgb color={1.000,0.000,0.000}}
\gppoint{gp mark 7}{(11.947,2.217)}
\gpcolor{color=gp lt color border}
\gpsetlinetype{gp lt border}
\draw[gp path] (1.688,8.381)--(1.688,0.985)--(11.947,0.985)--(11.947,8.381)--cycle;
%% coordinates of the plot area
\gpdefrectangularnode{gp plot 1}{\pgfpoint{1.688cm}{0.985cm}}{\pgfpoint{11.947cm}{8.381cm}}
\end{tikzpicture}
%% gnuplot variables

%     }
%     \caption{Err.}
%     \label{fig:ball_kalman_error_tikz}
% \end{figure}

% \begin{figure}[htb]
%     \centering
%     \begin{subfigure}[b]{0.49\textwidth}
%         \resizebox{\columnwidth}{!}{%
%             \begin{tikzpicture}[gnuplot]
%% generated with GNUPLOT 4.6p4 (Lua 5.1; terminal rev. 99, script rev. 100)
%% Wed 27 May 2015 02:02:52 AM CEST
\path (0.000,0.000) rectangle (12.500,8.750);
\gpcolor{color=gp lt color border}
\gpsetlinetype{gp lt border}
\gpsetlinewidth{1.00}
\draw[gp path] (1.504,0.985)--(1.684,0.985);
\draw[gp path] (11.947,0.985)--(11.767,0.985);
\node[gp node right,font={\fontsize{9pt}{10.8pt}\selectfont}] at (1.320,0.985) {-2};
\draw[gp path] (1.504,1.725)--(1.684,1.725);
\draw[gp path] (11.947,1.725)--(11.767,1.725);
\node[gp node right,font={\fontsize{9pt}{10.8pt}\selectfont}] at (1.320,1.725) {-1.5};
\draw[gp path] (1.504,2.464)--(1.684,2.464);
\draw[gp path] (11.947,2.464)--(11.767,2.464);
\node[gp node right,font={\fontsize{9pt}{10.8pt}\selectfont}] at (1.320,2.464) {-1};
\draw[gp path] (1.504,3.204)--(1.684,3.204);
\draw[gp path] (11.947,3.204)--(11.767,3.204);
\node[gp node right,font={\fontsize{9pt}{10.8pt}\selectfont}] at (1.320,3.204) {-0.5};
\draw[gp path] (1.504,3.943)--(1.684,3.943);
\draw[gp path] (11.947,3.943)--(11.767,3.943);
\node[gp node right,font={\fontsize{9pt}{10.8pt}\selectfont}] at (1.320,3.943) { 0};
\draw[gp path] (1.504,4.683)--(1.684,4.683);
\draw[gp path] (11.947,4.683)--(11.767,4.683);
\node[gp node right,font={\fontsize{9pt}{10.8pt}\selectfont}] at (1.320,4.683) { 0.5};
\draw[gp path] (1.504,5.423)--(1.684,5.423);
\draw[gp path] (11.947,5.423)--(11.767,5.423);
\node[gp node right,font={\fontsize{9pt}{10.8pt}\selectfont}] at (1.320,5.423) { 1};
\draw[gp path] (1.504,6.162)--(1.684,6.162);
\draw[gp path] (11.947,6.162)--(11.767,6.162);
\node[gp node right,font={\fontsize{9pt}{10.8pt}\selectfont}] at (1.320,6.162) { 1.5};
\draw[gp path] (1.504,6.902)--(1.684,6.902);
\draw[gp path] (11.947,6.902)--(11.767,6.902);
\node[gp node right,font={\fontsize{9pt}{10.8pt}\selectfont}] at (1.320,6.902) { 2};
\draw[gp path] (1.504,7.641)--(1.684,7.641);
\draw[gp path] (11.947,7.641)--(11.767,7.641);
\node[gp node right,font={\fontsize{9pt}{10.8pt}\selectfont}] at (1.320,7.641) { 2.5};
\draw[gp path] (1.504,8.381)--(1.684,8.381);
\draw[gp path] (11.947,8.381)--(11.767,8.381);
\node[gp node right,font={\fontsize{9pt}{10.8pt}\selectfont}] at (1.320,8.381) { 3};
\draw[gp path] (1.504,0.985)--(1.504,1.165);
\draw[gp path] (1.504,8.381)--(1.504,8.201);
\node[gp node center,font={\fontsize{9pt}{10.8pt}\selectfont}] at (1.504,0.677) { 0};
\draw[gp path] (2.548,0.985)--(2.548,1.165);
\draw[gp path] (2.548,8.381)--(2.548,8.201);
\node[gp node center,font={\fontsize{9pt}{10.8pt}\selectfont}] at (2.548,0.677) { 50};
\draw[gp path] (3.593,0.985)--(3.593,1.165);
\draw[gp path] (3.593,8.381)--(3.593,8.201);
\node[gp node center,font={\fontsize{9pt}{10.8pt}\selectfont}] at (3.593,0.677) { 100};
\draw[gp path] (4.637,0.985)--(4.637,1.165);
\draw[gp path] (4.637,8.381)--(4.637,8.201);
\node[gp node center,font={\fontsize{9pt}{10.8pt}\selectfont}] at (4.637,0.677) { 150};
\draw[gp path] (5.681,0.985)--(5.681,1.165);
\draw[gp path] (5.681,8.381)--(5.681,8.201);
\node[gp node center,font={\fontsize{9pt}{10.8pt}\selectfont}] at (5.681,0.677) { 200};
\draw[gp path] (6.726,0.985)--(6.726,1.165);
\draw[gp path] (6.726,8.381)--(6.726,8.201);
\node[gp node center,font={\fontsize{9pt}{10.8pt}\selectfont}] at (6.726,0.677) { 250};
\draw[gp path] (7.770,0.985)--(7.770,1.165);
\draw[gp path] (7.770,8.381)--(7.770,8.201);
\node[gp node center,font={\fontsize{9pt}{10.8pt}\selectfont}] at (7.770,0.677) { 300};
\draw[gp path] (8.814,0.985)--(8.814,1.165);
\draw[gp path] (8.814,8.381)--(8.814,8.201);
\node[gp node center,font={\fontsize{9pt}{10.8pt}\selectfont}] at (8.814,0.677) { 350};
\draw[gp path] (9.858,0.985)--(9.858,1.165);
\draw[gp path] (9.858,8.381)--(9.858,8.201);
\node[gp node center,font={\fontsize{9pt}{10.8pt}\selectfont}] at (9.858,0.677) { 400};
\draw[gp path] (10.903,0.985)--(10.903,1.165);
\draw[gp path] (10.903,8.381)--(10.903,8.201);
\node[gp node center,font={\fontsize{9pt}{10.8pt}\selectfont}] at (10.903,0.677) { 450};
\draw[gp path] (11.947,0.985)--(11.947,1.165);
\draw[gp path] (11.947,8.381)--(11.947,8.201);
\node[gp node center,font={\fontsize{9pt}{10.8pt}\selectfont}] at (11.947,0.677) { 500};
\draw[gp path] (1.504,8.381)--(1.504,0.985)--(11.947,0.985)--(11.947,8.381)--cycle;
\node[gp node center,rotate=-270,font={\fontsize{9pt}{10.8pt}\selectfont}] at (0.246,4.683) {angle (rad)};
\node[gp node center,font={\fontsize{9pt}{10.8pt}\selectfont}] at (6.725,0.215) {time (s)};
\node[gp node right,font={\fontsize{9pt}{10.8pt}\selectfont}] at (10.479,8.047) {q-real-0};
\gpcolor{color=gp lt color 0}
\gpsetlinetype{gp lt plot 0}
\draw[gp path] (10.663,8.047)--(11.579,8.047);
\draw[gp path] (1.504,5.666)--(1.525,5.666)--(1.546,5.666)--(1.567,5.666)--(1.588,5.666)%
  --(1.608,5.666)--(1.629,5.666)--(1.650,5.666)--(1.671,5.666)--(1.692,5.666)--(1.713,5.666)%
  --(1.734,5.666)--(1.755,6.056)--(1.776,6.056)--(1.796,6.056)--(1.817,6.056)--(1.838,6.056)%
  --(1.859,6.056)--(1.880,6.056)--(1.901,6.056)--(1.922,6.056)--(1.943,6.056)--(1.963,6.056)%
  --(1.984,6.056)--(2.005,6.056)--(2.026,6.220)--(2.047,6.220)--(2.068,6.220)--(2.089,6.220)%
  --(2.110,6.220)--(2.131,6.220)--(2.151,6.220)--(2.172,6.220)--(2.193,6.220)--(2.214,6.220)%
  --(2.235,6.220)--(2.256,6.220)--(2.277,6.220)--(2.298,6.220)--(2.319,6.235)--(2.339,6.235)%
  --(2.360,6.235)--(2.381,6.235)--(2.402,6.235)--(2.423,6.235)--(2.444,6.235)--(2.465,6.235)%
  --(2.486,6.235)--(2.507,6.235)--(2.527,6.235)--(2.548,6.235)--(2.569,6.235)--(2.590,6.235)%
  --(2.611,6.235)--(2.632,6.235)--(2.653,6.235)--(2.674,6.235)--(2.695,6.235)--(2.715,6.235)%
  --(2.736,6.235)--(2.757,5.881)--(2.778,5.881)--(2.799,5.881)--(2.820,5.881)--(2.841,5.881)%
  --(2.862,5.881)--(2.882,5.881)--(2.903,5.881)--(2.924,5.881)--(2.945,5.881)--(2.966,5.881)%
  --(2.987,5.881)--(3.008,5.881)--(3.029,5.881)--(3.050,5.881)--(3.070,5.881)--(3.091,5.881)%
  --(3.112,5.881)--(3.133,5.881)--(3.154,5.881)--(3.175,5.881)--(3.196,5.881)--(3.217,5.881)%
  --(3.238,5.881)--(3.258,5.881)--(3.279,5.881)--(3.300,5.881)--(3.321,5.881)--(3.342,5.881)%
  --(3.363,5.881)--(3.384,5.881)--(3.405,5.881)--(3.426,5.881)--(3.446,5.881)--(3.467,5.881)%
  --(3.488,5.881)--(3.509,5.881)--(3.530,5.881)--(3.551,5.881)--(3.572,5.881)--(3.593,5.881)%
  --(3.613,5.881)--(3.634,5.881)--(3.655,5.881)--(3.676,5.881)--(3.697,5.881)--(3.718,5.881)%
  --(3.739,5.881)--(3.760,5.881)--(3.781,5.881)--(3.801,5.881)--(3.822,5.881)--(3.843,5.881)%
  --(3.864,5.881)--(3.885,5.881)--(3.906,5.881)--(3.927,5.881)--(3.948,5.881)--(3.969,5.881)%
  --(3.989,5.881)--(4.010,5.881)--(4.031,5.881)--(4.052,5.881)--(4.073,5.881)--(4.094,5.881)%
  --(4.115,5.881)--(4.136,5.881)--(4.157,5.881)--(4.177,5.881)--(4.198,5.881)--(4.219,5.881)%
  --(4.240,5.881)--(4.261,5.881)--(4.282,5.881)--(4.303,5.881)--(4.324,5.881)--(4.344,5.881)%
  --(4.365,5.881)--(4.386,5.881)--(4.407,5.881)--(4.428,5.881)--(4.449,5.881)--(4.470,5.881)%
  --(4.491,5.881)--(4.512,5.881)--(4.532,5.881)--(4.553,5.881)--(4.574,5.881)--(4.595,5.881)%
  --(4.616,5.881)--(4.637,5.881)--(4.658,5.881)--(4.679,5.881)--(4.700,5.881)--(4.720,5.881)%
  --(4.741,5.881)--(4.762,5.881)--(4.783,5.881)--(4.804,5.881)--(4.825,5.881)--(4.846,5.881)%
  --(4.867,5.881)--(4.888,5.881)--(4.908,5.881)--(4.929,5.881)--(4.950,5.881)--(4.971,5.881)%
  --(4.992,5.881)--(5.013,5.881)--(5.034,5.881)--(5.055,5.881)--(5.076,5.881)--(5.096,5.881)%
  --(5.117,5.881)--(5.138,5.881)--(5.159,5.881)--(5.180,5.881)--(5.201,5.881)--(5.222,5.881)%
  --(5.243,5.881)--(5.263,5.881)--(5.284,5.827)--(5.305,5.827)--(5.326,5.827)--(5.347,5.827)%
  --(5.368,5.827)--(5.389,5.827)--(5.410,5.827)--(5.431,5.827)--(5.451,5.827)--(5.472,5.827)%
  --(5.493,5.827)--(5.514,5.827)--(5.535,5.827)--(5.556,5.827)--(5.577,5.827)--(5.598,5.827)%
  --(5.619,5.827)--(5.639,5.827)--(5.660,5.827)--(5.681,5.827)--(5.702,5.827)--(5.723,5.594)%
  --(5.744,5.594)--(5.765,5.594)--(5.786,5.594)--(5.807,5.594)--(5.827,5.594)--(5.848,5.594)%
  --(5.869,5.594)--(5.890,5.594)--(5.911,5.594)--(5.932,5.594)--(5.953,5.594)--(5.974,5.594)%
  --(5.994,5.594)--(6.015,5.594)--(6.036,5.594)--(6.057,5.594)--(6.078,5.411)--(6.099,5.411)%
  --(6.120,5.411)--(6.141,5.411)--(6.162,5.411)--(6.182,5.411)--(6.203,5.411)--(6.224,5.411)%
  --(6.245,5.411)--(6.266,5.411)--(6.287,5.411)--(6.308,5.411)--(6.329,5.411)--(6.350,5.411)%
  --(6.370,5.411)--(6.391,6.019)--(6.412,6.019)--(6.433,6.019)--(6.454,6.019)--(6.475,6.019)%
  --(6.496,6.019)--(6.517,6.019)--(6.538,6.019)--(6.558,6.019)--(6.579,6.019)--(6.600,6.019)%
  --(6.621,6.019)--(6.642,6.019)--(6.663,6.019)--(6.684,6.019)--(6.705,6.019)--(6.726,6.019)%
  --(6.746,6.019)--(6.767,6.019)--(6.788,6.019)--(6.809,6.019)--(6.830,6.019)--(6.851,6.019)%
  --(6.872,6.019)--(6.893,6.019)--(6.913,6.019)--(6.934,6.019)--(6.955,6.019)--(6.976,6.019)%
  --(6.997,6.019)--(7.018,6.019)--(7.039,6.019)--(7.060,6.019)--(7.081,6.019)--(7.101,6.019)%
  --(7.122,6.019)--(7.143,6.019)--(7.164,6.019)--(7.185,6.019)--(7.206,6.019)--(7.227,6.019)%
  --(7.248,6.019)--(7.269,6.019)--(7.289,6.019)--(7.310,6.019)--(7.331,6.019)--(7.352,6.019)%
  --(7.373,6.019)--(7.394,6.019)--(7.415,6.019)--(7.436,6.019)--(7.457,6.019)--(7.477,6.019)%
  --(7.498,6.019)--(7.519,6.019)--(7.540,6.019)--(7.561,6.019)--(7.582,6.019)--(7.603,6.019)%
  --(7.624,6.019)--(7.644,6.019)--(7.665,6.019)--(7.686,6.019)--(7.707,6.019)--(7.728,6.019)%
  --(7.749,6.019)--(7.770,6.019)--(7.791,6.019)--(7.812,6.019)--(7.832,6.019)--(7.853,6.019)%
  --(7.874,6.019)--(7.895,6.019)--(7.916,6.019)--(7.937,6.019)--(7.958,6.019)--(7.979,6.019)%
  --(8.000,6.019)--(8.020,6.019)--(8.041,6.019)--(8.062,6.019)--(8.083,6.019)--(8.104,6.019)%
  --(8.125,6.019)--(8.146,6.019)--(8.167,6.019)--(8.188,6.019)--(8.208,6.019)--(8.229,6.019)%
  --(8.250,6.019)--(8.271,6.019)--(8.292,6.019)--(8.313,6.019)--(8.334,6.019)--(8.355,6.019)%
  --(8.375,6.019)--(8.396,6.019)--(8.417,6.019)--(8.438,6.019)--(8.459,6.019)--(8.480,6.019)%
  --(8.501,6.019)--(8.522,6.019)--(8.543,6.019)--(8.563,6.019)--(8.584,6.019)--(8.605,6.019)%
  --(8.626,6.019)--(8.647,6.019)--(8.668,6.019)--(8.689,6.019)--(8.710,6.019)--(8.731,6.019)%
  --(8.751,6.019)--(8.772,6.019)--(8.793,6.019)--(8.814,6.019)--(8.835,6.019)--(8.856,6.019)%
  --(8.877,6.019)--(8.898,6.019)--(8.919,8.104)--(8.939,8.104)--(8.960,8.104)--(8.981,8.104)%
  --(9.002,8.104)--(9.023,8.104)--(9.044,8.104)--(9.065,8.104)--(9.086,8.104)--(9.107,8.104)%
  --(9.127,8.104)--(9.148,8.104)--(9.169,8.104)--(9.190,8.104)--(9.211,8.104)--(9.232,8.104)%
  --(9.253,8.104)--(9.274,8.104)--(9.294,8.104)--(9.315,8.104)--(9.336,8.104)--(9.357,8.104)%
  --(9.378,8.104)--(9.399,8.104)--(9.420,8.104)--(9.441,8.104)--(9.462,8.104)--(9.482,8.104)%
  --(9.503,8.104)--(9.524,8.104)--(9.545,8.104)--(9.566,8.104)--(9.587,8.104)--(9.608,8.104)%
  --(9.629,8.104)--(9.650,8.104)--(9.670,8.104)--(9.691,8.104)--(9.712,8.104)--(9.733,8.104)%
  --(9.754,8.104)--(9.775,8.104)--(9.796,8.104)--(9.817,8.104)--(9.838,8.104)--(9.858,8.104)%
  --(9.879,8.104)--(9.900,8.104)--(9.921,8.104)--(9.942,8.104)--(9.963,8.104)--(9.984,8.104)%
  --(10.005,8.104)--(10.025,8.104)--(10.046,8.104)--(10.067,8.104)--(10.088,8.104)--(10.109,8.104)%
  --(10.130,8.104)--(10.151,8.104)--(10.172,8.104)--(10.193,8.104)--(10.213,8.104)--(10.234,8.104)%
  --(10.255,8.104)--(10.276,8.104)--(10.297,8.104)--(10.318,8.104)--(10.339,8.104)--(10.360,8.104)%
  --(10.381,8.104)--(10.401,8.104)--(10.422,8.104)--(10.443,8.104)--(10.464,8.104)--(10.485,8.104)%
  --(10.506,8.104)--(10.527,8.104)--(10.548,8.104)--(10.569,8.104)--(10.589,8.104)--(10.610,8.104)%
  --(10.631,8.104)--(10.652,8.104)--(10.673,8.104)--(10.694,8.104)--(10.715,8.104)--(10.736,8.104)%
  --(10.756,8.104)--(10.777,8.104)--(10.798,8.104)--(10.819,8.104)--(10.840,8.104)--(10.861,8.104)%
  --(10.882,8.104)--(10.903,8.104)--(10.924,8.104)--(10.944,8.104)--(10.965,8.104)--(10.986,8.104)%
  --(11.007,8.104)--(11.028,8.104);
\gpcolor{color=gp lt color border}
\node[gp node right,font={\fontsize{9pt}{10.8pt}\selectfont}] at (10.479,7.739) {q-desired-0};
\gpcolor{color=gp lt color 1}
\gpsetlinetype{gp lt plot 1}
\draw[gp path] (10.663,7.739)--(11.579,7.739);
\draw[gp path] (1.504,5.668)--(1.525,5.668)--(1.546,5.668)--(1.567,5.668)--(1.588,5.668)%
  --(1.608,5.668)--(1.629,5.668)--(1.650,5.668)--(1.671,5.668)--(1.692,5.668)--(1.713,5.668)%
  --(1.734,6.205)--(1.755,6.205)--(1.776,6.205)--(1.796,6.205)--(1.817,6.205)--(1.838,6.205)%
  --(1.859,6.205)--(1.880,6.205)--(1.901,6.205)--(1.922,6.205)--(1.943,6.205)--(1.963,6.205)%
  --(1.984,6.205)--(2.005,6.221)--(2.026,6.221)--(2.047,6.221)--(2.068,6.221)--(2.089,6.221)%
  --(2.110,6.221)--(2.131,6.221)--(2.151,6.221)--(2.172,6.221)--(2.193,6.221)--(2.214,6.221)%
  --(2.235,6.221)--(2.256,6.221)--(2.277,6.221)--(2.298,6.221)--(2.319,6.237)--(2.339,6.237)%
  --(2.360,6.237)--(2.381,6.237)--(2.402,6.237)--(2.423,6.237)--(2.444,6.237)--(2.465,6.237)%
  --(2.486,6.237)--(2.507,6.237)--(2.527,6.237)--(2.548,6.237)--(2.569,6.237)--(2.590,6.237)%
  --(2.611,6.237)--(2.632,6.237)--(2.653,6.237)--(2.674,6.237)--(2.695,6.237)--(2.715,6.237)%
  --(2.736,6.237)--(2.757,5.611)--(2.778,5.611)--(2.799,5.611)--(2.820,5.611)--(2.841,5.611)%
  --(2.862,5.611)--(2.882,5.611)--(2.903,5.611)--(2.924,5.611)--(2.945,5.611)--(2.966,5.611)%
  --(2.987,5.611)--(3.008,5.611)--(3.029,5.611)--(3.050,5.611)--(3.070,5.611)--(3.091,5.611)%
  --(3.112,5.611)--(3.133,5.611)--(3.154,5.611)--(3.175,5.611)--(3.196,5.611)--(3.217,5.611)%
  --(3.238,5.611)--(3.258,5.611)--(3.279,5.611)--(3.300,5.611)--(3.321,5.611)--(3.342,5.611)%
  --(3.363,5.611)--(3.384,5.611)--(3.405,5.611)--(3.426,5.611)--(3.446,5.611)--(3.467,5.611)%
  --(3.488,5.611)--(3.509,5.611)--(3.530,5.611)--(3.551,5.611)--(3.572,5.611)--(3.593,5.611)%
  --(3.613,5.611)--(3.634,5.611)--(3.655,5.611)--(3.676,5.611)--(3.697,5.611)--(3.718,5.611)%
  --(3.739,5.611)--(3.760,5.611)--(3.781,5.611)--(3.801,5.611)--(3.822,5.611)--(3.843,5.611)%
  --(3.864,5.611)--(3.885,5.611)--(3.906,5.611)--(3.927,5.611)--(3.948,5.611)--(3.969,5.611)%
  --(3.989,5.611)--(4.010,5.611)--(4.031,5.611)--(4.052,5.611)--(4.073,5.611)--(4.094,5.611)%
  --(4.115,5.611)--(4.136,5.611)--(4.157,5.611)--(4.177,5.611)--(4.198,5.611)--(4.219,5.611)%
  --(4.240,5.611)--(4.261,5.611)--(4.282,5.611)--(4.303,5.611)--(4.324,5.611)--(4.344,5.611)%
  --(4.365,5.611)--(4.386,5.611)--(4.407,5.611)--(4.428,5.611)--(4.449,5.611)--(4.470,5.611)%
  --(4.491,5.611)--(4.512,5.611)--(4.532,5.611)--(4.553,5.611)--(4.574,5.611)--(4.595,5.611)%
  --(4.616,5.611)--(4.637,5.611)--(4.658,5.611)--(4.679,5.611)--(4.700,5.611)--(4.720,5.611)%
  --(4.741,5.611)--(4.762,5.611)--(4.783,5.611)--(4.804,5.611)--(4.825,5.611)--(4.846,5.611)%
  --(4.867,5.611)--(4.888,5.611)--(4.908,5.611)--(4.929,5.611)--(4.950,5.611)--(4.971,5.611)%
  --(4.992,5.611)--(5.013,5.611)--(5.034,5.611)--(5.055,5.611)--(5.076,5.611)--(5.096,5.611)%
  --(5.117,5.611)--(5.138,5.611)--(5.159,5.611)--(5.180,5.611)--(5.201,5.611)--(5.222,5.611)%
  --(5.243,5.611)--(5.263,5.857)--(5.284,5.857)--(5.305,5.857)--(5.326,5.857)--(5.347,5.857)%
  --(5.368,5.857)--(5.389,5.857)--(5.410,5.857)--(5.431,5.857)--(5.451,5.857)--(5.472,5.857)%
  --(5.493,5.857)--(5.514,5.857)--(5.535,5.857)--(5.556,5.857)--(5.577,5.857)--(5.598,5.857)%
  --(5.619,5.857)--(5.639,5.857)--(5.660,5.857)--(5.681,5.857)--(5.702,5.516)--(5.723,5.516)%
  --(5.744,5.516)--(5.765,5.516)--(5.786,5.516)--(5.807,5.516)--(5.827,5.516)--(5.848,5.516)%
  --(5.869,5.516)--(5.890,5.516)--(5.911,5.516)--(5.932,5.516)--(5.953,5.516)--(5.974,5.516)%
  --(5.994,5.516)--(6.015,5.516)--(6.036,5.516)--(6.057,5.405)--(6.078,5.405)--(6.099,5.405)%
  --(6.120,5.405)--(6.141,5.405)--(6.162,5.405)--(6.182,5.405)--(6.203,5.405)--(6.224,5.405)%
  --(6.245,5.405)--(6.266,5.405)--(6.287,5.405)--(6.308,5.405)--(6.329,5.405)--(6.350,5.405)%
  --(6.370,5.405)--(6.391,8.243)--(6.412,8.243)--(6.433,8.243)--(6.454,8.243)--(6.475,8.243)%
  --(6.496,8.243)--(6.517,8.243)--(6.538,8.243)--(6.558,8.243)--(6.579,8.243)--(6.600,8.243)%
  --(6.621,8.243)--(6.642,8.243)--(6.663,8.243)--(6.684,8.243)--(6.705,8.243)--(6.726,8.243)%
  --(6.746,8.243)--(6.767,8.243)--(6.788,8.243)--(6.809,8.243)--(6.830,8.243)--(6.851,8.243)%
  --(6.872,8.243)--(6.893,8.243)--(6.913,8.243)--(6.934,8.243)--(6.955,8.243)--(6.976,8.243)%
  --(6.997,8.243)--(7.018,8.243)--(7.039,8.243)--(7.060,8.243)--(7.081,8.243)--(7.101,8.243)%
  --(7.122,8.243)--(7.143,8.243)--(7.164,8.243)--(7.185,8.243)--(7.206,8.243)--(7.227,8.243)%
  --(7.248,8.243)--(7.269,8.243)--(7.289,8.243)--(7.310,8.243)--(7.331,8.243)--(7.352,8.243)%
  --(7.373,8.243)--(7.394,8.243)--(7.415,8.243)--(7.436,8.243)--(7.457,8.243)--(7.477,8.243)%
  --(7.498,8.243)--(7.519,8.243)--(7.540,8.243)--(7.561,8.243)--(7.582,8.243)--(7.603,8.243)%
  --(7.624,8.243)--(7.644,8.243)--(7.665,8.243)--(7.686,8.243)--(7.707,8.243)--(7.728,8.243)%
  --(7.749,8.243)--(7.770,8.243)--(7.791,8.243)--(7.812,8.243)--(7.832,8.243)--(7.853,8.243)%
  --(7.874,8.243)--(7.895,8.243)--(7.916,8.243)--(7.937,8.243)--(7.958,8.243)--(7.979,8.243)%
  --(8.000,8.243)--(8.020,8.243)--(8.041,8.243)--(8.062,8.243)--(8.083,8.243)--(8.104,8.243)%
  --(8.125,8.243)--(8.146,8.243)--(8.167,8.243)--(8.188,8.243)--(8.208,8.243)--(8.229,8.243)%
  --(8.250,8.243)--(8.271,8.243)--(8.292,8.243)--(8.313,8.243)--(8.334,8.243)--(8.355,8.243)%
  --(8.375,8.243)--(8.396,8.243)--(8.417,8.243)--(8.438,8.243)--(8.459,8.243)--(8.480,8.243)%
  --(8.501,8.243)--(8.522,8.243)--(8.543,8.243)--(8.563,8.243)--(8.584,8.243)--(8.605,8.243)%
  --(8.626,8.243)--(8.647,8.243)--(8.668,8.243)--(8.689,8.243)--(8.710,8.243)--(8.731,8.243)%
  --(8.751,8.243)--(8.772,8.243)--(8.793,8.243)--(8.814,8.243)--(8.835,8.243)--(8.856,8.243)%
  --(8.877,8.243)--(8.898,8.243)--(8.919,8.093)--(8.939,8.093)--(8.960,8.093)--(8.981,8.093)%
  --(9.002,8.093)--(9.023,8.093)--(9.044,8.093)--(9.065,8.093)--(9.086,8.093)--(9.107,8.093)%
  --(9.127,8.093)--(9.148,8.093)--(9.169,8.093)--(9.190,8.093)--(9.211,8.093)--(9.232,8.093)%
  --(9.253,8.093)--(9.274,8.093)--(9.294,8.093)--(9.315,8.093)--(9.336,8.093)--(9.357,8.093)%
  --(9.378,8.093)--(9.399,8.093)--(9.420,8.093)--(9.441,8.093)--(9.462,8.093)--(9.482,8.093)%
  --(9.503,8.093)--(9.524,8.093)--(9.545,8.093)--(9.566,8.093)--(9.587,8.093)--(9.608,8.093)%
  --(9.629,8.093)--(9.650,8.093)--(9.670,8.093)--(9.691,8.093)--(9.712,8.093)--(9.733,8.093)%
  --(9.754,8.093)--(9.775,8.093)--(9.796,8.093)--(9.817,8.093)--(9.838,8.093)--(9.858,8.093)%
  --(9.879,8.093)--(9.900,8.093)--(9.921,8.093)--(9.942,8.093)--(9.963,8.093)--(9.984,8.093)%
  --(10.005,8.093)--(10.025,8.093)--(10.046,8.093)--(10.067,8.093)--(10.088,8.093)--(10.109,8.093)%
  --(10.130,8.093)--(10.151,8.093)--(10.172,8.093)--(10.193,8.093)--(10.213,8.093)--(10.234,8.093)%
  --(10.255,8.093)--(10.276,8.093)--(10.297,8.093)--(10.318,8.093)--(10.339,8.093)--(10.360,8.093)%
  --(10.381,8.093)--(10.401,8.093)--(10.422,8.093)--(10.443,8.093)--(10.464,8.093)--(10.485,8.093)%
  --(10.506,8.093)--(10.527,8.093)--(10.548,8.093)--(10.569,8.093)--(10.589,8.093)--(10.610,8.093)%
  --(10.631,8.093)--(10.652,8.093)--(10.673,8.093)--(10.694,8.093)--(10.715,8.093)--(10.736,8.093)%
  --(10.756,8.093)--(10.777,8.093)--(10.798,8.093)--(10.819,8.093)--(10.840,8.093)--(10.861,8.093)%
  --(10.882,8.093)--(10.903,8.093)--(10.924,8.093)--(10.944,8.093)--(10.965,8.093)--(10.986,8.093)%
  --(11.007,8.093)--(11.028,8.093);
\gpcolor{color=gp lt color border}
\node[gp node right,font={\fontsize{9pt}{10.8pt}\selectfont}] at (10.479,7.431) {q-real-1};
\gpcolor{color=gp lt color 2}
\gpsetlinetype{gp lt plot 2}
\draw[gp path] (10.663,7.431)--(11.579,7.431);
\draw[gp path] (1.504,2.570)--(1.525,2.570)--(1.546,2.570)--(1.567,2.570)--(1.588,2.570)%
  --(1.608,2.570)--(1.629,2.570)--(1.650,2.570)--(1.671,2.570)--(1.692,2.570)--(1.713,2.570)%
  --(1.734,2.570)--(1.755,3.050)--(1.776,3.050)--(1.796,3.050)--(1.817,3.050)--(1.838,3.050)%
  --(1.859,3.050)--(1.880,3.050)--(1.901,3.050)--(1.922,3.050)--(1.943,3.050)--(1.963,3.050)%
  --(1.984,3.050)--(2.005,3.050)--(2.026,3.543)--(2.047,3.543)--(2.068,3.543)--(2.089,3.543)%
  --(2.110,3.543)--(2.131,3.543)--(2.151,3.543)--(2.172,3.543)--(2.193,3.543)--(2.214,3.543)%
  --(2.235,3.543)--(2.256,3.543)--(2.277,3.543)--(2.298,3.543)--(2.319,3.795)--(2.339,3.795)%
  --(2.360,3.795)--(2.381,3.795)--(2.402,3.795)--(2.423,3.795)--(2.444,3.795)--(2.465,3.795)%
  --(2.486,3.795)--(2.507,3.795)--(2.527,3.795)--(2.548,3.795)--(2.569,3.795)--(2.590,3.795)%
  --(2.611,3.795)--(2.632,3.795)--(2.653,3.795)--(2.674,3.795)--(2.695,3.795)--(2.715,3.795)%
  --(2.736,3.795)--(2.757,4.259)--(2.778,4.259)--(2.799,4.259)--(2.820,4.259)--(2.841,4.259)%
  --(2.862,4.259)--(2.882,4.259)--(2.903,4.259)--(2.924,4.259)--(2.945,4.259)--(2.966,4.259)%
  --(2.987,4.259)--(3.008,4.259)--(3.029,4.259)--(3.050,4.259)--(3.070,4.259)--(3.091,4.259)%
  --(3.112,4.259)--(3.133,4.259)--(3.154,4.259)--(3.175,4.259)--(3.196,4.259)--(3.217,4.259)%
  --(3.238,4.259)--(3.258,4.259)--(3.279,4.259)--(3.300,4.259)--(3.321,4.259)--(3.342,4.259)%
  --(3.363,4.259)--(3.384,4.259)--(3.405,4.259)--(3.426,4.259)--(3.446,4.259)--(3.467,4.259)%
  --(3.488,4.259)--(3.509,4.259)--(3.530,4.259)--(3.551,4.259)--(3.572,4.259)--(3.593,4.259)%
  --(3.613,4.259)--(3.634,4.259)--(3.655,4.259)--(3.676,4.259)--(3.697,4.259)--(3.718,4.259)%
  --(3.739,4.259)--(3.760,4.259)--(3.781,4.259)--(3.801,4.259)--(3.822,4.259)--(3.843,4.259)%
  --(3.864,4.259)--(3.885,4.259)--(3.906,4.259)--(3.927,4.259)--(3.948,4.259)--(3.969,4.259)%
  --(3.989,4.259)--(4.010,4.259)--(4.031,4.259)--(4.052,4.259)--(4.073,4.259)--(4.094,4.259)%
  --(4.115,4.259)--(4.136,4.259)--(4.157,4.259)--(4.177,4.259)--(4.198,4.259)--(4.219,4.259)%
  --(4.240,4.259)--(4.261,4.259)--(4.282,4.259)--(4.303,4.259)--(4.324,4.259)--(4.344,4.259)%
  --(4.365,4.259)--(4.386,4.259)--(4.407,4.259)--(4.428,4.259)--(4.449,4.259)--(4.470,4.259)%
  --(4.491,4.259)--(4.512,4.259)--(4.532,4.259)--(4.553,4.259)--(4.574,4.259)--(4.595,4.259)%
  --(4.616,4.259)--(4.637,4.259)--(4.658,4.259)--(4.679,4.259)--(4.700,4.259)--(4.720,4.259)%
  --(4.741,4.259)--(4.762,4.259)--(4.783,4.259)--(4.804,4.259)--(4.825,4.259)--(4.846,4.259)%
  --(4.867,4.259)--(4.888,4.259)--(4.908,4.259)--(4.929,4.259)--(4.950,4.259)--(4.971,4.259)%
  --(4.992,4.259)--(5.013,4.259)--(5.034,4.259)--(5.055,4.259)--(5.076,4.259)--(5.096,4.259)%
  --(5.117,4.259)--(5.138,4.259)--(5.159,4.259)--(5.180,4.259)--(5.201,4.259)--(5.222,4.259)%
  --(5.243,4.259)--(5.263,4.259)--(5.284,4.503)--(5.305,4.503)--(5.326,4.503)--(5.347,4.503)%
  --(5.368,4.503)--(5.389,4.503)--(5.410,4.503)--(5.431,4.503)--(5.451,4.503)--(5.472,4.503)%
  --(5.493,4.503)--(5.514,4.503)--(5.535,4.503)--(5.556,4.503)--(5.577,4.503)--(5.598,4.503)%
  --(5.619,4.503)--(5.639,4.503)--(5.660,4.503)--(5.681,4.503)--(5.702,4.503)--(5.723,4.651)%
  --(5.744,4.651)--(5.765,4.651)--(5.786,4.651)--(5.807,4.651)--(5.827,4.651)--(5.848,4.651)%
  --(5.869,4.651)--(5.890,4.651)--(5.911,4.651)--(5.932,4.651)--(5.953,4.651)--(5.974,4.651)%
  --(5.994,4.651)--(6.015,4.651)--(6.036,4.651)--(6.057,4.651)--(6.078,4.523)--(6.099,4.523)%
  --(6.120,4.523)--(6.141,4.523)--(6.162,4.523)--(6.182,4.523)--(6.203,4.523)--(6.224,4.523)%
  --(6.245,4.523)--(6.266,4.523)--(6.287,4.523)--(6.308,4.523)--(6.329,4.523)--(6.350,4.523)%
  --(6.370,4.523)--(6.391,4.311)--(6.412,4.311)--(6.433,4.311)--(6.454,4.311)--(6.475,4.311)%
  --(6.496,4.311)--(6.517,4.311)--(6.538,4.311)--(6.558,4.311)--(6.579,4.311)--(6.600,4.311)%
  --(6.621,4.311)--(6.642,4.311)--(6.663,4.311)--(6.684,4.311)--(6.705,4.311)--(6.726,4.311)%
  --(6.746,4.311)--(6.767,4.311)--(6.788,4.311)--(6.809,4.311)--(6.830,4.311)--(6.851,4.311)%
  --(6.872,4.311)--(6.893,4.311)--(6.913,4.311)--(6.934,4.311)--(6.955,4.311)--(6.976,4.311)%
  --(6.997,4.311)--(7.018,4.311)--(7.039,4.311)--(7.060,4.311)--(7.081,4.311)--(7.101,4.311)%
  --(7.122,4.311)--(7.143,4.311)--(7.164,4.311)--(7.185,4.311)--(7.206,4.311)--(7.227,4.311)%
  --(7.248,4.311)--(7.269,4.311)--(7.289,4.311)--(7.310,4.311)--(7.331,4.311)--(7.352,4.311)%
  --(7.373,4.311)--(7.394,4.311)--(7.415,4.311)--(7.436,4.311)--(7.457,4.311)--(7.477,4.311)%
  --(7.498,4.311)--(7.519,4.311)--(7.540,4.311)--(7.561,4.311)--(7.582,4.311)--(7.603,4.311)%
  --(7.624,4.311)--(7.644,4.311)--(7.665,4.311)--(7.686,4.311)--(7.707,4.311)--(7.728,4.311)%
  --(7.749,4.311)--(7.770,4.311)--(7.791,4.311)--(7.812,4.311)--(7.832,4.311)--(7.853,4.311)%
  --(7.874,4.311)--(7.895,4.311)--(7.916,4.311)--(7.937,4.311)--(7.958,4.311)--(7.979,4.311)%
  --(8.000,4.311)--(8.020,4.311)--(8.041,4.311)--(8.062,4.311)--(8.083,4.311)--(8.104,4.311)%
  --(8.125,4.311)--(8.146,4.311)--(8.167,4.311)--(8.188,4.311)--(8.208,4.311)--(8.229,4.311)%
  --(8.250,4.311)--(8.271,4.311)--(8.292,4.311)--(8.313,4.311)--(8.334,4.311)--(8.355,4.311)%
  --(8.375,4.311)--(8.396,4.311)--(8.417,4.311)--(8.438,4.311)--(8.459,4.311)--(8.480,4.311)%
  --(8.501,4.311)--(8.522,4.311)--(8.543,4.311)--(8.563,4.311)--(8.584,4.311)--(8.605,4.311)%
  --(8.626,4.311)--(8.647,4.311)--(8.668,4.311)--(8.689,4.311)--(8.710,4.311)--(8.731,4.311)%
  --(8.751,4.311)--(8.772,4.311)--(8.793,4.311)--(8.814,4.311)--(8.835,4.311)--(8.856,4.311)%
  --(8.877,4.311)--(8.898,4.311)--(8.919,3.832)--(8.939,3.832)--(8.960,3.832)--(8.981,3.832)%
  --(9.002,3.832)--(9.023,3.832)--(9.044,3.832)--(9.065,3.832)--(9.086,3.832)--(9.107,3.832)%
  --(9.127,3.832)--(9.148,3.832)--(9.169,3.832)--(9.190,3.832)--(9.211,3.832)--(9.232,3.832)%
  --(9.253,3.832)--(9.274,3.832)--(9.294,3.832)--(9.315,3.832)--(9.336,3.832)--(9.357,3.832)%
  --(9.378,3.832)--(9.399,3.832)--(9.420,3.832)--(9.441,3.832)--(9.462,3.832)--(9.482,3.832)%
  --(9.503,3.832)--(9.524,3.832)--(9.545,3.832)--(9.566,3.832)--(9.587,3.832)--(9.608,3.832)%
  --(9.629,3.832)--(9.650,3.832)--(9.670,3.832)--(9.691,3.832)--(9.712,3.832)--(9.733,3.832)%
  --(9.754,3.832)--(9.775,3.832)--(9.796,3.832)--(9.817,3.832)--(9.838,3.832)--(9.858,3.832)%
  --(9.879,3.832)--(9.900,3.832)--(9.921,3.832)--(9.942,3.832)--(9.963,3.832)--(9.984,3.832)%
  --(10.005,3.832)--(10.025,3.832)--(10.046,3.832)--(10.067,3.832)--(10.088,3.832)--(10.109,3.832)%
  --(10.130,3.832)--(10.151,3.832)--(10.172,3.832)--(10.193,3.832)--(10.213,3.832)--(10.234,3.832)%
  --(10.255,3.832)--(10.276,3.832)--(10.297,3.832)--(10.318,3.832)--(10.339,3.832)--(10.360,3.832)%
  --(10.381,3.832)--(10.401,3.832)--(10.422,3.832)--(10.443,3.832)--(10.464,3.832)--(10.485,3.832)%
  --(10.506,3.832)--(10.527,3.832)--(10.548,3.832)--(10.569,3.832)--(10.589,3.832)--(10.610,3.832)%
  --(10.631,3.832)--(10.652,3.832)--(10.673,3.832)--(10.694,3.832)--(10.715,3.832)--(10.736,3.832)%
  --(10.756,3.832)--(10.777,3.832)--(10.798,3.832)--(10.819,3.832)--(10.840,3.832)--(10.861,3.832)%
  --(10.882,3.832)--(10.903,3.832)--(10.924,3.832)--(10.944,3.832)--(10.965,3.832)--(10.986,3.832)%
  --(11.007,3.832)--(11.028,3.832);
\gpcolor{color=gp lt color border}
\node[gp node right,font={\fontsize{9pt}{10.8pt}\selectfont}] at (10.479,7.123) {q-desired-1};
\gpcolor{color=gp lt color 3}
\gpsetlinetype{gp lt plot 3}
\draw[gp path] (10.663,7.123)--(11.579,7.123);
\draw[gp path] (1.504,2.473)--(1.525,2.473)--(1.546,2.473)--(1.567,2.473)--(1.588,2.473)%
  --(1.608,2.473)--(1.629,2.473)--(1.650,2.473)--(1.671,2.473)--(1.692,2.473)--(1.713,2.473)%
  --(1.734,3.294)--(1.755,3.294)--(1.776,3.294)--(1.796,3.294)--(1.817,3.294)--(1.838,3.294)%
  --(1.859,3.294)--(1.880,3.294)--(1.901,3.294)--(1.922,3.294)--(1.943,3.294)--(1.963,3.294)%
  --(1.984,3.294)--(2.005,3.551)--(2.026,3.551)--(2.047,3.551)--(2.068,3.551)--(2.089,3.551)%
  --(2.110,3.551)--(2.131,3.551)--(2.151,3.551)--(2.172,3.551)--(2.193,3.551)--(2.214,3.551)%
  --(2.235,3.551)--(2.256,3.551)--(2.277,3.551)--(2.298,3.551)--(2.319,3.854)--(2.339,3.854)%
  --(2.360,3.854)--(2.381,3.854)--(2.402,3.854)--(2.423,3.854)--(2.444,3.854)--(2.465,3.854)%
  --(2.486,3.854)--(2.507,3.854)--(2.527,3.854)--(2.548,3.854)--(2.569,3.854)--(2.590,3.854)%
  --(2.611,3.854)--(2.632,3.854)--(2.653,3.854)--(2.674,3.854)--(2.695,3.854)--(2.715,3.854)%
  --(2.736,3.854)--(2.757,4.672)--(2.778,4.672)--(2.799,4.672)--(2.820,4.672)--(2.841,4.672)%
  --(2.862,4.672)--(2.882,4.672)--(2.903,4.672)--(2.924,4.672)--(2.945,4.672)--(2.966,4.672)%
  --(2.987,4.672)--(3.008,4.672)--(3.029,4.672)--(3.050,4.672)--(3.070,4.672)--(3.091,4.672)%
  --(3.112,4.672)--(3.133,4.672)--(3.154,4.672)--(3.175,4.672)--(3.196,4.672)--(3.217,4.672)%
  --(3.238,4.672)--(3.258,4.672)--(3.279,4.672)--(3.300,4.672)--(3.321,4.672)--(3.342,4.672)%
  --(3.363,4.672)--(3.384,4.672)--(3.405,4.672)--(3.426,4.672)--(3.446,4.672)--(3.467,4.672)%
  --(3.488,4.672)--(3.509,4.672)--(3.530,4.672)--(3.551,4.672)--(3.572,4.672)--(3.593,4.672)%
  --(3.613,4.672)--(3.634,4.672)--(3.655,4.672)--(3.676,4.672)--(3.697,4.672)--(3.718,4.672)%
  --(3.739,4.672)--(3.760,4.672)--(3.781,4.672)--(3.801,4.672)--(3.822,4.672)--(3.843,4.672)%
  --(3.864,4.672)--(3.885,4.672)--(3.906,4.672)--(3.927,4.672)--(3.948,4.672)--(3.969,4.672)%
  --(3.989,4.672)--(4.010,4.672)--(4.031,4.672)--(4.052,4.672)--(4.073,4.672)--(4.094,4.672)%
  --(4.115,4.672)--(4.136,4.672)--(4.157,4.672)--(4.177,4.672)--(4.198,4.672)--(4.219,4.672)%
  --(4.240,4.672)--(4.261,4.672)--(4.282,4.672)--(4.303,4.672)--(4.324,4.672)--(4.344,4.672)%
  --(4.365,4.672)--(4.386,4.672)--(4.407,4.672)--(4.428,4.672)--(4.449,4.672)--(4.470,4.672)%
  --(4.491,4.672)--(4.512,4.672)--(4.532,4.672)--(4.553,4.672)--(4.574,4.672)--(4.595,4.672)%
  --(4.616,4.672)--(4.637,4.672)--(4.658,4.672)--(4.679,4.672)--(4.700,4.672)--(4.720,4.672)%
  --(4.741,4.672)--(4.762,4.672)--(4.783,4.672)--(4.804,4.672)--(4.825,4.672)--(4.846,4.672)%
  --(4.867,4.672)--(4.888,4.672)--(4.908,4.672)--(4.929,4.672)--(4.950,4.672)--(4.971,4.672)%
  --(4.992,4.672)--(5.013,4.672)--(5.034,4.672)--(5.055,4.672)--(5.076,4.672)--(5.096,4.672)%
  --(5.117,4.672)--(5.138,4.672)--(5.159,4.672)--(5.180,4.672)--(5.201,4.672)--(5.222,4.672)%
  --(5.243,4.672)--(5.263,4.485)--(5.284,4.485)--(5.305,4.485)--(5.326,4.485)--(5.347,4.485)%
  --(5.368,4.485)--(5.389,4.485)--(5.410,4.485)--(5.431,4.485)--(5.451,4.485)--(5.472,4.485)%
  --(5.493,4.485)--(5.514,4.485)--(5.535,4.485)--(5.556,4.485)--(5.577,4.485)--(5.598,4.485)%
  --(5.619,4.485)--(5.639,4.485)--(5.660,4.485)--(5.681,4.485)--(5.702,4.694)--(5.723,4.694)%
  --(5.744,4.694)--(5.765,4.694)--(5.786,4.694)--(5.807,4.694)--(5.827,4.694)--(5.848,4.694)%
  --(5.869,4.694)--(5.890,4.694)--(5.911,4.694)--(5.932,4.694)--(5.953,4.694)--(5.974,4.694)%
  --(5.994,4.694)--(6.015,4.694)--(6.036,4.694)--(6.057,4.512)--(6.078,4.512)--(6.099,4.512)%
  --(6.120,4.512)--(6.141,4.512)--(6.162,4.512)--(6.182,4.512)--(6.203,4.512)--(6.224,4.512)%
  --(6.245,4.512)--(6.266,4.512)--(6.287,4.512)--(6.308,4.512)--(6.329,4.512)--(6.350,4.512)%
  --(6.370,4.512)--(6.391,3.723)--(6.412,3.723)--(6.433,3.723)--(6.454,3.723)--(6.475,3.723)%
  --(6.496,3.723)--(6.517,3.723)--(6.538,3.723)--(6.558,3.723)--(6.579,3.723)--(6.600,3.723)%
  --(6.621,3.723)--(6.642,3.723)--(6.663,3.723)--(6.684,3.723)--(6.705,3.723)--(6.726,3.723)%
  --(6.746,3.723)--(6.767,3.723)--(6.788,3.723)--(6.809,3.723)--(6.830,3.723)--(6.851,3.723)%
  --(6.872,3.723)--(6.893,3.723)--(6.913,3.723)--(6.934,3.723)--(6.955,3.723)--(6.976,3.723)%
  --(6.997,3.723)--(7.018,3.723)--(7.039,3.723)--(7.060,3.723)--(7.081,3.723)--(7.101,3.723)%
  --(7.122,3.723)--(7.143,3.723)--(7.164,3.723)--(7.185,3.723)--(7.206,3.723)--(7.227,3.723)%
  --(7.248,3.723)--(7.269,3.723)--(7.289,3.723)--(7.310,3.723)--(7.331,3.723)--(7.352,3.723)%
  --(7.373,3.723)--(7.394,3.723)--(7.415,3.723)--(7.436,3.723)--(7.457,3.723)--(7.477,3.723)%
  --(7.498,3.723)--(7.519,3.723)--(7.540,3.723)--(7.561,3.723)--(7.582,3.723)--(7.603,3.723)%
  --(7.624,3.723)--(7.644,3.723)--(7.665,3.723)--(7.686,3.723)--(7.707,3.723)--(7.728,3.723)%
  --(7.749,3.723)--(7.770,3.723)--(7.791,3.723)--(7.812,3.723)--(7.832,3.723)--(7.853,3.723)%
  --(7.874,3.723)--(7.895,3.723)--(7.916,3.723)--(7.937,3.723)--(7.958,3.723)--(7.979,3.723)%
  --(8.000,3.723)--(8.020,3.723)--(8.041,3.723)--(8.062,3.723)--(8.083,3.723)--(8.104,3.723)%
  --(8.125,3.723)--(8.146,3.723)--(8.167,3.723)--(8.188,3.723)--(8.208,3.723)--(8.229,3.723)%
  --(8.250,3.723)--(8.271,3.723)--(8.292,3.723)--(8.313,3.723)--(8.334,3.723)--(8.355,3.723)%
  --(8.375,3.723)--(8.396,3.723)--(8.417,3.723)--(8.438,3.723)--(8.459,3.723)--(8.480,3.723)%
  --(8.501,3.723)--(8.522,3.723)--(8.543,3.723)--(8.563,3.723)--(8.584,3.723)--(8.605,3.723)%
  --(8.626,3.723)--(8.647,3.723)--(8.668,3.723)--(8.689,3.723)--(8.710,3.723)--(8.731,3.723)%
  --(8.751,3.723)--(8.772,3.723)--(8.793,3.723)--(8.814,3.723)--(8.835,3.723)--(8.856,3.723)%
  --(8.877,3.723)--(8.898,3.723)--(8.919,3.838)--(8.939,3.838)--(8.960,3.838)--(8.981,3.838)%
  --(9.002,3.838)--(9.023,3.838)--(9.044,3.838)--(9.065,3.838)--(9.086,3.838)--(9.107,3.838)%
  --(9.127,3.838)--(9.148,3.838)--(9.169,3.838)--(9.190,3.838)--(9.211,3.838)--(9.232,3.838)%
  --(9.253,3.838)--(9.274,3.838)--(9.294,3.838)--(9.315,3.838)--(9.336,3.838)--(9.357,3.838)%
  --(9.378,3.838)--(9.399,3.838)--(9.420,3.838)--(9.441,3.838)--(9.462,3.838)--(9.482,3.838)%
  --(9.503,3.838)--(9.524,3.838)--(9.545,3.838)--(9.566,3.838)--(9.587,3.838)--(9.608,3.838)%
  --(9.629,3.838)--(9.650,3.838)--(9.670,3.838)--(9.691,3.838)--(9.712,3.838)--(9.733,3.838)%
  --(9.754,3.838)--(9.775,3.838)--(9.796,3.838)--(9.817,3.838)--(9.838,3.838)--(9.858,3.838)%
  --(9.879,3.838)--(9.900,3.838)--(9.921,3.838)--(9.942,3.838)--(9.963,3.838)--(9.984,3.838)%
  --(10.005,3.838)--(10.025,3.838)--(10.046,3.838)--(10.067,3.838)--(10.088,3.838)--(10.109,3.838)%
  --(10.130,3.838)--(10.151,3.838)--(10.172,3.838)--(10.193,3.838)--(10.213,3.838)--(10.234,3.838)%
  --(10.255,3.838)--(10.276,3.838)--(10.297,3.838)--(10.318,3.838)--(10.339,3.838)--(10.360,3.838)%
  --(10.381,3.838)--(10.401,3.838)--(10.422,3.838)--(10.443,3.838)--(10.464,3.838)--(10.485,3.838)%
  --(10.506,3.838)--(10.527,3.838)--(10.548,3.838)--(10.569,3.838)--(10.589,3.838)--(10.610,3.838)%
  --(10.631,3.838)--(10.652,3.838)--(10.673,3.838)--(10.694,3.838)--(10.715,3.838)--(10.736,3.838)%
  --(10.756,3.838)--(10.777,3.838)--(10.798,3.838)--(10.819,3.838)--(10.840,3.838)--(10.861,3.838)%
  --(10.882,3.838)--(10.903,3.838)--(10.924,3.838)--(10.944,3.838)--(10.965,3.838)--(10.986,3.838)%
  --(11.007,3.838)--(11.028,3.838);
\gpcolor{color=gp lt color border}
\node[gp node right,font={\fontsize{9pt}{10.8pt}\selectfont}] at (10.479,6.815) {q-real-2};
\gpcolor{color=gp lt color 4}
\gpsetlinetype{gp lt plot 4}
\draw[gp path] (10.663,6.815)--(11.579,6.815);
\draw[gp path] (1.504,3.972)--(1.525,3.972)--(1.546,3.972)--(1.567,3.972)--(1.588,3.972)%
  --(1.608,3.972)--(1.629,3.972)--(1.650,3.972)--(1.671,3.972)--(1.692,3.972)--(1.713,3.972)%
  --(1.734,3.972)--(1.755,3.331)--(1.776,3.331)--(1.796,3.331)--(1.817,3.331)--(1.838,3.331)%
  --(1.859,3.331)--(1.880,3.331)--(1.901,3.331)--(1.922,3.331)--(1.943,3.331)--(1.963,3.331)%
  --(1.984,3.331)--(2.005,3.331)--(2.026,3.057)--(2.047,3.057)--(2.068,3.057)--(2.089,3.057)%
  --(2.110,3.057)--(2.131,3.057)--(2.151,3.057)--(2.172,3.057)--(2.193,3.057)--(2.214,3.057)%
  --(2.235,3.057)--(2.256,3.057)--(2.277,3.057)--(2.298,3.057)--(2.319,2.948)--(2.339,2.948)%
  --(2.360,2.948)--(2.381,2.948)--(2.402,2.948)--(2.423,2.948)--(2.444,2.948)--(2.465,2.948)%
  --(2.486,2.948)--(2.507,2.948)--(2.527,2.948)--(2.548,2.948)--(2.569,2.948)--(2.590,2.948)%
  --(2.611,2.948)--(2.632,2.948)--(2.653,2.948)--(2.674,2.948)--(2.695,2.948)--(2.715,2.948)%
  --(2.736,2.948)--(2.757,2.659)--(2.778,2.659)--(2.799,2.659)--(2.820,2.659)--(2.841,2.659)%
  --(2.862,2.659)--(2.882,2.659)--(2.903,2.659)--(2.924,2.659)--(2.945,2.659)--(2.966,2.659)%
  --(2.987,2.659)--(3.008,2.659)--(3.029,2.659)--(3.050,2.659)--(3.070,2.659)--(3.091,2.659)%
  --(3.112,2.659)--(3.133,2.659)--(3.154,2.659)--(3.175,2.659)--(3.196,2.659)--(3.217,2.659)%
  --(3.238,2.659)--(3.258,2.659)--(3.279,2.659)--(3.300,2.659)--(3.321,2.659)--(3.342,2.659)%
  --(3.363,2.659)--(3.384,2.659)--(3.405,2.659)--(3.426,2.659)--(3.446,2.659)--(3.467,2.659)%
  --(3.488,2.659)--(3.509,2.659)--(3.530,2.659)--(3.551,2.659)--(3.572,2.659)--(3.593,2.659)%
  --(3.613,2.659)--(3.634,2.659)--(3.655,2.659)--(3.676,2.659)--(3.697,2.659)--(3.718,2.659)%
  --(3.739,2.659)--(3.760,2.659)--(3.781,2.659)--(3.801,2.659)--(3.822,2.659)--(3.843,2.659)%
  --(3.864,2.659)--(3.885,2.659)--(3.906,2.659)--(3.927,2.659)--(3.948,2.659)--(3.969,2.659)%
  --(3.989,2.659)--(4.010,2.659)--(4.031,2.659)--(4.052,2.659)--(4.073,2.659)--(4.094,2.659)%
  --(4.115,2.659)--(4.136,2.659)--(4.157,2.659)--(4.177,2.659)--(4.198,2.659)--(4.219,2.659)%
  --(4.240,2.659)--(4.261,2.659)--(4.282,2.659)--(4.303,2.659)--(4.324,2.659)--(4.344,2.659)%
  --(4.365,2.659)--(4.386,2.659)--(4.407,2.659)--(4.428,2.659)--(4.449,2.659)--(4.470,2.659)%
  --(4.491,2.659)--(4.512,2.659)--(4.532,2.659)--(4.553,2.659)--(4.574,2.659)--(4.595,2.659)%
  --(4.616,2.659)--(4.637,2.659)--(4.658,2.659)--(4.679,2.659)--(4.700,2.659)--(4.720,2.659)%
  --(4.741,2.659)--(4.762,2.659)--(4.783,2.659)--(4.804,2.659)--(4.825,2.659)--(4.846,2.659)%
  --(4.867,2.659)--(4.888,2.659)--(4.908,2.659)--(4.929,2.659)--(4.950,2.659)--(4.971,2.659)%
  --(4.992,2.659)--(5.013,2.659)--(5.034,2.659)--(5.055,2.659)--(5.076,2.659)--(5.096,2.659)%
  --(5.117,2.659)--(5.138,2.659)--(5.159,2.659)--(5.180,2.659)--(5.201,2.659)--(5.222,2.659)%
  --(5.243,2.659)--(5.263,2.659)--(5.284,2.479)--(5.305,2.479)--(5.326,2.479)--(5.347,2.479)%
  --(5.368,2.479)--(5.389,2.479)--(5.410,2.479)--(5.431,2.479)--(5.451,2.479)--(5.472,2.479)%
  --(5.493,2.479)--(5.514,2.479)--(5.535,2.479)--(5.556,2.479)--(5.577,2.479)--(5.598,2.479)%
  --(5.619,2.479)--(5.639,2.479)--(5.660,2.479)--(5.681,2.479)--(5.702,2.479)--(5.723,2.631)%
  --(5.744,2.631)--(5.765,2.631)--(5.786,2.631)--(5.807,2.631)--(5.827,2.631)--(5.848,2.631)%
  --(5.869,2.631)--(5.890,2.631)--(5.911,2.631)--(5.932,2.631)--(5.953,2.631)--(5.974,2.631)%
  --(5.994,2.631)--(6.015,2.631)--(6.036,2.631)--(6.057,2.631)--(6.078,2.552)--(6.099,2.552)%
  --(6.120,2.552)--(6.141,2.552)--(6.162,2.552)--(6.182,2.552)--(6.203,2.552)--(6.224,2.552)%
  --(6.245,2.552)--(6.266,2.552)--(6.287,2.552)--(6.308,2.552)--(6.329,2.552)--(6.350,2.552)%
  --(6.370,2.552)--(6.391,2.317)--(6.412,2.317)--(6.433,2.317)--(6.454,2.317)--(6.475,2.317)%
  --(6.496,2.317)--(6.517,2.317)--(6.538,2.317)--(6.558,2.317)--(6.579,2.317)--(6.600,2.317)%
  --(6.621,2.317)--(6.642,2.317)--(6.663,2.317)--(6.684,2.317)--(6.705,2.317)--(6.726,2.317)%
  --(6.746,2.317)--(6.767,2.317)--(6.788,2.317)--(6.809,2.317)--(6.830,2.317)--(6.851,2.317)%
  --(6.872,2.317)--(6.893,2.317)--(6.913,2.317)--(6.934,2.317)--(6.955,2.317)--(6.976,2.317)%
  --(6.997,2.317)--(7.018,2.317)--(7.039,2.317)--(7.060,2.317)--(7.081,2.317)--(7.101,2.317)%
  --(7.122,2.317)--(7.143,2.317)--(7.164,2.317)--(7.185,2.317)--(7.206,2.317)--(7.227,2.317)%
  --(7.248,2.317)--(7.269,2.317)--(7.289,2.317)--(7.310,2.317)--(7.331,2.317)--(7.352,2.317)%
  --(7.373,2.317)--(7.394,2.317)--(7.415,2.317)--(7.436,2.317)--(7.457,2.317)--(7.477,2.317)%
  --(7.498,2.317)--(7.519,2.317)--(7.540,2.317)--(7.561,2.317)--(7.582,2.317)--(7.603,2.317)%
  --(7.624,2.317)--(7.644,2.317)--(7.665,2.317)--(7.686,2.317)--(7.707,2.317)--(7.728,2.317)%
  --(7.749,2.317)--(7.770,2.317)--(7.791,2.317)--(7.812,2.317)--(7.832,2.317)--(7.853,2.317)%
  --(7.874,2.317)--(7.895,2.317)--(7.916,2.317)--(7.937,2.317)--(7.958,2.317)--(7.979,2.317)%
  --(8.000,2.317)--(8.020,2.317)--(8.041,2.317)--(8.062,2.317)--(8.083,2.317)--(8.104,2.317)%
  --(8.125,2.317)--(8.146,2.317)--(8.167,2.317)--(8.188,2.317)--(8.208,2.317)--(8.229,2.317)%
  --(8.250,2.317)--(8.271,2.317)--(8.292,2.317)--(8.313,2.317)--(8.334,2.317)--(8.355,2.317)%
  --(8.375,2.317)--(8.396,2.317)--(8.417,2.317)--(8.438,2.317)--(8.459,2.317)--(8.480,2.317)%
  --(8.501,2.317)--(8.522,2.317)--(8.543,2.317)--(8.563,2.317)--(8.584,2.317)--(8.605,2.317)%
  --(8.626,2.317)--(8.647,2.317)--(8.668,2.317)--(8.689,2.317)--(8.710,2.317)--(8.731,2.317)%
  --(8.751,2.317)--(8.772,2.317)--(8.793,2.317)--(8.814,2.317)--(8.835,2.317)--(8.856,2.317)%
  --(8.877,2.317)--(8.898,2.317)--(8.919,1.731)--(8.939,1.731)--(8.960,1.731)--(8.981,1.731)%
  --(9.002,1.731)--(9.023,1.731)--(9.044,1.731)--(9.065,1.731)--(9.086,1.731)--(9.107,1.731)%
  --(9.127,1.731)--(9.148,1.731)--(9.169,1.731)--(9.190,1.731)--(9.211,1.731)--(9.232,1.731)%
  --(9.253,1.731)--(9.274,1.731)--(9.294,1.731)--(9.315,1.731)--(9.336,1.731)--(9.357,1.731)%
  --(9.378,1.731)--(9.399,1.731)--(9.420,1.731)--(9.441,1.731)--(9.462,1.731)--(9.482,1.731)%
  --(9.503,1.731)--(9.524,1.731)--(9.545,1.731)--(9.566,1.731)--(9.587,1.731)--(9.608,1.731)%
  --(9.629,1.731)--(9.650,1.731)--(9.670,1.731)--(9.691,1.731)--(9.712,1.731)--(9.733,1.731)%
  --(9.754,1.731)--(9.775,1.731)--(9.796,1.731)--(9.817,1.731)--(9.838,1.731)--(9.858,1.731)%
  --(9.879,1.731)--(9.900,1.731)--(9.921,1.731)--(9.942,1.731)--(9.963,1.731)--(9.984,1.731)%
  --(10.005,1.731)--(10.025,1.731)--(10.046,1.731)--(10.067,1.731)--(10.088,1.731)--(10.109,1.731)%
  --(10.130,1.731)--(10.151,1.731)--(10.172,1.731)--(10.193,1.731)--(10.213,1.731)--(10.234,1.731)%
  --(10.255,1.731)--(10.276,1.731)--(10.297,1.731)--(10.318,1.731)--(10.339,1.731)--(10.360,1.731)%
  --(10.381,1.731)--(10.401,1.731)--(10.422,1.731)--(10.443,1.731)--(10.464,1.731)--(10.485,1.731)%
  --(10.506,1.731)--(10.527,1.731)--(10.548,1.731)--(10.569,1.731)--(10.589,1.731)--(10.610,1.731)%
  --(10.631,1.731)--(10.652,1.731)--(10.673,1.731)--(10.694,1.731)--(10.715,1.731)--(10.736,1.731)%
  --(10.756,1.731)--(10.777,1.731)--(10.798,1.731)--(10.819,1.731)--(10.840,1.731)--(10.861,1.731)%
  --(10.882,1.731)--(10.903,1.731)--(10.924,1.731)--(10.944,1.731)--(10.965,1.731)--(10.986,1.731)%
  --(11.007,1.731)--(11.028,1.731);
\gpcolor{color=gp lt color border}
\node[gp node right,font={\fontsize{9pt}{10.8pt}\selectfont}] at (10.479,6.507) {q-desired-2};
\gpcolor{color=gp lt color 5}
\gpsetlinetype{gp lt plot 5}
\draw[gp path] (10.663,6.507)--(11.579,6.507);
\draw[gp path] (1.504,3.973)--(1.525,3.973)--(1.546,3.973)--(1.567,3.973)--(1.588,3.973)%
  --(1.608,3.973)--(1.629,3.973)--(1.650,3.973)--(1.671,3.973)--(1.692,3.973)--(1.713,3.973)%
  --(1.734,3.063)--(1.755,3.063)--(1.776,3.063)--(1.796,3.063)--(1.817,3.063)--(1.838,3.063)%
  --(1.859,3.063)--(1.880,3.063)--(1.901,3.063)--(1.922,3.063)--(1.943,3.063)--(1.963,3.063)%
  --(1.984,3.063)--(2.005,3.057)--(2.026,3.057)--(2.047,3.057)--(2.068,3.057)--(2.089,3.057)%
  --(2.110,3.057)--(2.131,3.057)--(2.151,3.057)--(2.172,3.057)--(2.193,3.057)--(2.214,3.057)%
  --(2.235,3.057)--(2.256,3.057)--(2.277,3.057)--(2.298,3.057)--(2.319,2.940)--(2.339,2.940)%
  --(2.360,2.940)--(2.381,2.940)--(2.402,2.940)--(2.423,2.940)--(2.444,2.940)--(2.465,2.940)%
  --(2.486,2.940)--(2.507,2.940)--(2.527,2.940)--(2.548,2.940)--(2.569,2.940)--(2.590,2.940)%
  --(2.611,2.940)--(2.632,2.940)--(2.653,2.940)--(2.674,2.940)--(2.695,2.940)--(2.715,2.940)%
  --(2.736,2.940)--(2.757,2.540)--(2.778,2.540)--(2.799,2.540)--(2.820,2.540)--(2.841,2.540)%
  --(2.862,2.540)--(2.882,2.540)--(2.903,2.540)--(2.924,2.540)--(2.945,2.540)--(2.966,2.540)%
  --(2.987,2.540)--(3.008,2.540)--(3.029,2.540)--(3.050,2.540)--(3.070,2.540)--(3.091,2.540)%
  --(3.112,2.540)--(3.133,2.540)--(3.154,2.540)--(3.175,2.540)--(3.196,2.540)--(3.217,2.540)%
  --(3.238,2.540)--(3.258,2.540)--(3.279,2.540)--(3.300,2.540)--(3.321,2.540)--(3.342,2.540)%
  --(3.363,2.540)--(3.384,2.540)--(3.405,2.540)--(3.426,2.540)--(3.446,2.540)--(3.467,2.540)%
  --(3.488,2.540)--(3.509,2.540)--(3.530,2.540)--(3.551,2.540)--(3.572,2.540)--(3.593,2.540)%
  --(3.613,2.540)--(3.634,2.540)--(3.655,2.540)--(3.676,2.540)--(3.697,2.540)--(3.718,2.540)%
  --(3.739,2.540)--(3.760,2.540)--(3.781,2.540)--(3.801,2.540)--(3.822,2.540)--(3.843,2.540)%
  --(3.864,2.540)--(3.885,2.540)--(3.906,2.540)--(3.927,2.540)--(3.948,2.540)--(3.969,2.540)%
  --(3.989,2.540)--(4.010,2.540)--(4.031,2.540)--(4.052,2.540)--(4.073,2.540)--(4.094,2.540)%
  --(4.115,2.540)--(4.136,2.540)--(4.157,2.540)--(4.177,2.540)--(4.198,2.540)--(4.219,2.540)%
  --(4.240,2.540)--(4.261,2.540)--(4.282,2.540)--(4.303,2.540)--(4.324,2.540)--(4.344,2.540)%
  --(4.365,2.540)--(4.386,2.540)--(4.407,2.540)--(4.428,2.540)--(4.449,2.540)--(4.470,2.540)%
  --(4.491,2.540)--(4.512,2.540)--(4.532,2.540)--(4.553,2.540)--(4.574,2.540)--(4.595,2.540)%
  --(4.616,2.540)--(4.637,2.540)--(4.658,2.540)--(4.679,2.540)--(4.700,2.540)--(4.720,2.540)%
  --(4.741,2.540)--(4.762,2.540)--(4.783,2.540)--(4.804,2.540)--(4.825,2.540)--(4.846,2.540)%
  --(4.867,2.540)--(4.888,2.540)--(4.908,2.540)--(4.929,2.540)--(4.950,2.540)--(4.971,2.540)%
  --(4.992,2.540)--(5.013,2.540)--(5.034,2.540)--(5.055,2.540)--(5.076,2.540)--(5.096,2.540)%
  --(5.117,2.540)--(5.138,2.540)--(5.159,2.540)--(5.180,2.540)--(5.201,2.540)--(5.222,2.540)%
  --(5.243,2.540)--(5.263,2.477)--(5.284,2.477)--(5.305,2.477)--(5.326,2.477)--(5.347,2.477)%
  --(5.368,2.477)--(5.389,2.477)--(5.410,2.477)--(5.431,2.477)--(5.451,2.477)--(5.472,2.477)%
  --(5.493,2.477)--(5.514,2.477)--(5.535,2.477)--(5.556,2.477)--(5.577,2.477)--(5.598,2.477)%
  --(5.619,2.477)--(5.639,2.477)--(5.660,2.477)--(5.681,2.477)--(5.702,2.652)--(5.723,2.652)%
  --(5.744,2.652)--(5.765,2.652)--(5.786,2.652)--(5.807,2.652)--(5.827,2.652)--(5.848,2.652)%
  --(5.869,2.652)--(5.890,2.652)--(5.911,2.652)--(5.932,2.652)--(5.953,2.652)--(5.974,2.652)%
  --(5.994,2.652)--(6.015,2.652)--(6.036,2.652)--(6.057,2.549)--(6.078,2.549)--(6.099,2.549)%
  --(6.120,2.549)--(6.141,2.549)--(6.162,2.549)--(6.182,2.549)--(6.203,2.549)--(6.224,2.549)%
  --(6.245,2.549)--(6.266,2.549)--(6.287,2.549)--(6.308,2.549)--(6.329,2.549)--(6.350,2.549)%
  --(6.370,2.549)--(6.391,1.648)--(6.412,1.648)--(6.433,1.648)--(6.454,1.648)--(6.475,1.648)%
  --(6.496,1.648)--(6.517,1.648)--(6.538,1.648)--(6.558,1.648)--(6.579,1.648)--(6.600,1.648)%
  --(6.621,1.648)--(6.642,1.648)--(6.663,1.648)--(6.684,1.648)--(6.705,1.648)--(6.726,1.648)%
  --(6.746,1.648)--(6.767,1.648)--(6.788,1.648)--(6.809,1.648)--(6.830,1.648)--(6.851,1.648)%
  --(6.872,1.648)--(6.893,1.648)--(6.913,1.648)--(6.934,1.648)--(6.955,1.648)--(6.976,1.648)%
  --(6.997,1.648)--(7.018,1.648)--(7.039,1.648)--(7.060,1.648)--(7.081,1.648)--(7.101,1.648)%
  --(7.122,1.648)--(7.143,1.648)--(7.164,1.648)--(7.185,1.648)--(7.206,1.648)--(7.227,1.648)%
  --(7.248,1.648)--(7.269,1.648)--(7.289,1.648)--(7.310,1.648)--(7.331,1.648)--(7.352,1.648)%
  --(7.373,1.648)--(7.394,1.648)--(7.415,1.648)--(7.436,1.648)--(7.457,1.648)--(7.477,1.648)%
  --(7.498,1.648)--(7.519,1.648)--(7.540,1.648)--(7.561,1.648)--(7.582,1.648)--(7.603,1.648)%
  --(7.624,1.648)--(7.644,1.648)--(7.665,1.648)--(7.686,1.648)--(7.707,1.648)--(7.728,1.648)%
  --(7.749,1.648)--(7.770,1.648)--(7.791,1.648)--(7.812,1.648)--(7.832,1.648)--(7.853,1.648)%
  --(7.874,1.648)--(7.895,1.648)--(7.916,1.648)--(7.937,1.648)--(7.958,1.648)--(7.979,1.648)%
  --(8.000,1.648)--(8.020,1.648)--(8.041,1.648)--(8.062,1.648)--(8.083,1.648)--(8.104,1.648)%
  --(8.125,1.648)--(8.146,1.648)--(8.167,1.648)--(8.188,1.648)--(8.208,1.648)--(8.229,1.648)%
  --(8.250,1.648)--(8.271,1.648)--(8.292,1.648)--(8.313,1.648)--(8.334,1.648)--(8.355,1.648)%
  --(8.375,1.648)--(8.396,1.648)--(8.417,1.648)--(8.438,1.648)--(8.459,1.648)--(8.480,1.648)%
  --(8.501,1.648)--(8.522,1.648)--(8.543,1.648)--(8.563,1.648)--(8.584,1.648)--(8.605,1.648)%
  --(8.626,1.648)--(8.647,1.648)--(8.668,1.648)--(8.689,1.648)--(8.710,1.648)--(8.731,1.648)%
  --(8.751,1.648)--(8.772,1.648)--(8.793,1.648)--(8.814,1.648)--(8.835,1.648)--(8.856,1.648)%
  --(8.877,1.648)--(8.898,1.648)--(8.919,1.734)--(8.939,1.734)--(8.960,1.734)--(8.981,1.734)%
  --(9.002,1.734)--(9.023,1.734)--(9.044,1.734)--(9.065,1.734)--(9.086,1.734)--(9.107,1.734)%
  --(9.127,1.734)--(9.148,1.734)--(9.169,1.734)--(9.190,1.734)--(9.211,1.734)--(9.232,1.734)%
  --(9.253,1.734)--(9.274,1.734)--(9.294,1.734)--(9.315,1.734)--(9.336,1.734)--(9.357,1.734)%
  --(9.378,1.734)--(9.399,1.734)--(9.420,1.734)--(9.441,1.734)--(9.462,1.734)--(9.482,1.734)%
  --(9.503,1.734)--(9.524,1.734)--(9.545,1.734)--(9.566,1.734)--(9.587,1.734)--(9.608,1.734)%
  --(9.629,1.734)--(9.650,1.734)--(9.670,1.734)--(9.691,1.734)--(9.712,1.734)--(9.733,1.734)%
  --(9.754,1.734)--(9.775,1.734)--(9.796,1.734)--(9.817,1.734)--(9.838,1.734)--(9.858,1.734)%
  --(9.879,1.734)--(9.900,1.734)--(9.921,1.734)--(9.942,1.734)--(9.963,1.734)--(9.984,1.734)%
  --(10.005,1.734)--(10.025,1.734)--(10.046,1.734)--(10.067,1.734)--(10.088,1.734)--(10.109,1.734)%
  --(10.130,1.734)--(10.151,1.734)--(10.172,1.734)--(10.193,1.734)--(10.213,1.734)--(10.234,1.734)%
  --(10.255,1.734)--(10.276,1.734)--(10.297,1.734)--(10.318,1.734)--(10.339,1.734)--(10.360,1.734)%
  --(10.381,1.734)--(10.401,1.734)--(10.422,1.734)--(10.443,1.734)--(10.464,1.734)--(10.485,1.734)%
  --(10.506,1.734)--(10.527,1.734)--(10.548,1.734)--(10.569,1.734)--(10.589,1.734)--(10.610,1.734)%
  --(10.631,1.734)--(10.652,1.734)--(10.673,1.734)--(10.694,1.734)--(10.715,1.734)--(10.736,1.734)%
  --(10.756,1.734)--(10.777,1.734)--(10.798,1.734)--(10.819,1.734)--(10.840,1.734)--(10.861,1.734)%
  --(10.882,1.734)--(10.903,1.734)--(10.924,1.734)--(10.944,1.734)--(10.965,1.734)--(10.986,1.734)%
  --(11.007,1.734)--(11.028,1.734);
\gpcolor{color=gp lt color border}
\gpsetlinetype{gp lt border}
\draw[gp path] (1.504,8.381)--(1.504,0.985)--(11.947,0.985)--(11.947,8.381)--cycle;
%% coordinates of the plot area
\gpdefrectangularnode{gp plot 1}{\pgfpoint{1.504cm}{0.985cm}}{\pgfpoint{11.947cm}{8.381cm}}
\end{tikzpicture}
%% gnuplot variables

%         }
%         \caption{Real configuration as reported by the PA10 controller versus the desired configuration for the first 3 joints.}
%         \label{fig:q_real_desired}
%     \end{subfigure}~
%     \begin{subfigure}[b]{0.49\textwidth}
%         \resizebox{\columnwidth}{!}{%
%             \input{figures/q_real_desired_2_tikz}
%         }
%         \caption{Real configuration as reported by the PA10 controller versus the desired configuration for the first 3 joints.}
%         \label{fig:q_real_desired_2}
%     \end{subfigure}
%     \caption{asdf}
%     \label{fig:asdf}
% \end{figure}

% \begin{figure}[htb]
%     \centering
%     \resizebox{\columnwidth}{!}{%
%         \input{figures/q_real_desired_2_tikz}
%     }
%     \caption{Real configuration as reported by the PA10 controller versus the desired configuration for the first 3 joints.}
%     \label{fig:q_real_desired_2}
% \end{figure}
