%!TEX root = ../report.tex

\chapter{Introduction}
\label{chap:introduction}

\section{Overall description}
\label{sec:overall_description}
The presented report contains the description of the  project ``Path planning for paths that keep
object in sight'' developed as a part of the course RoVi2: Robotics and Computer Vision 2.
The aim of the designed system is to enable the existing setup, consisting of a PA10 robotic arm with a tool mounted stereo vision rig, to safely plan the required three-dimensional trajectories for keeping a specific user-controlled target object within the field of view of its cameras at any moment.

\section{Report structure}
\label{sec:report_structure}
The project has been divided into two well differentiated parts. On one side the computer vision part, under which the design of the marker to track and its recognition, and modeling of its physical behavior is treated.
On the other side the robotics part, which gathers all the processes in charge of the trajectory planning and collision detection for the robot motion.

The report is structured as follows:
For better understanding, and as guide, a the \emph{ROS nodes structure} (Chap.~\ref{chap:ros_nodes_structure}) is presented. The report continues with \emph{Feature extraction} (Chap.~\ref{chap:feature_extraction}) and \emph{Stereopsis} (Chap.~\ref{chap:stereopsis}).
Then, the Kalman filter is introduced in \emph{Prediction} (Chap.~\ref{chap:prediction}) and in \emph{Path planning} (Chap.~\ref{chap:path_planning}) how the robot moves to keep the object in sight is explained.
Finish with the \emph{Experiments} (Chap.~\ref{cha:experiments}), the \emph{Discussion} (Chap.~\ref{chap:discussion}) and ends up with the \emph{Conclusions} (Chap.~\ref{cha:conclusions}).
